\begin{problem}{1}
Considera las siguientes matrices de covarianzas de un vector aleatorio X. Obtener la matriz de coeficiente de correlaciones para cada caso. Después, interpreta los valores obtenidos.
\begin{align*}
A) \quad \mathbf{\Sigma} &=
\begin{bmatrix}
4&3&-2\\
3&6&2\\
-2&2&5
\end{bmatrix}\\
B) \quad \mathbf{\Sigma} &=
\begin{bmatrix}
10&-3&-1&5\\
-3&8&3&0\\
-1&3&15&1\\
5&0&1&4
\end{bmatrix}
\end{align*}

\end{problem}
\begin{sol}
Sabemos que:
\begin{align*}
\rho_{ij} = \frac{\sigma_{ij}}{\sqrt{\sigma_{ii}\sigma_{jj}}}
\end{align*}
Usaremos $\mathbf{\Sigma}$ para sacar los valores, ya que para A:
\begin{align*}
\mathbf{\Sigma} &= 
\begin{pmatrix}
\sigma_{11} & \sigma_{12} & \sigma_{13} \\ 
\sigma_{21} & \sigma_{22} & \sigma_{23} \\ 
\sigma_{31} & \sigma_{32} & \sigma_{33} \\ 
\end{pmatrix}
\end{align*}
Entonces tenemos que :
\begin{align*}
\sigma_{11} &= 4\\
\sigma_{12} &= 3\\
\sigma_{13} &= -2\\
\sigma_{22} &= 6\\
\sigma_{23} &= 2\\
\sigma_{33} &= 5\\
\end{align*}
Nos queda nuestra matriz:
\begin{align*}
\rho &= \begin{pmatrix}
\frac{4}{\sqrt{4\cdot 4}} & \frac{3}{\sqrt{4 \cdot 6}} & -\frac{2}{\sqrt{4 \cdot 5}} \\
\frac{3}{\sqrt{6 \cdot 4}} & \frac{6}{\sqrt{6\cdot 6}} & \frac{2}{\sqrt{6 \cdot 5}} \\
-\frac{2}{\sqrt{5 \cdot 4}} & \frac{2}{\sqrt{5 \cdot 6}} & \frac{5}{\sqrt{5\cdot 5}}
\end{pmatrix}\\
\rho &= \begin{pmatrix}
1 & \frac{3}{2\sqrt{6}} & -\frac{1}{\sqrt{5}} \\
\frac{3}{2\sqrt{6}} & 1 & \frac{2}{\sqrt{30}} \\
-\frac{1}{\sqrt{5}} & \frac{2}{\sqrt{30}} & 1
\end{pmatrix}\\
\rho &= 
\begin{pmatrix}
1 & 0.6123 & -0.4472 \\
0.6123 & 1 & 0.3651 \\
-0.4472 &  0.3651 & 1
\end{pmatrix}
\end{align*}
Observamos que $\rho_{12} = 0.6123$ esto significa que hay correlación positiva moderada entre la primera y segunda entrada del vector aleatorio, para $\rho_{13} = -0.4472$ hay una correlación debil negativa entre la primera y tercera entrada del vector aleatorio, finalmente, para $\rho_{23} = 0.3651$ hay una correlación debil positiva entre la segunda y tercera entrada del vector aleatorio.\\\\
Para B) tenemos que :
\begin{align*}
\sigma_{11} &= 10\\
\sigma_{12} &= -3\\
\sigma_{13} &= -1\\
\sigma_{14} &= 5\\
\sigma_{22} &= 8\\
\sigma_{23} &= 3\\
\sigma_{24} &= 0\\
\sigma_{33} &= 15\\
\sigma_{34} &= 1\\
\sigma_{44} &= 4\\
\end{align*}
Nos queda nuestra matriz: 
\begin{align*}
\rho &=
\begin{pmatrix}
\frac{10}{\sqrt{10 \cdot 10}} & -\frac{3}{\sqrt{10 \cdot 8}} & -\frac{1}{\sqrt{10 \cdot 15}} & \frac{5}{\sqrt{10 \cdot 4}} \\
-\frac{3}{\sqrt{10 \cdot 8}} & \frac{8}{\sqrt{8 \cdot 8}} & \frac{3}{\sqrt{8 \cdot 15}} & 0 \\
 -\frac{1}{\sqrt{10 \cdot 15}} & \frac{3}{\sqrt{8 \cdot 15}} & \frac{15}{\sqrt{15 \cdot 15}} & \frac{1}{\sqrt{15 \cdot 4}} \\
 \frac{5}{\sqrt{10 \cdot 4}} &0 &\frac{1}{\sqrt{15 \cdot 4}}  & \frac{4}{\sqrt{4 \cdot 4}} \\
\end{pmatrix}\\
\rho &= 
\begin{pmatrix}
\frac{10}{10} & -\frac{3}{\sqrt{80}}& -\frac{1}{\sqrt{150}}& \frac{5}{\sqrt{40}} \\
-\frac{3}{\sqrt{80}} & \frac{8}{8}& \frac{3}{\sqrt{120}}& 0\\
-\frac{1}{\sqrt{150}} & \frac{3}{\sqrt{120}}& \frac{15}{15}& \frac{1}{\sqrt{60}}\\
\frac{5}{\sqrt{40}} & 0& \frac{1}{\sqrt{60}}& \frac{4}{4}\\
\end{pmatrix}\\
\rho &=
\begin{pmatrix}
1 & -0.3354 & -0.0816 & 0.7905 \\ 
-0.3354 & 1 & 0.2738 & 0 \\ 
-0.0816 & 0.2738 & 1 & 0.1290 \\
0.7905 & 0 & 0.1290 & 1
\end{pmatrix} 
\end{align*}
Observamos que $\rho_{12} = -0.3354$ tiene una correlación negativa débil entre la primer y segunda entrada del vector aleatorio, $\rho_{13}=-0.0816$ hay correlación negativa extremadamente débil entre la primera y tercera entrada del vector aleatorio, $\rho_{14}=0.7905$ hay correlación positiva fuerte entre la primera y cuarta entrada del vector aleatorio, $\rho_{23}=0.2738$ hay correlación débil positiva entre la segunda y tercera entrada del vector aleatorio, $\rho_{24}=0$ no hay correlación entre la segunda y cuarta entrada del vector aleatorio, $\rho_{34}=0.1290$ hay correlación positiva débil entre la tercera y cuarta entrada del vector aleatorio.
\end{sol}