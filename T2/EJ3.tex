\begin{problem}{2}
Para las matrices del problema 1, obtener $\mathbf{V}^{\frac{1}{2}},\mathbf{V}^{-\frac{1}{2}} $ y verificar que: $\mathbf{\Sigma}=\mathbf{V^{\frac{1}{2}} \rho V^{\frac{1}{2}}}$ y $\mathbf{\rho}=\mathbf{V^{-\frac{1}{2}} \Sigma V^{-\frac{1}{2}}}$
\end{problem}

\begin{sol}
Sabemos que
\begin{align*}
\mathbf{V}^{\frac{1}{2}}=
\begin{pmatrix}
\sqrt{\sigma_{11}} & 0 & \cdots & 0 \\
0 & \sqrt{\sigma_{22}} & \cdots & 0 \\
\vdots & \vdots & \ddots & \vdots \\
0 & 0 & \cdots & \sqrt{\sigma_{pp}}
\end{pmatrix}
\end{align*}
Por tanto, para A):
\begin{align*}
\mathbf{V}^{\frac{1}{2}} &=
\begin{pmatrix}
\sqrt{4} & 0 & 0 \\
0 & \sqrt{6} & 0 \\
0 & 0 & \sqrt{5}
\end{pmatrix}\\
\end{align*}
Usando Gauss-Jordan:
\begin{align*}
(\mathbf{V}^{\frac{1}{2}})^{-1} &:
\begin{pmatrix}
\sqrt{4} & 0 & 0 & \vert &1 & 0 & 0 \\
0 & \sqrt{6} & 0 & \vert &0 & 1 & 0 \\
0 & 0 & \sqrt{5} & \vert & 0 & 0 & 1 
\end{pmatrix}\\
(\mathbf{V}^{\frac{1}{2}})^{-1} &:
\begin{pmatrix}
1 & 0 & 0 & \vert &\frac{1}{\sqrt{4}} & 0 & 0 \\
0 & 1 & 0 & \vert &0 &\frac{1}{\sqrt{6}} & 0 \\
0 & 0 & 1 & \vert & 0 & 0 & \frac{1}{\sqrt{5}}
\end{pmatrix}\\
(\mathbf{V}^{\frac{1}{2}})^{-1} &= 
\begin{pmatrix}
\frac{1}{\sqrt{4}} & 0 & 0 \\
0 &\frac{1}{\sqrt{6}} & 0 \\
 0 & 0 & \frac{1}{\sqrt{5}}
\end{pmatrix}\\
\end{align*}
Verificaremos $\mathbf{\Sigma}=\mathbf{V^{\frac{1}{2}} \rho V^{\frac{1}{2}}}$:
\begin{align*}
\begin{pmatrix}
\sqrt{4} & 0 & 0 \\
0 & \sqrt{6} & 0 \\
0 & 0 & \sqrt{5}
\end{pmatrix}
\begin{pmatrix}
\frac{4}{\sqrt{4\cdot 4}} & \frac{3}{\sqrt{4 \cdot 6}} & -\frac{2}{\sqrt{4 \cdot 5}} \\
\frac{3}{\sqrt{6 \cdot 4}} & \frac{6}{\sqrt{6\cdot 6}} & \frac{2}{\sqrt{6 \cdot 5}} \\
-\frac{2}{\sqrt{5 \cdot 4}} & \frac{2}{\sqrt{5 \cdot 6}} & \frac{5}{\sqrt{5\cdot 5}}
\end{pmatrix}                                                                                                                                                                                                       
\begin{pmatrix}
\sqrt{4} & 0 & 0 \\
0 & \sqrt{6} & 0 \\
0 & 0 & \sqrt{5}
\end{pmatrix} &= 
\begin{pmatrix}
\frac{4}{\sqrt{4}} & \frac{3}{\sqrt{6}} & -\frac{2}{\sqrt{5}} \\
\frac{3}{\sqrt{4}} & \frac{6}{\sqrt{6}} & \frac{2}{\sqrt{5}} \\
-\frac{2}{\sqrt{4}} & \frac{2}{\sqrt{6}} & \frac{5}{\sqrt{5}}
\end{pmatrix}
\begin{pmatrix}
\sqrt{4} & 0 & 0 \\
0 & \sqrt{6} & 0 \\
0 & 0 & \sqrt{5}
\end{pmatrix}\\
&=\begin{pmatrix}
4&3&-2\\
3&6&2\\
-2&2&5
\end{pmatrix}
\end{align*}
Observamos que es igual al $\mathbf{\Sigma}$ del ejercicio 1. Ahora verificaremos $\mathbf{\rho}=\mathbf{V^{-\frac{1}{2}} \Sigma V^{-\frac{1}{2}}}$
\begin{align*}
\rho &= \begin{pmatrix}
\frac{1}{\sqrt{4}} & 0 & 0 \\
0 &\frac{1}{\sqrt{6}} & 0 \\
 0 & 0 & \frac{1}{\sqrt{5}}
\end{pmatrix}
\begin{pmatrix}
4&3&-2\\
3&6&2\\
-2&2&5
\end{pmatrix}
\begin{pmatrix}
\frac{1}{\sqrt{4}} & 0 & 0 \\
0 &\frac{1}{\sqrt{6}} & 0 \\
 0 & 0 & \frac{1}{\sqrt{5}}
\end{pmatrix}\\
&= \begin{pmatrix}
\frac{4}{\sqrt{4}} & \frac{3}{\sqrt{4}} & \frac{-2}{\sqrt{4}} \\
\frac{3}{\sqrt{6}} & \frac{6}{\sqrt{6}} & \frac{2}{\sqrt{6}} \\
\frac{-2}{\sqrt{5}} & \frac{2}{\sqrt{5}} & \frac{5}{\sqrt{5}}
\end{pmatrix}
\begin{pmatrix}
\frac{1}{\sqrt{4}} & 0 & 0 \\
0 &\frac{1}{\sqrt{6}} & 0 \\
 0 & 0 & \frac{1}{\sqrt{5}}
\end{pmatrix}\\
&= \begin{pmatrix}
\frac{4}{4} & \frac{3}{\sqrt{24}} & -\frac{2}{\sqrt{20}} \\
\frac{3}{\sqrt{24}} & \frac{6}{6} & \frac{2}{\sqrt{30}} \\
-\frac{2}{\sqrt{20}} & \frac{2}{\sqrt{30}} & \frac{5}{5}
\end{pmatrix}\\
\rho &= \begin{pmatrix}
1 & \frac{3}{2\sqrt{6}} & -\frac{1}{\sqrt{5}} \\
\frac{3}{2\sqrt{6}} & 1 & \frac{2}{\sqrt{30}} \\
-\frac{1}{\sqrt{5}} & \frac{2}{\sqrt{30}} & 1
\end{pmatrix}\\
\rho &= 
\begin{pmatrix}
1 & 0.6123 & -0.4472 \\
0.6123 & 1 & 0.3651 \\
-0.4472 &  0.3651 & 1
\end{pmatrix}
\end{align*}
Observamos que es igual al $\rho$ que obtuvimos en el ejercicio 1.\\\\
Ahora para B):
\begin{align*}\mathbf{V}^{\frac{1}{2}} &=
\begin{pmatrix}
\sqrt{10} & 0 & 0 & 0\\
0 & \sqrt{8} & 0 & 0\\
0 & 0 & \sqrt{15} & 0\\
0 &0 &0 & \sqrt{4}
\end{pmatrix}\\ 
(\mathbf{V}^{\frac{1}{2}})^{-1} &= 
\begin{pmatrix}
\frac{1}{\sqrt{10}} & 0 & 0 & 0\\
0 &\frac{1}{\sqrt{8}}  & 0 & 0\\
0 & 0 & \frac{1}{\sqrt{15}}  & 0\\
0 &0 &0 & \frac{1}{\sqrt{4}} 
\end{pmatrix}\\ 
\end{align*}
Verificaremos $\mathbf{\Sigma}=\mathbf{V^{\frac{1}{2}} \rho V^{\frac{1}{2}}}$:
\begin{align*}
&=\begin{pmatrix}
\sqrt{10} & 0 & 0 & 0\\
0 & \sqrt{8} & 0 & 0\\
0 & 0 & \sqrt{15} & 0\\
0 &0 &0 & \sqrt{4}
\end{pmatrix}
\begin{pmatrix}
\frac{10}{\sqrt{10 \cdot 10}} & -\frac{3}{\sqrt{10 \cdot 8}} & -\frac{1}{\sqrt{10 \cdot 15}} & \frac{5}{\sqrt{10 \cdot 4}} \\
-\frac{3}{\sqrt{10 \cdot 8}} & \frac{8}{\sqrt{8 \cdot 8}} & \frac{3}{\sqrt{8 \cdot 15}} & 0 \\
 -\frac{1}{\sqrt{10 \cdot 15}} & \frac{3}{\sqrt{8 \cdot 15}} & \frac{15}{\sqrt{15 \cdot 15}} & \frac{1}{\sqrt{15 \cdot 4}} \\
 \frac{5}{\sqrt{10 \cdot 4}} &0 &\frac{1}{\sqrt{15 \cdot 4}}  & \frac{4}{\sqrt{4 \cdot 4}} \\
\end{pmatrix}
\begin{pmatrix}
\sqrt{10} & 0 & 0 & 0\\
0 & \sqrt{8} & 0 & 0\\
0 & 0 & \sqrt{15} & 0\\
0 &0 &0 & \sqrt{4}
\end{pmatrix} \\
&= 
\begin{pmatrix}
\frac{10}{\sqrt{10}} & -\frac{3}{\sqrt{8}} & -\frac{1}{\sqrt{15}} & \frac{5}{\sqrt{4}} \\
-\frac{3}{\sqrt{10}} & \frac{8}{\sqrt{8}} & \frac{3}{\sqrt{15}} & 0 \\
-\frac{1}{\sqrt{10}} & \frac{3}{\sqrt{8}} & \frac{15}{\sqrt{15}} & \frac{1}{\sqrt{4}} \\
\frac{5}{\sqrt{10}} & 0 & \frac{1}{\sqrt{15}} & \frac{4}{\sqrt{4}} \\
\end{pmatrix} 
\begin{pmatrix}
\sqrt{10} & 0 & 0 & 0\\
0 & \sqrt{8} & 0 & 0\\
0 & 0 & \sqrt{15} & 0\\
0 &0 &0 & \sqrt{4}
\end{pmatrix}\\
&=\begin{pmatrix}
10&-3&-1&5\\
-3&8&3&0\\
-1&3&15&1\\
5&0&1&4
\end{pmatrix}
\end{align*}
Observamos que es igual al $\mathbf{\Sigma}$ del ejercicio 1. Ahora verificaremos $\mathbf{\rho}=\mathbf{V^{-\frac{1}{2}} \Sigma V^{-\frac{1}{2}}}$:
\begin{align*}
\rho &= \begin{pmatrix}
\frac{1}{\sqrt{10}} & 0 & 0 & 0\\
0 &\frac{1}{\sqrt{8}}  & 0 & 0\\
0 & 0 & \frac{1}{\sqrt{15}}  & 0\\
0 &0 &0 & \frac{1}{\sqrt{4}} 
\end{pmatrix}
\begin{pmatrix}
10&-3&-1&5\\
-3&8&3&0\\
-1&3&15&1\\
5&0&1&4
\end{pmatrix}
\begin{pmatrix}
\frac{1}{\sqrt{10}} & 0 & 0 & 0\\
0 &\frac{1}{\sqrt{8}}  & 0 & 0\\
0 & 0 & \frac{1}{\sqrt{15}}  & 0\\
0 &0 &0 & \frac{1}{\sqrt{4}} 
\end{pmatrix} \\
&= 
\begin{pmatrix}
\frac{10}{\sqrt{10}} & -\frac{3}{\sqrt{10}} & -\frac{1}{\sqrt{10}} & \frac{5}{\sqrt{10}} \\
-\frac{3}{\sqrt{8}} & \frac{8}{\sqrt{8}} & \frac{3}{\sqrt{8}} & 0 \\
-\frac{1}{\sqrt{15}} & \frac{3}{\sqrt{15}} & \frac{15}{\sqrt{15}} & \frac{1}{\sqrt{15}} \\
\frac{5}{\sqrt{4}} & 0 & \frac{1}{\sqrt{4}} & \frac{4}{\sqrt{4}} \\
\end{pmatrix}
\begin{pmatrix}
\frac{1}{\sqrt{10}} & 0 & 0 & 0\\
0 &\frac{1}{\sqrt{8}}  & 0 & 0\\
0 & 0 & \frac{1}{\sqrt{15}}  & 0\\
0 &0 &0 & \frac{1}{\sqrt{4}} 
\end{pmatrix} \\
\rho &= 
\begin{pmatrix}
\frac{10}{10} & -\frac{3}{\sqrt{80}}& -\frac{1}{\sqrt{150}}& \frac{5}{\sqrt{40}} \\
-\frac{3}{\sqrt{80}} & \frac{8}{8}& \frac{3}{\sqrt{120}}& 0\\
-\frac{1}{\sqrt{150}} & \frac{3}{\sqrt{120}}& \frac{15}{15}& \frac{1}{\sqrt{60}}\\
\frac{5}{\sqrt{40}} & 0& \frac{1}{\sqrt{60}}& \frac{4}{4}\\
\end{pmatrix} \\
\rho &=
\begin{pmatrix}
1 & -0.3354 & -0.0816 & 0.7905 \\ 
-0.3354 & 1 & 0.2738 & 0 \\ 
-0.0816 & 0.2738 & 1 & 0.1290 \\
0.7905 & 0 & 0.1290 & 1
\end{pmatrix} 
\end{align*}
Observamos que es igual al $\rho$ que obtuvimos en el ejercicio 1.\\\\
\end{sol}