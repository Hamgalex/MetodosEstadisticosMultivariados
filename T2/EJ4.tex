\begin{problem}{4}
Considera que $\mathbf{\Sigma}= \begin{bmatrix}
12&5&12\\
5&16&15\\
12&15&20
\end{bmatrix}$. Obtener la matriz de covarianzas del vector aleatorio \textbf{Z} donde:\\
A) $Z_1=3X_1-2X_2+5X_3, Z_2=9X_1+6X_2-8X_3,Z_3=4X_1+X_2-X_3$\\
B) $Z_1=5X_1-2X_2+9X_3, Z_2=X_1+X_2-X_3,Z_3=3X_1+X_2-2X_3,Z_4=4X_1+X_2$
\end{problem}
\begin{sol}
Para A) tenemos:
\begin{align*}
\mathbf{Z}=\begin{pmatrix}
3X_1-2X_2+5X_3\\
9X_1+6X_2-8X_3\\
4X_1+X_2-X_3
\end{pmatrix} =
\begin{pmatrix}
3 & -2 & 5\\
9 & 6 & -8 \\
4 & 1 &-1
\end{pmatrix}
\begin{pmatrix}
X_1\\X_2\\X_3
\end{pmatrix} = \mathbf{CX}
\end{align*}
Sabemos que Cov$(\mathbf{Z}) = \mathbf{C \Sigma_X C'}$ esto es:
\begin{align*}
\mathbf{C \Sigma_X C'} &= \begin{pmatrix}
3 & -2 & 5\\
9 & 6 & -8 \\
4 & 1 &-1
\end{pmatrix}
\begin{pmatrix}
12&5&12\\
5&16&15\\
12&15&20
\end{pmatrix}
\begin{pmatrix}
3 & 9 & 4\\
-2 & 6 & 1 \\
5 & -8 &-1
\end{pmatrix}\\
\end{align*}
Usando el siguiente código de multiplicación de matrices en RStudio:
\begin{verbatim}
C <- matrix(c(3, -2, 5, 9, 6, -8, 4, 1, -1), nrow = 3, byrow = TRUE)
SIGMA <- matrix(c(12, 5, 12, 5, 16, 15, 12, 15, 20), nrow = 3, byrow = TRUE)
CT <- matrix(c(3, 9, 4, -2, 6, 1, 5, -8, -1), nrow = 3, byrow = TRUE)

resultado_CSIGMA <- C %*% SIGMA
resultado <- resultado_CSIGMA %*% CT
\end{verbatim}
Nos resulta:
\begin{align*}
\mathbf{C \Sigma_X C'} &= \begin{pmatrix}
3 & -2 & 5\\
9 & 6 & -8 \\
4 & 1 &-1
\end{pmatrix}
\begin{pmatrix}
12&5&12\\
5&16&15\\
12&15&20
\end{pmatrix}
\begin{pmatrix}
3 & 9 & 4\\
-2 & 6 & 1 \\
5 & -8 &-1
\end{pmatrix}\\
&= 
\begin{pmatrix}
86 & 58 & 106 \\
42 & 21 & 38 \\
41 & 21 & 43
\end{pmatrix}
\begin{pmatrix}
3 & 9 & 4\\
-2 & 6 & 1 \\
5 & -8 &-1
\end{pmatrix} \\
\text{Cov}(\mathbf{Z})&= 
\begin{pmatrix}
672 & 274 & 296 \\
274 & 200 &151 \\
296 &151 & 142
\end{pmatrix}
\end{align*}
\pagebreak

Para B) tenemos:
\begin{align*}
\mathbf{Z}=\begin{pmatrix}
5X_1-2X_2+9X_3\\
X_1+X_2-X_3\\
3X_1+X_2-2X_3\\
4X_1+X_2
\end{pmatrix}=
\begin{pmatrix}
5 &-2 &9 \\
1 & 1 &-1\\
3 & 1 &-2\\
4 & 1 & 0
\end{pmatrix}
\begin{pmatrix}
X_1\\X_2\\X_3
\end{pmatrix}
= \mathbf{CX}
\end{align*}
Sabemos que Cov$(\mathbf{Z}) = \mathbf{C \Sigma_X C'}$ esto es:
\begin{align*}
\mathbf{C \Sigma_X C'} &=
\begin{pmatrix}
5 &-2 &9 \\
1 & 1 &-1\\
3 & 1 &-2\\
4 & 1 & 0
\end{pmatrix}
\begin{pmatrix}
12&5&12\\
5&16&15\\
12&15&20
\end{pmatrix}
\begin{pmatrix}
5 & 1&3 &4\\
-2& 1&1&1\\
9 &-1 & -2 &0
\end{pmatrix}
\end{align*}
Usando el siguiente código de multiplicación de matrices en RStudio:
\begin{verbatim}
C <- matrix(c(5, -2, 9, 1, 1, -1, 3, 1, -2, 4, 1, 0), nrow = 4, byrow = TRUE)
SIGMA <- matrix(c(12, 5, 12, 5, 16, 15, 12, 15, 20), nrow = 3, byrow = TRUE)
CT <- matrix(c(5, 1, 3, 4, -2, 1, 1, 1, 9, -1, -2, 0), nrow = 3, byrow = TRUE)

resultado_CSIGMA <- C %*% SIGMA
resultado <- resultado_CSIGMA %*% CT
\end{verbatim}
Nos resulta:
\begin{align*}
\mathbf{C \Sigma_X C'} &= \begin{pmatrix}
5 &-2 &9 \\
1 & 1 &-1\\
3 & 1 &-2\\
4 & 1 & 0
\end{pmatrix}
\begin{pmatrix}
12&5&12\\
5&16&15\\
12&15&20
\end{pmatrix}
\begin{pmatrix}
5 & 1&3 &4\\
-2& 1&1&1\\
9 &-1 & -2 &0
\end{pmatrix}\\
&= 
\begin{pmatrix}
158 & 128 & 210 \\
5&6&7\\
17&1&11\\
53 & 36 & 63
\end{pmatrix} 
\begin{pmatrix}
5 & 1&3 &4\\
-2& 1&1&1\\
9 &-1 & -2 &0
\end{pmatrix}\\
\text{Cov}(\mathbf{Z})&= 
\begin{pmatrix}
2424 & 76 & 182 & 760 \\
76&4&7&26 \\
182 & 7 &30 &69\\
760 & 26 &69 &248
\end{pmatrix}
\end{align*}
\end{sol}
