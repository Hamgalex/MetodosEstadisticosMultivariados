\begin{problem}{8}
Considera la muestra aleatoria del excel. Obtener $\mathbf{\bar{X}}$ y $\mathbf{S}$. Interpreta los valores obtenidos.
\end{problem}
\begin{sol}
Para esto usaremos R para leer los datos, y de ahí usar funciones para obtener el vector de medias muestrales y la matriz de covarianza muestral:
\begin{verbatim}
datos <- read_excel("C:/Users/hamga/Downloads/datos_tarea2.xlsx")
media_muestral <- colMeans(datos)
matriz_covarianzas <- cov(datos) 
\end{verbatim}
Esto nos arroja los siguientes resultados:
\begin{align*}
\mathbf{\bar{X}} &= 
\begin{pmatrix}
91.218 \\
101.9279\\
149.2095 \\
300.4512 \\
501.207
\end{pmatrix} \\
\mathbf{S} &=
\begin{pmatrix}
27.08847 & 43.02910 & 26.50251 & 13.00336 & 17.40344 \\
43.02910 & 109.93837 & 35.67109 & 49.54382 & 53.91932 \\
26.50251 & 35.67109 & 103.32817 & 33.26684 & 36.49187 \\
13.00336 & 49.54382 & 33.26684 & 80.25960 & 22.00554 \\
17.40344 & 53.91932 & 36.49187 & 22.00554 & 70.95242
\end{pmatrix}
\end{align*}
De estos resultados podemos concluir varias cosas:
\begin{itemize}
\item La variable que mas varía es la $X_2$ pues su varianza de $109.9383$ es la mayor en toda la matriz.
\item La relación más débil es entre las variables $X_4$ y $X_5$, pues es de $22.0055$
\item La relación mas fuerte es entre las variables $X_2$ y $X_5$ pues es de $53.9193$
\end{itemize}
\end{sol}