\begin{problem}{7}
Sea \textbf{X} un vector aleatorio con distribución $N_5(\mathbf{\mu,\Sigma})$, obtener:
\begin{itemize}
\item A) P\{$\mathbf{(X-\mu)'\Sigma^{-1}(X-\mu)}>3$ \}
\item B) P\{$4<\mathbf{(X-\mu)'\Sigma^{-1}(X-\mu)}<6$ \}
\item C) P\{$\mathbf{(X-\mu)'\Sigma^{-1}(X-\mu)}<2$ \}
\item D) Obtener el valor de $w$ tal que P\{$\mathbf{(X-\mu)'\Sigma^{-1}(X-\mu)}<w$ \} $=0.85$
\end{itemize}
\end{problem}
\begin{sol}
\begin{itemize}
\item A) Las propiedades de la distribución normal multivariada $N_p(\mathbf{\mu,\Sigma})$ nos dicen que la variable aleatoria $\mathbf{(x-\mu)'\Sigma (x-\mu)}$ tiene una distribución $\chi_p^2$, en este caso, $p=5$, por lo que calcularemos:
\begin{align*}
 P\{\mathbf{(X-\mu)'\Sigma^{-1}(X-\mu)}>3 \} = P \{\chi_5^2>3\} = 1 - P \{\chi_5^2<3 \}
\end{align*}
Usando la función de distribución acumulativa de chi cuadrada en R:
\begin{verbatim}
1 - pchisq(3, 5)
\end{verbatim}
Nos resulta $ P\{\mathbf{(X-\mu)'\Sigma^{-1}(X-\mu)}>3 \} = 1 - P \{\chi_5^2<3 \}=0.6999$
\item B)
\begin{align*}
P\{4<\mathbf{(X-\mu)'\Sigma^{-1}(X-\mu)}<6 \} =  P\{4<\chi_5^2<6 \}
\end{align*}
Usando la función en R:
\begin{verbatim}
pchisq(6, 5) - pchisq (4,5)
\end{verbatim}
Nos queda:
\begin{align*}
P\{4<\mathbf{(X-\mu)'\Sigma^{-1}(X-\mu)}<6 \} =  P\{4<\chi_5^2<6 \} = 0.2431
\end{align*}
\item C)
\begin{align*}
P\{\mathbf{(X-\mu)'\Sigma^{-1}(X-\mu)}<2 \} = P\{\chi_5^2<2 \}
\end{align*}
Usando R:
\begin{verbatim}
pchisq(2, 5)
\end{verbatim}
Nos queda:
\begin{align*}
P\{\mathbf{(X-\mu)'\Sigma^{-1}(X-\mu)}<2 \} = P\{\chi_5^2<2 \} = 0.1508
\end{align*}
\item D)
Para este inciso usaremos una función en R la cual calcula el cuantil de la distribución chi cuadrada para cierta probabilidad. Esto es:
\begin{verbatim}
qchisq(0.85, 5)
\end{verbatim}
Por tanto, si queremos que se cumpla que P\{$\mathbf{(X-\mu)'\Sigma^{-1}(X-\mu)}<w$ \} $=$ P\{$\chi_5^2<w$\}$=0.85$ entonces $w=8.1151$
\end{itemize}
\end{sol}