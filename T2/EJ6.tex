\begin{problem}{6}
Considera las matrices de covarianzas del problema 1. Para cada una obtener su varianza generalizada.
\end{problem}
\begin{sol}
Sabemos que la varianza generalizada es su determinante, entonces, para A) del ejercicio 1 donde $\mathbf{\Sigma} =
\begin{bmatrix}
4&3&-2\\
3&6&2\\
-2&2&5
\end{bmatrix}$ calcularemos con R el determinante:
\begin{verbatim}
Sigma <- matrix(c(4, 3, -2,
                  3, 6,  2,
                  -2, 2,  5), 
                nrow = 3, byrow = TRUE)
det_Sigma <- det(Sigma)
\end{verbatim}
Donde nos resulta:
\[
\det(\mathbf{\Sigma}) = 
\begin{vmatrix}
4 & 3 & -2 \\
3 & 6 & 2 \\
-2 & 2 & 5
\end{vmatrix}=11
\]
Ahora para B) usaremos el siguiente código:
\begin{verbatim}
Sigma <- matrix(c(10, -3, -1, 5,
                  -3,  8,  3, 0,
                  -1,  3, 15, 1,
                  5,  0,  1, 4), 
                nrow = 4, byrow = TRUE)
det_Sigma <- det(Sigma)
\end{verbatim}

\[
\det(\mathbf{\Sigma}) = 
\begin{vmatrix}
10 & -3 & -1 & 5 \\
-3 & 8 & 3 & 0 \\
-1 & 3 & 15 & 1 \\
5 & 0 & 1 & 4
\end{vmatrix}=1104
\]
\end{sol}