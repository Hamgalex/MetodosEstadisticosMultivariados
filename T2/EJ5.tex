\begin{problem}{5}
Considera la matriz $\Sigma$ del problema 4. Obtener la distancia estadística entre los dos puntos A,B dados. 
\begin{itemize}
\item A) A$(3,-4,5)$ y B$(1,3,-2)$
\item B) A$(4,-8,9)$ y B$(6,9,10)$
\item C) A$(10,4,15)$ y el origen
\end{itemize}
\end{problem}
\begin{sol}
Usaremos la fórmula que nos dice:
\begin{align*}
d^2(PQ) = (\mathbf{x}-\mathbf{y})' \Sigma^{-1} (\mathbf{x}-\mathbf{y})
\end{align*}
Para esto hallaremos la inversa de $\mathbf{\Sigma}$ mediante un código en RStudio:
\begin{verbatim}
SIGMA <- matrix(c(12, 5, 12, 5, 16, 15, 12, 15, 20), nrow = 3, byrow = TRUE)
SIGMAINVERSA <- solve(SIGMA)
\end{verbatim}
\begin{align*}
\mathbf{\Sigma^{-1}} =
\begin{pmatrix}
0.6985 &  0.5882 & -0.8602 \\
0.5882 & 0.7058 & -0.8823  \\
-0.8602 & -0.8823 & 1.2279
\end{pmatrix}
\end{align*}
Ahora solo tendremos que hacer las operaciones pertinentes, para A):
\begin{align*}
d^2(AB) &= (\mathbf{x}-\mathbf{y})' \Sigma^{-1} (\mathbf{x}-\mathbf{y})\\
&= 
\left( \begin{pmatrix}
3 \\ -4 \\ 5
\end{pmatrix}- 
\begin{pmatrix}
1\\3\\-2
\end{pmatrix} \right)'
\begin{pmatrix}
0.6985 &  0.5882 & -0.8602 \\
0.5882 & 0.7058 & -0.8823  \\
-0.8602 & -0.8823 & 1.2279
\end{pmatrix}
\left( \begin{pmatrix}
3 \\ -4 \\ 5
\end{pmatrix}- 
\begin{pmatrix}
1\\3\\-2
\end{pmatrix} \right)\\
&= 
\begin{pmatrix}
2&-7&7
\end{pmatrix}
\begin{pmatrix}
0.6985 &  0.5882 & -0.8602 \\
0.5882 & 0.7058 & -0.8823  \\
-0.8602 & -0.8823 & 1.2279
\end{pmatrix}
\begin{pmatrix}
2\\-7\\7
\end{pmatrix}\\
\end{align*}
Usando el código en R:
\begin{verbatim}
XMENOSYTRANSPUESTA <- matrix(c(2, -7, 7), nrow = 1)  
SIGMA <- matrix(c(12, 5, 12, 5, 16, 15, 12, 15, 20), nrow = 3, byrow = TRUE)
SIGMAINVERSA <- solve(SIGMA)
XMENOSY <- matrix(c(2, -7, 7), nrow = 3)

RESULTADO1 <- XMENOSYTRANSPUESTA %*% SIGMAINVERSA
RESULTADO <- RESULTADO1 %*% XMENOSY
\end{verbatim}
\begin{align*}d^2(AB) &=
\begin{pmatrix}
2&-7&7\end{pmatrix}
\begin{pmatrix}
0.6985 &  0.5882 & -0.8602 \\
0.5882 & 0.7058 & -0.8823  \\
-0.8602 & -0.8823 & 1.2279
\end{pmatrix}
\begin{pmatrix}
2\\-7\\7
\end{pmatrix}\\
&= \begin{pmatrix}
-8.7426 & -9.9411 & 13.0514
\end{pmatrix}\begin{pmatrix}
2\\-7\\7
\end{pmatrix}\\
&= (143.4632)\\
d(AB) &= \sqrt{143.4632}=11.9776
\end{align*}
Por tanto la distancia estadística entre A y B es de $7.6614$\\
Para B):
\begin{align*}
d^2(AB) &= (\mathbf{x}-\mathbf{y})' \Sigma^{-1} (\mathbf{x}-\mathbf{y})\\
&= 
\left( \begin{pmatrix}
4 \\ -8 \\ 9
\end{pmatrix}- 
\begin{pmatrix}
6\\9\\10
\end{pmatrix} \right)'
\begin{pmatrix}
0.6985 &  0.5882 & -0.8602 \\
0.5882 & 0.7058 & -0.8823  \\
-0.8602 & -0.8823 & 1.2279
\end{pmatrix}
\left( \begin{pmatrix}
4 \\ -8 \\ 9
\end{pmatrix}- 
\begin{pmatrix}
6\\9\\10
\end{pmatrix} \right)\\
&= 
\begin{pmatrix}
-2&-17&-1
\end{pmatrix}
\begin{pmatrix}
0.6985 &  0.5882 & -0.8602 \\
0.5882 & 0.7058 & -0.8823  \\
-0.8602 & -0.8823 & 1.2279
\end{pmatrix}
\begin{pmatrix}
-2\\-17\\-1
\end{pmatrix}\\
\end{align*}
Usando el código en R:
\begin{verbatim}
XMENOSYTRANSPUESTA <- matrix(c(-2, -17, -1), nrow = 1)  
SIGMA <- matrix(c(12, 5, 12, 5, 16, 15, 12, 15, 20), nrow = 3, byrow = TRUE)
SIGMAINVERSA <- solve(SIGMA)
XMENOSY <- matrix(c(-2, -17, -1), nrow = 3)

RESULTADO1 <- XMENOSYTRANSPUESTA %*% SIGMAINVERSA
RESULTADO <- RESULTADO1 %*% XMENOSY
\end{verbatim}
\begin{align*}d^2(AB) &=
\begin{pmatrix}
-2&-17&-1
\end{pmatrix}
\begin{pmatrix}
0.6985 &  0.5882 & -0.8602 \\
0.5882 & 0.7058 & -0.8823  \\
-0.8602 & -0.8823 & 1.2279
\end{pmatrix}
\begin{pmatrix}
-2\\-17\\-1
\end{pmatrix}\\
&= \begin{pmatrix}
-10.5367 & -12.2941 & 15.4926
\end{pmatrix}\begin{pmatrix}
2\\-7\\7
\end{pmatrix}\\
&= (214.5809)\\
d(AB) &= \sqrt{214.5809}=14.6485
\end{align*}
Por tanto la distancia estadística entre A y B es de $14.6485$\\
\pagebreak

Para C):
\begin{align*}
d^2(AO) &= \mathbf{x}' \Sigma^{-1} \mathbf{x})\\
&= 
\begin{pmatrix}
10 \\ 4 \\ 15
\end{pmatrix}'
\begin{pmatrix}
0.6985 &  0.5882 & -0.8602 \\
0.5882 & 0.7058 & -0.8823  \\
-0.8602 & -0.8823 & 1.2279
\end{pmatrix}
\begin{pmatrix}
10 \\ 4 \\ 15
\end{pmatrix}\\
&= 
\begin{pmatrix}
10&4&15
\end{pmatrix}
\begin{pmatrix}
0.6985 &  0.5882 & -0.8602 \\
0.5882 & 0.7058 & -0.8823  \\
-0.8602 & -0.8823 & 1.2279
\end{pmatrix}
\begin{pmatrix}
10\\4\\15
\end{pmatrix}\\
\end{align*}
Usando el código en R:
\begin{verbatim}
XMENOSYTRANSPUESTA <- matrix(c(10,4,15), nrow = 1)  
SIGMA <- matrix(c(12, 5, 12, 5, 16, 15, 12, 15, 20), nrow = 3, byrow = TRUE)
SIGMAINVERSA <- solve(SIGMA)
XMENOSY <- matrix(c(10,4,15), nrow = 3)

RESULTADO1 <- XMENOSYTRANSPUESTA %*% SIGMAINVERSA
RESULTADO <- RESULTADO1 %*% XMENOSY
\end{verbatim}
\begin{align*}d^2(AO) &=
\begin{pmatrix}
-2&-17&-1
\end{pmatrix}
\begin{pmatrix}
0.6985 &  0.5882 & -0.8602 \\
0.5882 & 0.7058 & -0.8823  \\
-0.8602 & -0.8823 & 1.2279
\end{pmatrix}
\begin{pmatrix}
-2\\-17\\-1
\end{pmatrix}\\
&= \begin{pmatrix}
-4.5661 & -4.5294 & 6.2867
\end{pmatrix}\begin{pmatrix}
2\\-7\\7
\end{pmatrix}\\
&= (40.5220)\\
d(AO) &= \sqrt{40.5220}=6.3656
\end{align*}
Por tanto la distancia estadística entre A y el origen es de $6.3656$\\
\end{sol}
