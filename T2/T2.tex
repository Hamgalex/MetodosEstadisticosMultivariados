%%%%%%%%%%%%%%%%%%%%%%%%%%%%%%%%%%%%%%%%%%%%%%%%%%%%%%%%%%%%%%%%%%%%%%%%%%%%%%%%%%%%
%Do not alter this block of commands.  If you're proficient at LaTeX, you may include additional packages, create macros, etc. immediately below this block of commands, but make sure to NOT alter the header, margin, and comment settings here. 
\documentclass[12pt]{article}
 \usepackage[margin=1in]{geometry} 
\usepackage{amsmath,amsthm,amssymb,amsfonts, enumitem, fancyhdr, color, comment, graphicx, environ}
\pagestyle{fancy}
\setlength{\headheight}{65pt}
\newenvironment{problem}[2][Problema]{\begin{trivlist}
\item[\hskip \labelsep {\bfseries #1}\hskip \labelsep {\bfseries #2.}]}{\end{trivlist}}
\newenvironment{sol}
    {\emph{Solución:}
    }
    {
    }
\specialcomment{com}{ \color{blue} \textbf{Comment:} }{\color{black}} %for instructor comments while grading
\NewEnviron{probscore}{\marginpar{ \color{blue} \tiny Problem Score: \BODY \color{black} }}
%%%%%%%%%%%%%%%%%%%%%%%%%%%%%%%%%%%%%%%%%%%%%%%%%%%%%%%%%%%%%%%%%%%%%%%%%%%%%%%%%





%%%%%%%%%%%%%%%%%%%%%%%%%%%%%%%%%%%%%%%%%%%%%
%Fill in the appropriate information below
\lhead{Héctor Alejandro Márquez González}  %replace with your name
\rhead{Métodos Estadísticos Multivariados \\ 1989936} %replace XYZ with the homework course number, semester (e.g. ``Spring 2019"), and assignment number.
%%%%%%%%%%%%%%%%%%%%%%%%%%%%%%%%%%%%%%%%%%%%%


%%%%%%%%%%%%%%%%%%%%%%%%%%%%%%%%%%%%%%
%Do not alter this block.
\begin{document}
%%%%%%%%%%%%%%%%%%%%%%%%%%%%%%%%%%%%%%

\begin{problem}{1}
Considera las siguientes matrices de covarianzas de un vector aleatorio X. Obtener la matriz de coeficiente de correlaciones para cada caso. Después, interpreta los valores obtenidos.
\begin{align*}
A) \quad \Sigma &=
\begin{bmatrix}
4&3&-2\\
3&6&2\\
-2&2&5
\end{bmatrix}\\
B) \quad \Sigma &=
\begin{bmatrix}
10&-3&-1&5\\
-3&8&3&0\\
-1&3&15&1\\
5&0&1&4
\end{bmatrix}
\end{align*}

\end{problem}
\begin{sol}
Sabemos que $\rho = (\mathbf{V}^{\frac{1}{2}})^{-1} \mathbf{\Sigma} (\mathbf{V}^{\frac{1}{2}})^{-1}$, donde:
\begin{align*}
\mathbf{V}^{\frac{1}{2}}=
\begin{pmatrix}
\sqrt{\sigma_{11}} & 0 & \cdots & 0 \\
0 & \sqrt{\sigma_{22}} & \cdots & 0 \\
\vdots & \vdots & \ddots & \vdots \\
0 & 0 & \cdots & \sqrt{\sigma_{pp}}
\end{pmatrix}
\end{align*}
Por tanto, para A):
\begin{align*}\mathbf{V}^{\frac{1}{2}} &=
\begin{pmatrix}
\sqrt{4} & 0 & 0 \\
0 & \sqrt{6} & 0 \\
0 & 0 & \sqrt{5}
\end{pmatrix}\\
\end{align*}
Usando Gauss-Jordan:
\begin{align*}
(\mathbf{V}^{\frac{1}{2}})^{-1} &:
\begin{pmatrix}
\sqrt{4} & 0 & 0 & \vert &1 & 0 & 0 \\
0 & \sqrt{6} & 0 & \vert &0 & 1 & 0 \\
0 & 0 & \sqrt{5} & \vert & 0 & 0 & 1 
\end{pmatrix}\\
(\mathbf{V}^{\frac{1}{2}})^{-1} &:
\begin{pmatrix}
1 & 0 & 0 & \vert &\frac{1}{\sqrt{4}} & 0 & 0 \\
0 & 1 & 0 & \vert &0 &\frac{1}{\sqrt{6}} & 0 \\
0 & 0 & 1 & \vert & 0 & 0 & \frac{1}{\sqrt{5}}
\end{pmatrix}\\
(\mathbf{V}^{\frac{1}{2}})^{-1} &= 
\begin{pmatrix}
\frac{1}{\sqrt{4}} & 0 & 0 \\
0 &\frac{1}{\sqrt{6}} & 0 \\
 0 & 0 & \frac{1}{\sqrt{5}}
\end{pmatrix}\\
\end{align*}
Entonces:
\begin{align*}
\rho &= \begin{pmatrix}
\frac{1}{\sqrt{4}} & 0 & 0 \\
0 &\frac{1}{\sqrt{6}} & 0 \\
 0 & 0 & \frac{1}{\sqrt{5}}
\end{pmatrix}
\begin{pmatrix}
4&3&-2\\
3&6&2\\
-2&2&5
\end{pmatrix}
\begin{pmatrix}
\frac{1}{\sqrt{4}} & 0 & 0 \\
0 &\frac{1}{\sqrt{6}} & 0 \\
 0 & 0 & \frac{1}{\sqrt{5}}
\end{pmatrix}\\
&= \begin{pmatrix}
\frac{4}{\sqrt{4}} & \frac{3}{\sqrt{4}} & \frac{-2}{\sqrt{4}} \\
\frac{3}{\sqrt{6}} & \frac{6}{\sqrt{6}} & \frac{2}{\sqrt{6}} \\
\frac{-2}{\sqrt{5}} & \frac{2}{\sqrt{5}} & \frac{5}{\sqrt{5}}
\end{pmatrix}
\begin{pmatrix}
\frac{1}{\sqrt{4}} & 0 & 0 \\
0 &\frac{1}{\sqrt{6}} & 0 \\
 0 & 0 & \frac{1}{\sqrt{5}}
\end{pmatrix}\\
&= \begin{pmatrix}
\frac{4}{4} & \frac{3}{\sqrt{24}} & -\frac{2}{\sqrt{20}} \\
\frac{3}{\sqrt{24}} & \frac{6}{6} & \frac{2}{\sqrt{30}} \\
-\frac{2}{\sqrt{20}} & \frac{2}{\sqrt{30}} & \frac{5}{5}
\end{pmatrix}\\
\rho &= \begin{pmatrix}
1 & \frac{3}{2\sqrt{6}} & -\frac{1}{\sqrt{5}} \\
\frac{3}{2\sqrt{6}} & 1 & \frac{2}{\sqrt{30}} \\
-\frac{1}{\sqrt{5}} & \frac{2}{\sqrt{30}} & 1
\end{pmatrix}\\
\rho &= 
\begin{pmatrix}
1 & 0.6123 & -0.4472 \\
0.6123 & 1 & 0.3651 \\
-0.4472 &  0.3651 & 1
\end{pmatrix}
\end{align*}
Observamos que $\rho_{12} = 0.6123$ esto significa que hay correlación positiva moderada entre la primera y segunda entrada del vector aleatorio, para $\rho_{13} = -0.4472$ hay una correlación debil negativa entre la primera y tercera entrada del vector aleatorio, finalmente, para $\rho_{23} = 0.3651$ hay una correlación debil positiva entre la segunda y tercera entrada del vector aleatorio.\\\\
Para B):
\begin{align*}\mathbf{V}^{\frac{1}{2}} &=
\begin{pmatrix}
\sqrt{10} & 0 & 0 & 0\\
0 & \sqrt{8} & 0 & 0\\
0 & 0 & \sqrt{15} & 0\\
0 &0 &0 & \sqrt{4}
\end{pmatrix}\\ 
(\mathbf{V}^{\frac{1}{2}})^{-1} &= 
\begin{pmatrix}
\frac{1}{\sqrt{10}} & 0 & 0 & 0\\
0 &\frac{1}{\sqrt{8}}  & 0 & 0\\
0 & 0 & \frac{1}{\sqrt{15}}  & 0\\
0 &0 &0 & \frac{1}{\sqrt{4}} 
\end{pmatrix}\\ 
\end{align*}
Entonces:
\begin{align*}
\rho &= \begin{pmatrix}
\frac{1}{\sqrt{10}} & 0 & 0 & 0\\
0 &\frac{1}{\sqrt{8}}  & 0 & 0\\
0 & 0 & \frac{1}{\sqrt{15}}  & 0\\
0 &0 &0 & \frac{1}{\sqrt{4}} 
\end{pmatrix}
\begin{pmatrix}
10&-3&-1&5\\
-3&8&3&0\\
-1&3&15&1\\
5&0&1&4
\end{pmatrix}
\begin{pmatrix}
\frac{1}{\sqrt{10}} & 0 & 0 & 0\\
0 &\frac{1}{\sqrt{8}}  & 0 & 0\\
0 & 0 & \frac{1}{\sqrt{15}}  & 0\\
0 &0 &0 & \frac{1}{\sqrt{4}} 
\end{pmatrix} \\
&= 
\begin{pmatrix}
\frac{10}{\sqrt{10}} & -\frac{3}{\sqrt{10}} & -\frac{1}{\sqrt{10}} & \frac{5}{\sqrt{10}} \\
-\frac{3}{\sqrt{8}} & \frac{8}{\sqrt{8}} & \frac{3}{\sqrt{8}} & 0 \\
-\frac{1}{\sqrt{15}} & \frac{3}{\sqrt{15}} & \frac{15}{\sqrt{15}} & \frac{1}{\sqrt{15}} \\
\frac{5}{\sqrt{4}} & 0 & \frac{1}{\sqrt{4}} & \frac{4}{\sqrt{4}} \\
\end{pmatrix}
\begin{pmatrix}
\frac{1}{\sqrt{10}} & 0 & 0 & 0\\
0 &\frac{1}{\sqrt{8}}  & 0 & 0\\
0 & 0 & \frac{1}{\sqrt{15}}  & 0\\
0 &0 &0 & \frac{1}{\sqrt{4}} 
\end{pmatrix} \\
\rho &= 
\begin{pmatrix}
\frac{10}{10} & -\frac{3}{\sqrt{80}}& -\frac{1}{\sqrt{150}}& \frac{5}{\sqrt{40}} \\
-\frac{3}{\sqrt{80}} & \frac{8}{8}& \frac{3}{\sqrt{120}}& 0\\
-\frac{1}{\sqrt{150}} & \frac{3}{\sqrt{120}}& \frac{15}{15}& \frac{1}{\sqrt{60}}\\
\frac{5}{\sqrt{40}} & 0& \frac{1}{\sqrt{60}}& \frac{4}{4}\\
\end{pmatrix} \\
\rho &=
\begin{pmatrix}
1 & -0.3354 & -0.0816 & 0.7905 \\ 
-0.3354 & 1 & 0.2738 & 0 \\ 
-0.0816 & 0.2738 & 1 & 0.1290 \\
0.7905 & 0 & 0.1290 & 1
\end{pmatrix} 
\end{align*}
Observamos que $\rho_{12} = -0.3354$ tiene una correlación negativa débil entre la primer y segunda entrada del vector aleatorio, $\rho_{13}=-0.0816$ hay correlación negativa extremadamente débil entre la primera y tercera entrada del vector aleatorio, $\rho_{14}=0.7905$ hay correlación positiva fuerte entre la primera y cuarta entrada del vector aleatorio, $\rho_{23}=0.2738$ hay correlación débil positiva entre la segunda y tercera entrada del vector aleatorio, $\rho_{24}=0$ no hay correlación entre la segunda y cuarta entrada del vector aleatorio, $\rho_{34}=0.1290$ hay correlación positiva débil entre la tercera y cuarta entrada del vector aleatorio.
\end{sol}
%%%%%%%%%%%%%%%%%%%%%%%%%%%%%%%%%%%%%
%Do not alter anything below this line.
\end{document}