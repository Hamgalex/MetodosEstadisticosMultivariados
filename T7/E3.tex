\begin{problem}{3}
Se colectaron datos sobre la contaminación del aire en cierta ciudad. Tratar de identificar factores no observables que estén relacionados con estas variables.
\end{problem}

\begin{sol}
Primero leeremos los datos y encontraremos la matriz de correlación:
    \begin{verbatim}
    file_path <- "C:/Users/hamga/Downloads/datos_tarea 7.xlsx"
    
    datos <- read_excel(file_path, sheet = 2)
    datos <- datos[,-1]
    head(datos)
    
    cor_matrix <- cor(datos)
    n_obs <- nrow(datos)
    
    corrplot(cor_matrix, method = "circle", addCoef.col = "black",
             order = "original", type = "upper")
    \end{verbatim}
    \includegraphics[width=1\textwidth]{img/17.png}\\
       Para poder identificar los factores primero tenemos que hacer la prueba de esfericidad de Bartlett para evaluar si la matriz de correlaciones es significativamente diferente de $\sigma^2 \bm{I}$, lo que indicaría que hay correlaciones significativas entre las variables y que el análisis factorial es apropiado:
    \begin{verbatim}
    > cortest.bartlett(cor_matrix, n=n_obs)
    \end{verbatim}
    \includegraphics[width=1\textwidth]{img/18.png}\\
  Observamos que el p value es muy pequeño, por lo que rechazamos nuestra hipótesis nula, por tanto, la matriz de correlaciones es significativamente diferente de $\sigma^2 \bm{I}$. Se infiere que si hay correlacion entre las variables, ahora determinaremos el número de factores:
    \begin{verbatim}
    fa.parallel(cor_matrix, fm = "pa", n.obs = n_obs, ylabel = "Eigenvalues")
    \end{verbatim}
    \includegraphics[width=1\textwidth]{img/19.png}\\
    Observamos que es recomendable usar 2 factores debido al codo de la gráfica. Ahora haremos el análisis de factores utilizando componentes principales:
    \begin{verbatim}
    acp <- principal(cor_matrix, nfactors = 2, rotate = "none")
    print(acp)
    \end{verbatim}
    \includegraphics[width=1\textwidth]{img/20.png}\\
     Podemos observar las cargas de los dos componentes PC1 y PC2. Ahora lo haremos de nuevo pero con una rotación varimax para obtener resultados mas precisos:
    \begin{verbatim}
    acp <- principal(cor_matrix, nfactors = 2, rotate = "varimax", n.obs = n_obs)
    
    (acp$r.scores)
    
    print(acp)
    \end{verbatim}
    \includegraphics[width=1\textwidth]{img/21.png}\\
	  Esto nos dice que el primer componente tendrá mayor carga para las variables CO, NO, NO2 Y HC mientras que el segundo tendra para la variable WIND, SOLAR RADIATION Y O3. Veremos la gráfica del análisis de componentes principales con rotación varimax:\\
    \includegraphics[width=1\textwidth]{img/22.png}\\    
    Observamos que las variables 2 y 6 pertenecen al segundo componente principal y las demás pertenecen al primero.
    Ahora haremos el análisis de factores:
\begin{verbatim}
    mlf <- fa(cor_matrix, nfactors = 2, fm="ml", rotate = "varimax", n.obs = n_obs,
              scores = "regression")
    print(mlf)
    \end{verbatim}
    \includegraphics[width=1\textwidth]{img/23.png}\\
        Los resultados nos dicen que el primer factor explica el 22\% de la varianza total, mientras que el segundo explica el 20\%, lo que hace que ambos factores expliquen el 42 \% de la varianza total.\\ En cuanto a los factores, el primero esta asociado fuertemente al CO, NO, NO2 y HC mientras que el segundoe está asociado a WIND, SOLAR RADIATION, y O3.
    Podemos ver en la siguiente gráfica cuando es de color morado tiene mayor carga para el factor dos y si es color blanco tiene mayor carga para el factor uno.\\
    \includegraphics[width=1\textwidth]{img/24.png}\\
Sin embargo, debido a la baja explicación de la varianza total y los errores, haremos otro análisis con cuatro factores, esto es:
\begin{verbatim}
    mlf <- fa(cor_matrix, nfactors = 4, fm="ml", rotate = "varimax", n.obs = n_obs,
              scores = "regression")
    print(mlf)
    \end{verbatim}
    \includegraphics[width=1\textwidth]{img/25.png}\\
Aqui podemos observar que entre los cuatro factores explican el 70\% de la varianza total. En total nuestros cuatro factores serán:
\begin{itemize}
\item factor 1: CO, NO2
\item factor 2: SOLAR RADIATION, O3
\item factor 3: HC
\item factor 4: WIND, NO
\end{itemize}
Los factores obtenidos pueden interpretarse de la siguiente manera:
\begin{itemize}
\item Emisiones directas de combustión (vehículos e industria). (CO, NO2): Representa contaminantes de combustión, asociados principalmente con emisiones de vehículos y procesos industriales.
\item Formación de ozono por procesos fotoquímicos  (Solar Radiation, O2): Relacionado con procesos fotoquímicos, destacando la formación de ozono en presencia de radiación solar.
\item Presencia de hidrocarburos en el ambiente. (HC): Captura la presencia de compuestos orgánicos volátiles, que pueden provenir de emisiones industriales y evaporación de combustibles.
\item Efectos meteorológicos en la dispersión de contaminantes. (Wind, NO): Sugiere la influencia de condiciones meteorológicas en la dispersión de contaminantes, en particular el impacto del viento sobre el dióxido de nitrógeno.

\end{itemize}
\end{sol}