\begin{problem}{2}
Se tiene la tasa de retorno semanal de 5 acciones bursátiles del NYSE. Tratar de identificar factores no observables que estén relacionados con estas variables.
\end{problem}
\begin{sol}
Primero leeremos los datos y encontraremos la matriz de correlación:
\begin{verbatim}
file_path <- "C:/Users/hamga/Downloads/datos_tarea 7.xlsx"

datos <- read_excel(file_path, sheet = 1)
datos <- datos[,-1]
head(datos)

cor_matrix <- cor(datos)
n_obs <- nrow(datos)

corrplot(cor_matrix, method = "circle", addCoef.col = "black",
         order = "original", type = "upper")
\end{verbatim}
\includegraphics[width=1\textwidth]{img/9.png}\\
Para poder identificar los factores primero tenemos que hacer la prueba de esfericidad de Bartlett para evaluar si la matriz de correlaciones es significativamente diferente de $\sigma^2 \bm{I}$, lo que indicaría que hay correlaciones significativas entre las variables y que el análisis factorial es apropiado:
\begin{verbatim}
> cortest.bartlett(cor_matrix, n=n_obs)
\end{verbatim}
\includegraphics[width=1\textwidth]{img/10.png}\\
Observamos que el p value es muy pequeño, por lo que rechazamos nuestra hipótesis nula, por tanto, la matriz de correlaciones es significativamente diferente de $\sigma^2 \bm{I}$. Se infiere que si hay correlacion entre las variables, ahora determinaremos el número de factores:
\begin{verbatim}
fa.parallel(cor_matrix, fm = "pa", n.obs = n_obs, ylabel = "Eigenvalues")
\end{verbatim}
\includegraphics[width=1\textwidth]{img/11.png}\\
Observamos que es recomendable usar 2 factores. Ahora haremos el análisis de factores utilizando componentes principales:
\begin{verbatim}
acp <- principal(cor_matrix, nfactors = 2, rotate = "none")
print(acp)
\end{verbatim}
\includegraphics[width=1\textwidth]{img/12.png}\\
Podemos observar las cargas de los dos componentes PC1 y PC2. Ahora lo haremos de nuevo pero con una rotación varimax para obtener resultados mas precisos:
\begin{verbatim}
acp <- principal(cor_matrix, nfactors = 2, rotate = "varimax", n.obs = n_obs)

(acp$r.scores)

print(acp)
\end{verbatim}
\includegraphics[width=1\textwidth]{img/13.png}\\
Esto nos dice que el primer componente tendrá mayor carga para las variables 1,2 y 3, mientras que el segundo tendra para la variable 4 y 5. Veremos la gráfica del análisis de componentes principales con rotación varimax:\\
\includegraphics[width=1\textwidth]{img/14.png}\\
Observamos que las variables 5 y 4 pertenecen al segundo componente principal y las 1,2 y 3 pertenecen al primero.
Ahora haremos el análisis de factores:
\begin{verbatim}
mlf <- fa(cor_matrix, nfactors = 2, fm="ml", rotate = "varimax", n.obs = n_obs,
          scores = "regression")
print(mlf)
\end{verbatim}
\includegraphics[width=1\textwidth]{img/15.png}\\
Los resultados nos dicen que el primer factor explica el 33\% de la varianza total, mientras que el segundo explica el 26\%, lo que hace que ambos factores expliquen el 60 \% de la varianza total.\\ En cuanto a los factores, el primero esta asociado fuertemente a la primera, segunda y tercer variable mientras que el segundo factor esta asociado fuertemente a la cuarta y quinta variable.\\ 
Podemos ver en la siguiente gráfica cuando es de color morado tiene mayor carga para el factor dos y si es color blanco tiene mayor carga para el factor uno.\\
\includegraphics[width=1\textwidth]{img/16.png}\\
La comunalidad (h2) nos dice que tanto se explica de la variabilidad de esa variable, observamos que para las variables Texaco y Du point se explica altamente la variabilidad mientras que para las otras no.\\
Al investigar a que se dedica cada empresa, tenemos que se dedican a la venta de:
\begin{itemize}
\item Allied Chemical: Productos químicos
\item Du pont: Ciencia, productos químicos y biotecnología
\item Union Carbide: Productos químicos
\item Exxon: Energía
\item Texaco: Energía
\end{itemize}
En conclusión, los resultados sugieren que las tasas de retorno de estas acciones están influenciadas por factores sectoriales, con un grupo relacionado con la industria química (factor 1) y otro con el sector energético (factor 2). Esta información puede ser útil para identificar patrones de comportamiento en el mercado y mejorar estrategias de diversificación.
\end{sol}