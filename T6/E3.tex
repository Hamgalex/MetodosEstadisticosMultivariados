\begin{problem}{3}
Se colectaron datos sobre un proceso de soldadura industrial.\\
A) Tratar de reducir la información a menos de 5 dimensiones usando componentes principales.\\
B) Obtener los valores de los componentes principales obtenidos en el a) para:
\begin{table}[h]
    \centering
    \begin{tabular}{lccccccccc}
        \toprule
        \textbf{x1} & \textbf{x2} & \textbf{x3} & \textbf{x4} & \textbf{x5} & \textbf{x6} & \textbf{x7} & \textbf{x8} & \textbf{x9} \\
        \midrule
        58 & 62 & 58 & 34 & 50 & 0.1 & 0.033 & 0.097 & 120 \\
        50 & 62 & 54 & 35 & 49 & 0.1 & 0.036 & 0.099 & 120 \\
        60 & 57 & 54 & 32 & 45 & 0.065 & 0.3 & 0.062 & 120 \\
        \bottomrule
    \end{tabular}
    \label{tab:datos}
\end{table}
\end{problem}


\begin{sol}A)
Para reducir la dimensionalidad, aplicamos el Análisis de Componentes Principales (PCA) con datos estandarizados, esto debido a que tenemos diferentes unidades de medida con escalas diferentes.
\begin{verbatim}
pca_matriz_correlacion <- function(datos){
  pca_corr <- prcomp(datos, center = TRUE, scale = TRUE)

  plot(pca_corr, type = "l", main = "Grafico de codo de x1, x2, ..., x6")

  print(pca_corr)

  print(summary(pca_corr))

  biplot(pca_corr, scale = 0)

  cat("eigenvalues: ", pca_corr$sdev^2 , "\n" )

  return(pca_corr)

}
\end{verbatim}
Nos resulta:\\\\
\includegraphics[width=1\textwidth]{img/5.png}\\
\includegraphics[width=1\textwidth]{img/6.png}\\
Esto nos quiere decir que tendremos:
\begin{align*}
\lambda_1 &= 2.777069, 
\lambda_2 &= 1.957986, 
\lambda_3 &= 1.183942, 
\lambda_4 &= 1.059898, 
\lambda_5 &= 0.8727796, 
\lambda_6 &= 0.6872157, \\
\lambda_7 &= 0.419001, 
\lambda_8 &= 0.03836691, 
\lambda_9 &= 0.003742198
\end{align*}
También de los datos podemos concluir que los primeros CUATRO componentes principales explican más del 77\% de la variabilidad de los datos, por tanto, reducimos las nueve variables a cuatro:
\begin{align*}
PC_1 &= -0.1345 \cdot x_1 + 0.0022 \cdot x_2 + 0.1476 \cdot x_3 + 0.1806 \cdot x_4 + 0.3641 \cdot x_5 + 0.5452 \cdot x_6 - 0.3997 \cdot x_7 \\ & + 0.5444 \cdot x_8 - 0.2036 \cdot x_9 \\
PC_2 &= -0.1900 \cdot x_1 + 0.2088 \cdot x_2 - 0.1939 \cdot x_3 - 0.5787 \cdot x_4 - 0.1335 \cdot x_5 + 0.1954 \cdot x_6 - 0.1876 \cdot x_7\\ & + 0.1947 \cdot x_8 + 0.6471 \cdot x_9 \\
PC_3 &= -0.4843 \cdot x_1 + 0.4436 \cdot x_2 - 0.4268 \cdot x_3 + 0.4077 \cdot x_4 - 0.3888 \cdot x_5 - 0.0415 \cdot x_6 - 0.1989 \cdot x_7 \\ &- 0.0453 \cdot x_8 - 0.1603 \cdot x_9 \\
PC_4 &= -0.4279 \cdot x_1 + 0.4121 \cdot x_2 + 0.7034 \cdot x_3 - 0.0186 \cdot x_4 - 0.0057 \cdot x_5 - 0.0022 \cdot x_6 + 0.3872 \cdot x_7 \\ &+ 0.0144 \cdot x_8 + 0.0429 \cdot x_9
\end{align*}

Nuestras nueve variables originales se combinaron linealmente para formar las cuatro primeras componentes principales PC1, PC2, PC3 y PC4.

\pagebreak

B) Usaremos el siguiente código: 
\begin{verbatim}
calcular_datos_dada_pca_corr <- function(pca_corr,datos_valores, medias, desvest){

  for (i in 1:nrow(datos_valores)) {

    # se tienen que escalar los valores debido a que para el pca
    # se uso la correlacion en vez de la covarianza.
    datos_valores_escalados <- scale(datos_valores[i,], center = medias, scale = desvest)

    cat("fila ",i,"\n")
    cat("y1=",sum(datos_valores_escalados*pca_corr$rotation[, "PC1"]),"\n")
    cat("y2=",sum(datos_valores_escalados*pca_corr$rotation[, "PC2"]),"\n")
    cat("y3=",sum(datos_valores_escalados*pca_corr$rotation[, "PC3"]),"\n")
    cat("y4=",sum(datos_valores_escalados*pca_corr$rotation[, "PC4"]),"\n\n")
  }
}
\end{verbatim}
Nos resulta:\\\\
\includegraphics[width=1\textwidth]{img/7.png}\\\\ \\

Por tanto:
\begin{table}[h]
    \centering
    \begin{tabular}{|c|c|c|c|c|}
        \hline
        \textbf{Fila} & \textbf{y1} & \textbf{y2} & \textbf{y3} & \textbf{y4} \\
        \hline
        1 & 0.777792 & 1.03079 & -0.03632513 & 0.8834595 \\
        2 & 0.8439702 & 2.26959 & 3.040871 & 0.3245175 \\
        3 & -3.392126 & -0.05219226 & -0.0785673 & -1.765081 \\
        \hline
    \end{tabular}
    \label{tab:valores_pca}
\end{table}
\end{sol}

