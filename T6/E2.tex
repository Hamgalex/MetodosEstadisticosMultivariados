\begin{problem}{2}
Se colectaron datos sobre la contaminación del aire en cierta ciudad.
A) Tratar de reducir la información a menos de 5 dimensiones usando componentes principales.
B) Obtener los valores de los componentes principales obtenidos en el a) para:
\begin{table}[h]
    \centering
    \begin{tabular}{|c|c|c|c|c|c|c|}
        \hline
        \textbf{X1} & \textbf{X2} & \textbf{X3} & \textbf{X4} & \textbf{X5} & \textbf{X6} & \textbf{X7} \\
        \hline
        5 & 86 & 7 & 2 & 13 & 18 & 2 \\
        7 & 79 & 7 & 4 & 9 & 25 & 3 \\
        7 & 79 & 5 & 2 & 8 & 6 & 2 \\
        \hline
    \end{tabular}
    \label{tab:datos}
\end{table}
\end{problem}

\begin{sol}
A) Para reducir la dimensionalidad, aplicamos el Análisis de Componentes Principales (PCA) con datos estandarizados:
\begin{verbatim}
pca_matriz_correlacion <- function(datos){
  pca_corr <- prcomp(datos, center = TRUE, scale = TRUE)

  plot(pca_corr, type = "l", main = "Grafico de codo de x1, x2, ..., x6")

  print(pca_corr)

  print(summary(pca_corr))

  biplot(pca_corr, scale = 0)

  cat("eigenvalues: ", pca_corr$sdev^2 , "\n" )

  return(pca_corr)

}
\end{verbatim}
Nos resulta:\\\\
\includegraphics[width=1\textwidth]{img/3.png}\\
Esto nos quiere decir que tendremos:
\begin{align*}
\lambda_1 &= 2.336783,
\lambda_2 &= 1.386001,
\lambda_3 &= 1.204066, 
\lambda_4 &= 0.7270865, 
\lambda_5 &= 0.6534765, 
\lambda_6 &= 0.5366888, \\
\lambda_7 &= 0.1558989
\end{align*}
También de los datos podemos concluir que los primeros cuatro componentes principales explican más del 80\% de la variabilidad de los datos, por tanto, reducimos las cinco variables a 4:
\[
PC_1 = -0.2368 \cdot X_1 + 0.2056 \cdot X_2 + 0.5511 \cdot X_3 + 0.3776 \cdot X_4 + 0.4980 \cdot X_5 + 0.3246 \cdot X_6 + 0.3194 \cdot X_7
\]

\[
PC_2 = 0.2784 \cdot X_1 - 0.5266 \cdot X_2 - 0.0068 \cdot X_3 + 0.4347 \cdot X_4 + 0.1998 \cdot X_5 - 0.5670 \cdot X_6 + 0.3079 \cdot X_7
\]

\[
PC_3 = 0.6435 \cdot X_1 + 0.2245 \cdot X_2 - 0.1136 \cdot X_3 - 0.4071 \cdot X_4 + 0.1966 \cdot X_5 + 0.1598 \cdot X_6 + 0.5410 \cdot X_7
\]

\[
PC_4 = 0.1727 \cdot X_1 + 0.7781 \cdot X_2 + 0.0053 \cdot X_3 + 0.2905 \cdot X_4 - 0.0424 \cdot X_5 - 0.5079 \cdot X_6 - 0.1431 \cdot X_7
\]

Nuestras siete variables originales se combinaron linealmente para formar las cuatro primeras componentes principales PC1, PC2, PC3 y PC4.

\pagebreak

B) Usaremos el siguiente código: 
\begin{verbatim}
calcular_datos_dada_pca_corr <- function(pca_corr,datos_valores, medias, desvest){

  for (i in 1:nrow(datos_valores)) {

    # se tienen que escalar los valores debido a que para el pca
    # se uso la correlacion en vez de la covarianza.
    datos_valores_escalados <- scale(datos_valores[i,], center = medias, scale = desvest)

    cat("fila ",i,"\n")
    cat("y1=",sum(datos_valores_escalados*pca_corr$rotation[, "PC1"]),"\n")
    cat("y2=",sum(datos_valores_escalados*pca_corr$rotation[, "PC2"]),"\n")
    cat("y3=",sum(datos_valores_escalados*pca_corr$rotation[, "PC3"]),"\n")
    cat("y4=",sum(datos_valores_escalados*pca_corr$rotation[, "PC4"]),"\n\n")
  }
}
\end{verbatim}
Nos resulta:\\\\
\includegraphics[width=1\textwidth]{img/4.png}\\\\

Por tanto: 
\begin{table}[ht]
\centering
\begin{tabular}{|c|c|c|c|c|}
\hline
\textbf{Fila} & $y_1$ & $y_2$ & $y_3$ & $y_4$ \\
\hline
1 & 1.979395 & -2.086912 & -1.452349 & -0.3633699 \\
2 & 2.570354 & -1.227583 & -0.7278856 & -0.7200659 \\
3 & -0.7349544 & -0.584906 & -1.181051 & 0.6903036 \\
\hline
\end{tabular}
\end{table}

\end{sol}
