%%%%%%%%%%%%%%%%%%%%%%%%%%%%%%%%%%%%%%%%%%%%%%%%%%%%%%%%%%%%%%%%%%%%%%%%%%%%%%%%%%%%
%Do not alter this block of commands.  If you're proficient at LaTeX, you may include additional packages, create macros, etc. immediately below this block of commands, but make sure to NOT alter the header, margin, and comment settings here. 
\documentclass[12pt]{article}
 \usepackage[margin=1in]{geometry} 
\usepackage{amsmath,amsthm,amssymb,amsfonts, enumitem, fancyhdr, color, comment, graphicx, environ,bm,booktabs}
\pagestyle{fancy}
\setlength{\headheight}{65pt}
\newenvironment{problem}[2][Problema]{\begin{trivlist}
\item[\hskip \labelsep {\bfseries #1}\hskip \labelsep {\bfseries #2.}]}{\end{trivlist}}
\newenvironment{sol}
    {\emph{Solución:}
    }
    {
    }
\specialcomment{com}{ \color{blue} \textbf{Comment:} }{\color{black}} %for instructor comments while grading
\NewEnviron{probscore}{\marginpar{ \color{blue} \tiny Problem Score: \BODY \color{black} }}
%%%%%%%%%%%%%%%%%%%%%%%%%%%%%%%%%%%%%%%%%%%%%%%%%%%%%%%%%%%%%%%%%%%%%%%%%%%%%%%%%





%%%%%%%%%%%%%%%%%%%%%%%%%%%%%%%%%%%%%%%%%%%%%
%Fill in the appropriate information below
\lhead{Héctor Alejandro Márquez González}  %replace with your name
\rhead{Métodos Estadísticos Multivariados \\ 1989936} %replace XYZ with the homework course number, semester (e.g. ``Spring 2019"), and assignment number.
%%%%%%%%%%%%%%%%%%%%%%%%%%%%%%%%%%%%%%%%%%%%%


%%%%%%%%%%%%%%%%%%%%%%%%%%%%%%%%%%%%%%
%Do not alter this block.
\begin{document}
%%%%%%%%%%%%%%%%%%%%%%%%%%%%%%%%%%%%%%

%\begin{problem}{1}
Se colectaron datos sobre la contaminacion del aire en cierta ciudad. Suponer normalidad
en los datos.\\
A) Determinar si el vector de media poblacional es $\mu_0=(8,74,5,2,10,9,3)$ con un nivel
de significancia del 5\% incluir el p-valor. \\  
B) Obtener IC simultaneos para las medias de cada componente, con un nivel de confianza  
global del 95\%. \\  
C) Obtener IC simultaneos para las diferencias de medias poblacionales (ignorando  
unidades) con un nivel de confianza global del 95\%. \\  
D) Obtener IC simultaneos para la media de cada componente, con un nivel de confianza  
global del 95\% usando el metodo de Bonferroni.  
\end{problem}
\begin{sol}
\begin{itemize}
\item A) Determinar si el vector de media poblacional es $\mu_0=(8,74,5,2,10,9,3)$ con un nivel
de significancia del 5\% incluir el p-valor.\\\\
Debido a que el vector de medias poblacionales y la matriz de covarianzas poblacionales son desconocidas, entonces usaremos la prueba de hipótesis para $\mu$ con parámetros desconocidos. Sea:
$H_0:\mu = \mu_0 $ v.s. $H_1:\mu \neq \mu_0$.
El estadístico de prueba es:
\begin{align*}
T^2=n(\bar{\mathbf{X}} - \mu_0)'\mathbf{S}^{-1}(\mathbf{\bar{X}}-\mu_0)
\end{align*}
Para esto, necesitamos calcular dos cosas, $\mathbf{\bar{X}}$, $\mathbf{S}$ y $\mathbf{S}^{-1}$, lo haremos con R:\\\\
\includegraphics[width=1\textwidth]{img/1.png}\\\\
Esto nos da:
\begin{align*}
\mathbf{\bar{X}} &= \begin{pmatrix}
  7.5000 \\
  73.8571 \\
  4.5476 \\
  2.1905 \\
  10.0476 \\
  9.4048 \\
  3.0952
\end{pmatrix}\\
\mathbf{S} &= \begin{pmatrix}
  2.5000 & -2.7805 & -0.3780 & -0.4634 & -0.5854 & -2.2317 &  0.1707 \\
 -2.7805 & 300.5157 &  3.9094 & -1.3868 &  6.7631 & 30.7909 &  0.6237 \\
 -0.3780 &   3.9094 &  1.5221 &  0.6736 &  2.3148 &  2.8217 &  0.1417 \\
 -0.4634 &  -1.3868 &  0.6736 &  1.1823 &  1.0883 & -0.8107 &  0.1765 \\
 -0.5854 &   6.7631 &  2.3148 &  1.0883 & 11.3635 &  3.1266 &  1.0441 \\
 -2.2317 &  30.7909 &  2.8217 & -0.8107 &  3.1266 & 30.9785 &  0.5947 \\
  0.1707 &   0.6237 &  0.1417 &  0.1765 &  1.0441 &  0.5947 &  0.4785
\end{pmatrix} \\
\mathbf{S^{-1}} &= \begin{pmatrix}
  0.5651 &  0.0023 & -0.2490 &  0.4382 &  0.0639 &  0.0762 & -0.5267 \\
  0.0023 &  0.0038 & -0.0067 &  0.0082 & -0.0006 & -0.0026 & -0.0021 \\
 -0.2490 & -0.0067 &  1.8812 & -1.1633 & -0.3075 & -0.1991 &  0.8880 \\
  0.4382 &  0.0082 & -1.1633 &  1.8575 &  0.1171 &  0.1852 & -0.9935 \\
  0.0639 & -0.0006 & -0.3075 &  0.1171 &  0.1707 &  0.0263 & -0.3793 \\
  0.0762 & -0.0026 & -0.1991 &  0.1852 &  0.0263 &  0.0640 & -0.1701 \\
 -0.5267 & -0.0021 &  0.8880 & -0.9935 & -0.3793 & -0.1701 &  3.4233
\end{pmatrix}
\end{align*}
Ahora calcularemos esta operación en R, $T^2=n(\bar{\mathbf{X}} - \mu_0)'\mathbf{S}^{-1}(\mathbf{\bar{X}}-\mu_0)$: \\\\
\includegraphics[width=1\textwidth]{img/2.png}\\\\
Por tanto $T^2 = 27.0684$. Nuestra regla de desición será que se rechaza $H_0$ con nivel de significancia $\alpha$ si:
\begin{align*}
T^2 > \frac{(n-1)p}{n-p}F_{\alpha,p,n-p}
\end{align*}
\includegraphics[width=1\textwidth]{img/3.png}\\\\
Debido a que $T^2>18.7389$ entonces rechazamos $H_0$ y aceptamos la hipótesis alternativa. Esto quiere decir que el vector de medias poblacionales no es igual a $\mu_0$\\
Para calcular el p-valor tenemos que:
\begin{align*}
P(F>\frac{(n-p)T^2}{(n-1)p})
\end{align*}
A continuación calcularemos esta probabilidad en R:\\\\
\includegraphics[width=1\textwidth]{img/4.png}\\\\
Por tanto p-valor$=0.0084$. \pagebreak
\item B) Obtener IC simultaneos para las medias de cada componente, con un nivel de confianza  
global del 95\%. \\  
Para esto usaremos la fórmula para los intervalos de confianza simultáneos, esto es:
\begin{align*}
\bar{x}_i-\sqrt{\frac{p(n-1)}{n(n-p)}F_{\alpha,p,n-p}s_{ii}}<\mu_i<\bar{x_i}+\sqrt{\frac{p(n-1)}{n(n-p)}F_{\alpha,p,n-p}s_{ii}}
\end{align*}
Calcularemos esto en R para todas las medias:\\\\
\includegraphics[width=1\textwidth]{img/5.png}\\\\
Esto es:
\begin{align*}
6.4439 < \mu_1 &< 8.5561 \\
62.2779 < \mu_2 &< 85.4364 \\
3.7235 < \mu_3 &< 5.3717 \\
1.4642 < \mu_4 &< 2.9168 \\
7.7959 < \mu_5 &< 12.2993 \\
5.6870 < \mu_6 &< 13.1225 \\
2.6332 < \mu_7 &< 3.5573 \\
\end{align*}
\pagebreak
\item C) Obtener IC simultaneos para las diferencias de medias poblacionales (ignorando  
unidades) con un nivel de confianza global del 95\%. \\
Para esto usaremos la fórmula de IC simultáneos para la diferencia de medias:
\begin{align*}
(\bar{x_i}-\bar{x_j}) \pm \sqrt{\frac{p(n-1)}{n(n-p)}F_{\alpha,p,n-p}(s_{ii}-2s_{ij}+s_{jj})}
\end{align*}
Observamos que al tener 7 variables entonces habrá $\binom{7}{2} = 21$ intervalos. Lo calcularemos en R:\\\\
\includegraphics[width=1\textwidth]{img/6.png}\\\\
Nos dan los siguientes resultados:
\begin{align*}
-78.0907 < \mu_{1} - \mu_{2} &< -54.6236 \\
1.4923 < \mu_{1} - \mu_{3} &< 4.4125 \\
3.8755 < \mu_{1} - \mu_{4} &< 6.7436 \\
-5.1376 < \mu_{1} - \mu_{5} &< 0.0423 \\
-6.0192 < \mu_{1} - \mu_{6} &< 2.2096 \\
3.3201 < \mu_{1} - \mu_{7} &< 5.4895 \\
57.8522 < \mu_{2} - \mu_{3} &< 80.7668 \\
60.0114 < \mu_{2} - \mu_{4} &< 83.3219 \\
52.2720 < \mu_{2} - \mu_{5} &< 75.3471 \\
53.4785 < \mu_{2} - \mu_{6} &< 75.4262 \\
59.1975 < \mu_{2} - \mu_{7} &< 82.3264 \\
1.5790 < \mu_{3} - \mu_{4} &< 3.1353 \\
-7.4193 < \mu_{3} - \mu_{5} &< -3.5807 \\
-8.3187 < \mu_{3} - \mu_{6} &< -1.3955 \\
0.5771 < \mu_{3} - \mu_{7} &< 2.3277 \\
-10.0081 < \mu_{4} - \mu_{5} &< -5.7062 \\
-11.0966 < \mu_{4} - \mu_{6} &< -3.3320 \\
-1.6686 < \mu_{4} - \mu_{7} &< -0.1409 \\
-3.3698 < \mu_{5} - \mu_{6} &< 4.6555 \\
4.8663 < \mu_{5} - \mu_{7} &< 9.0385 \\
2.6347 < \mu_{6} - \mu_{7} &< 9.9844 \\
\end{align*}
\pagebreak
\item D) Obtener IC simultaneos para la media de cada componente, con un nivel de confianza  
global del 95\% usando el metodo de Bonferroni. \\
El método de Bonferroni nos dice que ternemos el siguiente IC:
\begin{align*}
\bar{x_i}-t_{\alpha / 2p,n-1}\sqrt{\frac{s_{ii}}{n}}<\mu_i<\bar{x_i}+t_{\alpha / 2p,n-1}\sqrt{\frac{s_{ii}}{n}}
\end{align*}
\includegraphics[width=1\textwidth]{img/7.png}\\\\
Esto es:
\begin{align*}
6.8091 < \mu_1 &< 8.1909 \\
66.2823 < \mu_2 &< 81.4320 \\
4.0085 < \mu_3 &< 5.0867 \\
1.7153 < \mu_4 &< 2.6656 \\
8.5746 < \mu_5 &< 11.5206 \\
6.9727 < \mu_6 &< 11.8368 \\
2.7930 < \mu_7 &< 3.3975 \\
\end{align*}
\end{itemize}
\end{sol}
 \pagebreak

\begin{problem}{2}
Se colectaron datos sobre la contaminación del aire en cierta ciudad.
A) Tratar de reducir la información a menos de 5 dimensiones usando componentes principales.
B) Obtener los valores de los componentes principales obtenidos en el a) para:
\begin{table}[h]
    \centering
    \begin{tabular}{|c|c|c|c|c|c|c|}
        \hline
        \textbf{X1} & \textbf{X2} & \textbf{X3} & \textbf{X4} & \textbf{X5} & \textbf{X6} & \textbf{X7} \\
        \hline
        5 & 86 & 7 & 2 & 13 & 18 & 2 \\
        7 & 79 & 7 & 4 & 9 & 25 & 3 \\
        7 & 79 & 5 & 2 & 8 & 6 & 2 \\
        \hline
    \end{tabular}
    \label{tab:datos}
\end{table}
\end{problem}

\begin{sol}
A) Para reducir la dimensionalidad, aplicamos el Análisis de Componentes Principales (PCA) con datos estandarizados:
\begin{verbatim}
pca_matriz_correlacion <- function(datos){
  pca_corr <- prcomp(datos, center = TRUE, scale = TRUE)

  plot(pca_corr, type = "l", main = "Grafico de codo de x1, x2, ..., x6")

  print(pca_corr)

  print(summary(pca_corr))

  biplot(pca_corr, scale = 0)

  cat("eigenvalues: ", pca_corr$sdev^2 , "\n" )

  return(pca_corr)

}
\end{verbatim}
Nos resulta:\\\\
\includegraphics[width=1\textwidth]{img/3.png}\\
Esto nos quiere decir que tendremos:
\begin{align*}
\lambda_1 &= 2.336783,
\lambda_2 &= 1.386001,
\lambda_3 &= 1.204066, 
\lambda_4 &= 0.7270865, 
\lambda_5 &= 0.6534765, 
\lambda_6 &= 0.5366888, \\
\lambda_7 &= 0.1558989
\end{align*}
También de los datos podemos concluir que los primeros cuatro componentes principales explican más del 80\% de la variabilidad de los datos, por tanto, reducimos las cinco variables a 4:
\[
PC_1 = -0.2368 \cdot X_1 + 0.2056 \cdot X_2 + 0.5511 \cdot X_3 + 0.3776 \cdot X_4 + 0.4980 \cdot X_5 + 0.3246 \cdot X_6 + 0.3194 \cdot X_7
\]

\[
PC_2 = 0.2784 \cdot X_1 - 0.5266 \cdot X_2 - 0.0068 \cdot X_3 + 0.4347 \cdot X_4 + 0.1998 \cdot X_5 - 0.5670 \cdot X_6 + 0.3079 \cdot X_7
\]

\[
PC_3 = 0.6435 \cdot X_1 + 0.2245 \cdot X_2 - 0.1136 \cdot X_3 - 0.4071 \cdot X_4 + 0.1966 \cdot X_5 + 0.1598 \cdot X_6 + 0.5410 \cdot X_7
\]

\[
PC_4 = 0.1727 \cdot X_1 + 0.7781 \cdot X_2 + 0.0053 \cdot X_3 + 0.2905 \cdot X_4 - 0.0424 \cdot X_5 - 0.5079 \cdot X_6 - 0.1431 \cdot X_7
\]

Nuestras siete variables originales se combinaron linealmente para formar las cuatro primeras componentes principales PC1, PC2, PC3 y PC4.

\pagebreak

B) B) Usaremos el siguiente código: 
\begin{verbatim}
calcular_datos_dada_pca_corr <- function(pca_corr,datos_valores, medias, desvest){

  for (i in 1:nrow(datos_valores)) {

    # se tienen que escalar los valores debido a que para el pca
    # se uso la correlacion en vez de la covarianza.
    datos_valores_escalados <- scale(datos_valores[i,], center = medias, scale = desvest)

    cat("fila ",i,"\n")
    cat("y1=",sum(datos_valores_escalados*pca_corr$rotation[, "PC1"]),"\n")
    cat("y2=",sum(datos_valores_escalados*pca_corr$rotation[, "PC2"]),"\n")
    cat("y3=",sum(datos_valores_escalados*pca_corr$rotation[, "PC3"]),"\n")
    cat("y4=",sum(datos_valores_escalados*pca_corr$rotation[, "PC4"]),"\n\n")
  }
}
\end{verbatim}
Nos resulta:\\\\
\includegraphics[width=1\textwidth]{img/4.png}\\\\

Por tanto: 
\begin{table}[ht]
\centering
\begin{tabular}{|c|c|c|c|c|}
\hline
\textbf{Fila} & $y_1$ & $y_2$ & $y_3$ & $y_4$ \\
\hline
1 & 1.979395 & -2.086912 & -1.452349 & -0.3633699 \\
2 & 2.570354 & -1.227583 & -0.7278856 & -0.7200659 \\
3 & -0.7349544 & -0.584906 & -1.181051 & 0.6903036 \\
\hline
\end{tabular}
\end{table}

\end{sol}



%%%%%%%%%%%%%%%%%%%%%%%%%%%%%%%%%%%%%
%Do not alter anything below this line.
\end{document}