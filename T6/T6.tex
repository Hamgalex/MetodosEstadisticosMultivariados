%%%%%%%%%%%%%%%%%%%%%%%%%%%%%%%%%%%%%%%%%%%%%%%%%%%%%%%%%%%%%%%%%%%%%%%%%%%%%%%%%%%%
%Do not alter this block of commands.  If you're proficient at LaTeX, you may include additional packages, create macros, etc. immediately below this block of commands, but make sure to NOT alter the header, margin, and comment settings here. 
\documentclass[12pt]{article}
 \usepackage[margin=1in]{geometry} 
\usepackage{amsmath,amsthm,amssymb,amsfonts, enumitem, fancyhdr, color, comment, graphicx, environ,bm,booktabs}
\pagestyle{fancy}
\setlength{\headheight}{65pt}
\newenvironment{problem}[2][Problema]{\begin{trivlist}
\item[\hskip \labelsep {\bfseries #1}\hskip \labelsep {\bfseries #2.}]}{\end{trivlist}}
\newenvironment{sol}
    {\emph{Solución:}
    }
    {
    }
\specialcomment{com}{ \color{blue} \textbf{Comment:} }{\color{black}} %for instructor comments while grading
\NewEnviron{probscore}{\marginpar{ \color{blue} \tiny Problem Score: \BODY \color{black} }}
%%%%%%%%%%%%%%%%%%%%%%%%%%%%%%%%%%%%%%%%%%%%%%%%%%%%%%%%%%%%%%%%%%%%%%%%%%%%%%%%%





%%%%%%%%%%%%%%%%%%%%%%%%%%%%%%%%%%%%%%%%%%%%%
%Fill in the appropriate information below
\lhead{Héctor Alejandro Márquez González}  %replace with your name
\rhead{Métodos Estadísticos Multivariados \\ 1989936} %replace XYZ with the homework course number, semester (e.g. ``Spring 2019"), and assignment number.
%%%%%%%%%%%%%%%%%%%%%%%%%%%%%%%%%%%%%%%%%%%%%


%%%%%%%%%%%%%%%%%%%%%%%%%%%%%%%%%%%%%%
%Do not alter this block.
\begin{document}
%%%%%%%%%%%%%%%%%%%%%%%%%%%%%%%%%%%%%%

\begin{problem}{1}
Se tiene la tasa de retorno semanal de 5 acciones bursátiles del NYSE.\\
A) Tratar de reducir la información a menos de 5 dimensiones usando componenetes principales.\\
B) Obtener los valores de los componentes pricipales obtenidos en el a) para:
\begin{table}[h]
    \centering
    \begin{tabular}{lccccc}
        \toprule
         Allied Chemical & Du Pont & Union Carbide & Exxon & Texaco \\
        \midrule
        -0.030717 & 0.020202 & -0.04086 & -0.03905 & -0.05051 \\
        -0.003521 & 0.118812 & 0.089686 & 0.06007 & 0.021276 \\
        0.060071  & 0.079646 & 0.028807 & 0.036666 & 0.026041 \\
        \bottomrule
    \end{tabular}
    \label{tab:datos}
\end{table}
\end{problem}
\begin{sol}A)
Para reducir la dimensionalidad, aplicamos el Análisis de Componentes Principales (PCA) con datos estandarizados:
\begin{verbatim}
pca_matriz_correlacion <- function(datos){
  pca_corr <- prcomp(datos, center = TRUE, scale = TRUE)

  plot(pca_corr, type = "l", main = "Grafico de codo de x1, x2, ..., x6")

  print(pca_corr)

  print(summary(pca_corr))

  biplot(pca_corr, scale = 0)

  cat("eigenvalues: ", pca_corr$sdev^2 , "\n" )

  return(pca_corr)

}
\end{verbatim}
Nos resulta:\\\\
\includegraphics[width=1\textwidth]{img/1.png}\\
Esto nos quiere decir que tendremos:
\begin{align*}
\lambda_1 = 2.8564, \lambda_2 = 0.80911, \lambda_3 = 0.5400, \lambda_4 = 0.4513 , \lambda_5 = 0.343
\end{align*}
También de los datos podemos concluir que los primeros dos componentes principales explican más del 73\% de la variabilidad de los datos, por tanto, reducimos las cinco variables a 2:
\[
PC_1 = 0.4635 \cdot \text{Allied Chemical} + 0.4571 \cdot \text{Du Pont} + 0.4700 \cdot \text{Union Carbide} + 0.4217 \cdot \text{Exxon} + 0.4213 \cdot \text{Texaco}
\]
\[
PC_2 = 0.2408 \cdot \text{Allied Chemical} + 0.5091 \cdot \text{Du Pont} + 0.2606 \cdot \text{Union Carbide} - 0.5253 \cdot \text{Exxon} - 0.5822 \cdot \text{Texaco}
\]
Nuestras cinco variables originales se combinaron linealmente para formar las dos primeras componentes principales PC1 y PC2.

\pagebreak

B) Usaremos el siguiente código: 
\begin{verbatim}
calcular_datos_dada_pca_corr <- function(pca_corr,datos_valores){

  for (i in 1:nrow(datos_valores)) {
    cat("fila ",i,"\n")
    cat("y1=",sum(datos_valores[i,]*pca_corr$rotation[, "PC1"]),"\n")
    cat("y2=",sum(datos_valores[i,]*pca_corr$rotation[, "PC2"]),"\n\n")
  }
}
\end{verbatim}
Nos resulta:\\\\
\includegraphics[width=1\textwidth]{img/2.png}\\
\begin{table}[h]
    \centering
    \begin{tabular}{|c|c|c|}
        \hline
        \textbf{Fila} & \textbf{$y_1$ (PC1)} & \textbf{$y_2$ (PC2)} \\
        \hline
        1 & -0.06195594 & 0.04216006 \\
        2 & 0.129119 & 0.03906884 \\
        3 & 0.1042214 & 0.02810079 \\
        \hline
    \end{tabular}

    \label{tab:resultados_pca}
\end{table}
\end{sol}



%%%%%%%%%%%%%%%%%%%%%%%%%%%%%%%%%%%%%
%Do not alter anything below this line.
\end{document}