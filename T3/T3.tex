%%%%%%%%%%%%%%%%%%%%%%%%%%%%%%%%%%%%%%%%%%%%%%%%%%%%%%%%%%%%%%%%%%%%%%%%%%%%%%%%%%%%
%Do not alter this block of commands.  If you're proficient at LaTeX, you may include additional packages, create macros, etc. immediately below this block of commands, but make sure to NOT alter the header, margin, and comment settings here. 
\documentclass[12pt]{article}
 \usepackage[margin=1in]{geometry} 
\usepackage{amsmath,amsthm,amssymb,amsfonts, enumitem, fancyhdr, color, comment, graphicx, environ}
\pagestyle{fancy}
\setlength{\headheight}{65pt}
\newenvironment{problem}[2][Problema]{\begin{trivlist}
\item[\hskip \labelsep {\bfseries #1}\hskip \labelsep {\bfseries #2.}]}{\end{trivlist}}
\newenvironment{sol}
    {\emph{Solución:}
    }
    {
    }
\specialcomment{com}{ \color{blue} \textbf{Comment:} }{\color{black}} %for instructor comments while grading
\NewEnviron{probscore}{\marginpar{ \color{blue} \tiny Problem Score: \BODY \color{black} }}
%%%%%%%%%%%%%%%%%%%%%%%%%%%%%%%%%%%%%%%%%%%%%%%%%%%%%%%%%%%%%%%%%%%%%%%%%%%%%%%%%





%%%%%%%%%%%%%%%%%%%%%%%%%%%%%%%%%%%%%%%%%%%%%
%Fill in the appropriate information below
\lhead{Héctor Alejandro Márquez González}  %replace with your name
\rhead{Métodos Estadísticos Multivariados \\ 1989936} %replace XYZ with the homework course number, semester (e.g. ``Spring 2019"), and assignment number.
%%%%%%%%%%%%%%%%%%%%%%%%%%%%%%%%%%%%%%%%%%%%%


%%%%%%%%%%%%%%%%%%%%%%%%%%%%%%%%%%%%%%
%Do not alter this block.
\begin{document}
%%%%%%%%%%%%%%%%%%%%%%%%%%%%%%%%%%%%%%

\begin{problem}{1}
Considera un vector aleatorio $\mathbf{X}$ con distribución $N_5(\mathbf{\mu,\Sigma})$ donde:
\begin{align*}
\mathbf{\mu}' = (100,95,230,400,86), \quad \mathbf{\Sigma}=
\begin{bmatrix}
10 & -2 & 1 & 0 & 3 \\
-2 & 9 & -3 & 4 & 5 \\
1 & -3 & 15 & 7 & -2 \\
0 & 4 & 7 & 20 & 2 \\
3 & 5 & -2 & 2 & 5
\end{bmatrix}
\end{align*}
a) Obtener $P(90<X_2<100)$. b) Obtener $P(3X_1+4X_3-5X_5>800)$ \\
c) Sea el vector aleatorio:
\begin{align*}
\mathbf{Y}= \begin{bmatrix}
X_1+3X_2-4X_3+6X_4+X_5\\
2X_1+9X_2-10X_3+X_4-X_5\\
X_2+X_4-X_5
\end{bmatrix}
\end{align*}
Obtener la distribución de $\mathbf{Y}$\\
d) Obtener la distancia estadística de $\mathbf{X_1}'(110, 97, 230, 396, 85)$ a $\mathbf{X_2}'(96, 93, 237, 408, 90)$.\\
e) Indicar que componentes de $\mathbf{\bar{X}}$ son independientes.\\
e) Obtener la distribución de $\mathbf{W}= \begin{bmatrix}
\bar{X_1} + 2\bar{X_3}-\bar{X_4} \\
\bar{X_2}+\bar{X_5}
\end{bmatrix}$
\end{problem}
\begin{sol}a)Obtener $P(90<X_2<100)$.\\
Debido a que el vector aleatorio $\mathbf{X}$ sigue una distribución normal multivariada, entonces $X_i$ sigue una distribución normal, por tanto para calcular esta probabilidad observamos que: $\mu_2=95$ y $\sigma_{22}=9$. Usaremos código en R para calcular la probabilidad:\\ \\
\includegraphics[width=1\textwidth]{img/1.png}\\\\
Por tanto $P(90<X_2<100) = 0.9044$

\pagebreak

b)Obtener $P(3X_1+4X_3-5X_5>800)$.\\
Las combinaciones lineales de los componentes de $\mathbf{X}$ tienen distribución normal, esto es $N(\mathbf{a' \mu,a' \Sigma a})$
\begin{align*}
\begin{pmatrix}
3 & 0 & 4 & 0 & -5
\end{pmatrix} \mathbf{X} = 3X_1+4X_3-5X_5
\end{align*}
Calcularemos $\mathbf{a' \mu}$ y $\mathbf{a' \Sigma a}$ en R: \\\\
\includegraphics[width=1\textwidth]{img/2.png}\\\\
Ahora , para calcular la probabilidad sabemos que: $P(3X_1+4X_3-5X_5>800) = 1 - P(3X_1+4X_3-5X_5<800)$:\\\\
\includegraphics[width=1\textwidth]{img/3.png}\\\\
Por tanto, $P(3X_1+4X_3-5X_5>800)=0.3222$

\pagebreak

c) Obtener la distribución de $\mathbf{Y}$\\
Tenemos a $\mathbf{A}$:
\begin{align*} \mathbf{A} =
\begin{pmatrix}
1 & 3 & -4 & 6 & 1 \\
2 & 9 & -10 & 1 & -1\\
0 & 1 & 0 & 1 & -1 
\end{pmatrix}
\end{align*}
Sabemos que si $\mathbf{X}$ sigue una normal multivariada, entonces  $q$ combinaciones lineales sigue una distribución $N_q(\mathbf{A \mu , A \Sigma A'})$. Calcularemos estas operaciones en R:\\\\
\includegraphics[width=1\textwidth]{img/4.png}\\\\
Por tanto, la distribución de $\mathbf{Y}$ es:
\begin{align*}
\mathbf{Y}\sim N_3(\begin{pmatrix}
1951 \\ -931 \\ 409
\end{pmatrix}, 
\begin{pmatrix}
992 & 943 & 129 \\
943 & 2508 & 22 \\
129 & 22 & 28
\end{pmatrix})
\end{align*}

\pagebreak

c) Obtener la distancia estadística de $\mathbf{x_1}'(110, 97, 230, 396, 85)$ a $\mathbf{x_2}'(96, 93, 237, 408, 90)$. \\\\
Primero calcularemos  \(\mathbf{X_1} - \mathbf{X_2}\):
\begin{align*}
\mathbf{x_1} - \mathbf{x_2} &= (110 - 96, 97 - 93, 230 - 237, 396 - 408, 85 - 90) \\
\mathbf{x_1} - \mathbf{x_2}&= (14, 4, -7, -12, -5).
\end{align*}
Ahora calcularemos la inversa de $\mathbf{\Sigma}$ con R: \\\\
\includegraphics[width=1\textwidth]{img/5.png}\\\\
Por tanto:
\begin{align*}
\mathbf{\Sigma}^{-1} = \begin{bmatrix}
  0.5421 &  0.6879 & -0.0190 & -0.0300 & -1.0088 \\
  0.6879 &  1.1591 &  0.0229 & -0.0870 & -1.5279 \\
 -0.0190 &  0.0229 &  0.0990 & -0.0438 &  0.0457 \\
 -0.0300 & -0.0870 & -0.0438 &  0.0771 &  0.0567 \\
 -1.0088 & -1.5279 &  0.0457 &  0.0567 &  2.3288
\end{bmatrix}
\end{align*}
Ahora calcularemos la distancia con $d^2(\mathbf{x_1,x_2})=(\mathbf{x_1-x_2})\mathbf{\Sigma}^{-1}(\mathbf{x_1-x_2})'$ con R:\\\\
\includegraphics[width=1\textwidth]{img/6.png}\\\\
Por tanto la distancia estadística de $\mathbf{x_1}'(110, 97, 230, 396, 85)$ a $\mathbf{x_2}'(96, 93, 237, 408, 90)$ es de $22.4033$ 

\pagebreak

e) Indicar que componentes de $\mathbf{\bar{X}}$ son independientes.\\ 
En $\mathbf{\Sigma}$ tenemos que $\sigma_{14} = \sigma_{41} = 0$ y esta es la unica covarianza cero en la matriz de covarianzas poblacionales, esto sumado a que en una distribución normal multivariada la covarianza cero implica que los componentes correspondientes son independientes nos dice que los únicos componentes que son independientes son $\mathbf{X_1}$ y $\mathbf{X_4}$.
\end{sol}



%%%%%%%%%%%%%%%%%%%%%%%%%%%%%%%%%%%%%
%Do not alter anything below this line.
\end{document}