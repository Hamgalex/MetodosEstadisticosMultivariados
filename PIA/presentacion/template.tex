%%%%%%%%%%%%%%%%%%%%%%%%%%%%%%%%%%%%%%%%%
% Beamer Presentation
% LaTeX Template
% Version 2.0 (March 8, 2022)
%
% This template originates from:
% https://www.LaTeXTemplates.com
%
% Author:
% Vel (vel@latextemplates.com)
%
% License:
% CC BY-NC-SA 4.0 (https://creativecommons.org/licenses/by-nc-sa/4.0/)
%
%%%%%%%%%%%%%%%%%%%%%%%%%%%%%%%%%%%%%%%%%

%----------------------------------------------------------------------------------------
%	PACKAGES AND OTHER DOCUMENT CONFIGURATIONS
%----------------------------------------------------------------------------------------

\documentclass[
	11pt, % Set the default font size, options include: 8pt, 9pt, 10pt, 11pt, 12pt, 14pt, 17pt, 20pt
	%t, % Uncomment to vertically align all slide content to the top of the slide, rather than the default centered
	%aspectratio=169, % Uncomment to set the aspect ratio to a 16:9 ratio which matches the aspect ratio of 1080p and 4K screens and projectors
]{beamer}

\graphicspath{{Images/}{./}} % Specifies where to look for included images (trailing slash required)
\usepackage{colortbl}
\usepackage{bm}
\usepackage{booktabs} % Allows the use of \toprule, \midrule and \bottomrule for better rules in tables

%----------------------------------------------------------------------------------------
%	SELECT LAYOUT THEME
%----------------------------------------------------------------------------------------

\usetheme{Madrid}

%----------------------------------------------------------------------------------------
%	SELECT FONT THEME & FONTS
%----------------------------------------------------------------------------------------

\usefonttheme{default} % Typeset using the default sans serif font
\usepackage[default]{opensans} % Use the Open Sans font for sans serif text

%----------------------------------------------------------------------------------------
%	SELECT INNER THEME
%----------------------------------------------------------------------------------------

\useinnertheme{circles}

%----------------------------------------------------------------------------------------
%	PRESENTATION INFORMATION
%----------------------------------------------------------------------------------------

\title[Análisis de pacientes]{Análisis de pacientes menores a 15 años} % The short title in the optional parameter appears at the bottom of every slide, the full title in the main parameter is only on the title page

\subtitle{en el centro de salud en Puebla} % Presentation subtitle, remove this command if a subtitle isn't required

\author[Héctor Márquez]{Héctor Márquez} % Presenter name(s), the optional parameter can contain a shortened version to appear on the bottom of every slide, while the main parameter will appear on the title slide

\institute[UANL]{Universidad Autónoma de Nuevo León\\ \smallskip \textit{hector.marquez@uanl.edu.mx}} % Your institution, the optional parameter can be used for the institution shorthand and will appear on the bottom of every slide after author names, while the required parameter is used on the title slide and can include your email address or additional information on separate lines

\date[\today]{Métodos Estadísticos Multivariados \\ \today} % Presentation date or conference/meeting name, the optional parameter can contain a shortened version to appear on the bottom of every slide, while the required parameter value is output to the title slide

%----------------------------------------------------------------------------------------

\begin{document}

%----------------------------------------------------------------------------------------
%	TITLE SLIDE
%----------------------------------------------------------------------------------------

\begin{frame}
    \begin{columns}
        \column{0.2\textwidth}
            \includegraphics[width=0.9\linewidth]{logo_uanl.png}
        \column{0.6\textwidth}
            \centering
            {\Large \textbf{Análisis de pacientes menores a 15 años}}\\[0.3cm]
            {\small en el centro de salud Santa María Guadalupe Tecola, Puebla}\\[0.5cm]
            {\normalsize \textbf{Presentación de Proyecto de la Evaluación 2}}\\[0.3cm]
            \textbf{Posgrado:} Maestría en Ciencia de Datos\\
            \textbf{Unidad de aprendizaje:} Métodos Estadísticos Multivariados\\[0.3cm]
            \textbf{Profesor:} Rosa Isela Hernández Zamora\\
            \textbf{Alumno:} Héctor Alejandro Márquez González\\
            \textbf{Matrícula:} 1989936\\[0.3cm]
            \textbf{Fecha de presentación:} 25 de marzo del 2025
        \column{0.2\textwidth}
            \includegraphics[width=0.9\linewidth]{logo_fcfm.png}
    \end{columns}
\end{frame}


%----------------------------------------------------------------------------------------
%	TABLE OF CONTENTS SLIDE
%----------------------------------------------------------------------------------------

\begin{frame}
    \frametitle{Tabla de contenido} % Título de la diapositiva
    \begin{itemize}
        \item Objetivos
        \item Introducción
        \item Análisis descriptivo de los datos
        \item Análisis de las técnicas multivariadas
        \begin{itemize}
        \item Análisis Factorial
        \item Análisis de conglomerados 
        \end{itemize}
        \item Conclusión
    \end{itemize}
\end{frame}

%----------------------------------------------------------------------------------------
%	PRESENTATION BODY SLIDES
%----------------------------------------------------------------------------------------

\begin{frame}
    \frametitle{Objetivos} % Slide title
    \textbf{Objetivo del Proyecto:}
    \begin{itemize}
        \item Presentación de resultados donde se aplique una o varias técnicas multivariadas en el análisis de datos de varias variables aplicado a la solución de un problema real.
    \end{itemize}

    \vspace{0.5cm}

    \textbf{Objetivos Específicos:}
    \begin{itemize}
        \item Aplicar un análisis de factores para descubrir nuevas variables latentes y reducir la dimensionalidad del conjunto de datos sin perder información relevante.
        \item Implementar un análisis de conglomerados para agrupar a los pacientes en conjuntos según sus características clínicas y evaluar la relación entre estos grupos.
    \end{itemize}
\end{frame}

%----------------------------------------------------------------------------------------
%	INTRODUCTION SLIDE
%----------------------------------------------------------------------------------------

\begin{frame}
    \frametitle{Introducción} % Slide title
    \begin{itemize}
        \item El conjunto de datos contiene información clínica de 50 pacientes menores de 15 años atendidos en el centro de salud Santa María Guadalupe Tecola, Puebla.
        \item Se incluyen múltiples variables biomédicas que permiten evaluar la salud general y detectar posibles afecciones.
        \item Estas variables serán analizadas mediante técnicas multivariadas para identificar patrones y relaciones ocultas en los datos.
    \end{itemize}
\end{frame}

%----------------------------------------------------------------------------------------
%	VARIABLE DESCRIPTION SLIDE
%----------------------------------------------------------------------------------------

\begin{frame}
    \frametitle{Descripción de las Variables} % Slide title
    \scriptsize % Reduce el tamaño de la fuente
    \begin{table}[]
        \centering
        \begin{tabular}{ll}
            \hline
            \textbf{Variable} & \textbf{Descripción} \\
            \hline
            IMC & Índice de Masa Corporal (kg/m²) \\
            ALT & Alanina aminotransferasa, indica daño hepático (UI/L) \\
            LDL & Colesterol LDL (mg/dL), asociado a enfermedades cardiovasculares \\
            Gluc & Glucosa en sangre (mg/dL), mide el nivel de azúcar \\
            Creatinina & Indicador de función renal (mg/dL) \\
            TRI & Triglicéridos en sangre (mg/dL) \\
            Insulina & Hormona que regula la glucosa (µU/mL) \\
            DHL & Deshidrogenasa láctica, relacionada con daño celular (UI/L) \\
            \hline
        \end{tabular}
    \end{table}
\end{frame}

%----------------------------------------------------------------------------------------
%	DESCRIPTIVE ANALYSIS SLIDE
%----------------------------------------------------------------------------------------

\begin{frame}
    \frametitle{Análisis descriptivo del conjunto de datos} % Slide title
    \begin{itemize}
        \item La matriz de correlación muestra la relación entre las diferentes variables del estudio.
        \item Valores cercanos a 1 o -1 indican correlaciones fuertes, mientras que valores cercanos a 0 indican poca relación.
        \item Se utilizarán estas correlaciones para seleccionar variables clave en el análisis.
    \end{itemize}
\end{frame}

%----------------------------------------------------------------------------------------
%	CORRELATION PLOT SLIDE
%----------------------------------------------------------------------------------------

\begin{frame}
    \frametitle{Matriz de Correlación} % Slide title
    \begin{center}
        \includegraphics[width=0.8\linewidth]{correlation_plot.png} % Ajusta el tamaño según sea necesario
    \end{center}
\end{frame}

%----------------------------------------------------------------------------------------
%	HISTOGRAMS
%----------------------------------------------------------------------------------------

\begin{frame}
    \frametitle{Histogramas} % Slide title
    \begin{center}
        \includegraphics[width=0.8\linewidth]{histogramas_pacientes.png} % Ajusta el tamaño según sea necesario
    \end{center}
\end{frame}

%----------------------------------------------------------------------------------------
%	prueba normalidad
%----------------------------------------------------------------------------------------

\begin{frame}
    \frametitle{Prueba de Normalidad (Shapiro-Wilk)}

    \scriptsize
    \begin{table}[]
        \centering
        \rowcolors{2}{gray!25}{white}
        \begin{tabular}{l c}
            \hline
            \textbf{Variable} & \textbf{p-valor de Shapiro-Wilk} \\
            \hline
            IMC & \textcolor{green}{0.0814} \\
            ALT & \textcolor{red}{4.91e-08} \\
            LDL & \textcolor{red}{0.0108} \\
            Glucosa & \textcolor{green}{0.6602} \\
            Creatinina & \textcolor{red}{3.68e-05} \\
            Triglicéridos & \textcolor{green}{0.0987} \\
            Insulina & \textcolor{red}{2.87e-04} \\
            DHL & \textcolor{red}{2.38e-04} \\
            \hline
        \end{tabular}
    \end{table}

    \vspace{0.3cm}
    \textbf{Hipótesis de la prueba:}
    \begin{itemize}
        \item $H_0$: La variable sigue una distribución normal.
        \item $H_1$: La variable \textbf{no} sigue una distribución normal.
    \end{itemize}

    \vspace{0.2cm}
    \textbf{Conclusión:} IMC, Glucosa y Triglicéridos son compatibles con la normalidad.  
    ALT, LDL, Creatinina, Insulina y DHL \textbf{no} siguen una distribución normal (p-valor $<$ 0.05).
\end{frame}

%----------------------------------------------------------------------------------------
%	prueba normalidad multivariada
%----------------------------------------------------------------------------------------
\begin{frame}{Evaluación visual: Gráfico QQ con distancia de Mahalanobis}

\begin{itemize}
    \item Se graficaron los \textbf{cuantiles teóricos de una distribución chi-cuadrada} contra los \textbf{cuantiles muestrales de la distancia de Mahalanobis}.
    \item Si los datos siguen una \textbf{distribución normal multivariada}, los puntos deberían alinearse cerca de la diagonal roja.
    
\end{itemize}

\vspace{0.3cm}

\begin{center}
    \includegraphics[width=0.5\textwidth]{qqplot_mahalanobis.png}
\end{center}

\end{frame}

\begin{frame}{Prueba de Normalidad Multivariada}

\textbf{Hipótesis:}
\begin{itemize}
    \item $H_0$: Los datos siguen una distribución normal multivariada.
    \item $H_1$: Los datos \textbf{no} siguen una distribución normal multivariada.
\end{itemize}

\textbf{Método:} Prueba de Mardia, la cual evalúa:
\begin{itemize}
    \item Asimetría multivariada (skewness)
    \item Curtosis multivariada (kurtosis)
\end{itemize}

\vspace{0.2cm}
\textbf{Resultados en R:}
\begin{table}[]
    \centering
    \begin{tabular}{lcc}
        \toprule
        \textbf{Prueba} & \textbf{Estadístico} & \textbf{p-valor} \\
        \midrule
        Mardia Skewness & 219.77 & $< 0.001$ \\
        Mardia Kurtosis & 3.30 & $0.00095$ \\
        \bottomrule
    \end{tabular}
\end{table}

\vspace{0.3cm}
\textbf{Conclusión:} Como ambos p-valores son menores a 0.05, \textbf{rechazamos $H_0$}. No hay evidencia suficiente para asumir normalidad multivariada en los datos.

\end{frame}

%----------------------------------------------------------------------------------------
%	vector promedios y desviacion muestral
%----------------------------------------------------------------------------------------


\begin{frame}
  \frametitle{Vector de Promedios y Desviación Muestral}

  \begin{table}[ht]
    \centering
    \begin{tabular}{|l|c|c|}
      \hline
      \textbf{Variable} & \textbf{Promedio} & \textbf{Desviación Muestral} \\
      \hline
      IMC & 20.3320 & 3.9617 \\
      ALT & 27.5880 & 19.1643 \\
      LDL & 74.6680 & 28.3022 \\
      Gluc & 89.0800 & 6.5085 \\
      CreatininaSerica & 0.5500 & 0.1111 \\
      TRI & 111.1000 & 39.4328 \\
      Insulina & 13.4648 & 6.8496 \\
      DHL & 354.2000 & 168.4154 \\
      \hline
    \end{tabular}
    \caption{Vector de Promedios y Desviación Muestral para cada variable}
  \end{table}

\end{frame}

%---------------------------------------------------
%----------------------------------------------------------------------------------------
%	ANALISIS DE FACTORES
%----------------------------------------------------------------------------------------
%------------------------------------------------------

\begin{frame}
  \frametitle{Análisis de factores}

    \begin{itemize}
    \item \textbf{Prueba de Esfericidad de Bartlett:} Determinar si las variables están correlacionadas.
    \item \textbf{Análisis de Componentes Principales (PCA):} Identificar los componentes principales y la varianza explicada.
    \item \textbf{Gráfico de Codo:} Identificar el número óptimo de factores a extraer.
    \item \textbf{Matriz de Factores:} Observar las cargas de las variables para interpretar los factores.
  \end{itemize}
 
\end{frame}

%----------------------------------------------------------------------------------------
%	BARTLETT
%----------------------------------------------------------------------------------------

\begin{frame}
\frametitle{Prueba de Esfericidad de Bartlett}

\textbf{Objetivo:} Evaluar si las variables están suficientemente correlacionadas como para aplicar análisis factorial.

\vspace{0.3cm}
\textbf{Hipótesis a contrastar:}
\begin{itemize}
    \item $H_0$: $\Sigma = \sigma^2 \mathbf{I}$ — Las variables no están correlacionadas. No se recomienda el uso de análisis factorial
    
    \item $H_1$: $\Sigma \neq \sigma^2 \mathbf{I}$ — Existen correlaciones significativas entre variables.Es válido aplicar análisis factorial
\end{itemize}

\vspace{0.3cm}
\textbf{Resultados:}
\begin{itemize}
    \item Estadístico chi-cuadrado: $72.13$
    \item Valor p: $9.33 \times 10^{-6}$
    \item Grados de libertad: $28$
\end{itemize}

\vspace{0.3cm}
\textbf{Conclusión:} Como el p-valor es menor a 0.05, \textbf{rechazamos $H_0$}. Existen correlaciones entre las variables, por lo tanto, \textbf{sí se puede aplicar análisis factorial}.
\end{frame}



% Diapositiva 1: Introducción al PCA
\begin{frame}
\frametitle{Análisis de Componentes Principales (PCA)}

\begin{itemize}
    \item Se realizó un Análisis de Componentes Principales (PCA) para determinar el número de factores extraídos.
    \item El PCA permite reducir la dimensionalidad de los datos y entender qué proporción de la variabilidad total explican los primeros componentes.
    \item A continuación, mostramos la variación explicada por cada componente.
\end{itemize}

\end{frame}

% Diapositiva 2: Tabla de Variación Total Analizada
\begin{frame}
\frametitle{Variación Explicada por los Componentes Principales}

\begin{table}[ht]
\centering
\scriptsize
\begin{tabular}{|c|c|c|}
\hline
\textbf{Componente} & \textbf{Varianza} & \textbf{Acumulada} \\ \hline
PC1 & 29.76\% & 29.76\% \\ \hline
PC2 & 18.27\% & 48.03\% \\ \hline
PC3 & 15.62\% & 63.65\% \\ \hline
PC4 & 11.04\% & 74.69\% \\ \hline
PC5 & 9.59\% & 84.28\% \\ \hline
PC6 & 6.61\% & 90.89\% \\ \hline
PC7 & 5.02\% & 95.90\% \\ \hline
PC8 & 4.10\% & 100.00\% \\ \hline
\end{tabular}
\end{table}

\begin{itemize}
    \item Los primeros tres componentes (PC1, PC2 y PC3) explican el 63.65\% de la varianza total.
    \item Se podría considerar que tres componentes son suficientes para representar los datos de manera efectiva.
\end{itemize}

\end{frame}


\begin{frame}{Gráfico de Codo - Selección del Número de Factores}

\begin{itemize}
    \item El gráfico muestra los eigenvalues (valores propios) reales y simulados para determinar cuántos factores mantener.
    \item Se comparan los valores obtenidos con los datos reales (azul) y los simulados aleatoriamente (rojo).
    \item \textbf{Criterio:} Si el eigenvalue real es mayor al simulado, el factor es significativo.
\end{itemize}

\vspace{0.3cm}

\begin{center}
    \includegraphics[width=0.5\textwidth]{fa_parallel_plot.png}
\end{center}

\vspace{0.2cm}

\textbf{Conclusión:} Los factores 1, 2 y 3 tienen eigenvalues reales mayores que los simulados.  
\textcolor{blue}{\textbf{Se concluye que 3 factores son adecuados}} para el análisis, ya que explican varianza significativa más allá del azar.

\end{frame}



\begin{frame}
\frametitle{Matriz de Factores con Rotación Varimax}

\textbf{Análisis Realizado:}

Se realizó un análisis de factores con \textbf{3 factores} seleccionados, utilizando el método Varimax para la rotación de los factores. La rotación Varimax es una técnica ortogonal que busca maximizar la varianza de las cargas al cuadrado de cada factor, lo que facilita la interpretación de los factores.

\vspace{0.4cm}

\textbf{Resultados:} La matriz de factores obtenida muestra las cargas de cada variable en los factores seleccionados, lo que permite identificar qué variables están relacionadas con cada factor.

\end{frame}

\begin{frame}
\frametitle{Matriz de Factores con Rotación Varimax}

\textbf{Matriz de Factores con Rotación Varimax:}

\begin{table}[ht]
\centering
\begin{tabular}{|l|c|c|c|}
\hline
\textbf{Variable} & \textbf{Factor 1} & \textbf{Factor 2} & \textbf{Factor 3} \\
\hline
\textbf{IMC} & 0.70 & 0.49 & 0.26 \\
\hline
\textbf{ALT} & 0.70 & -0.18 & 0.34 \\
\hline
\textbf{LDL} & 0.08 & 0.26 & -0.58 \\
\hline
\textbf{Gluc} & 0.05 & 0.24 & 0.80 \\
\hline
\textbf{Creatinina Serica} & 0.13 & 0.72 & 0.10 \\
\hline
\textbf{TRI} & 0.80 & -0.20 & -0.14 \\
\hline
\textbf{Insulina} & 0.78 & 0.20 & -0.21 \\
\hline
\textbf{DHL} & 0.26 & -0.69 & 0.25 \\
\hline
\end{tabular}
\end{table}

\end{frame}


\begin{frame}
\frametitle{Identificación de Variables Latentes}

Las variables con las mayores cargas indican una fuerte relación con el factor correspondiente.

\vspace{0.4cm}

\textbf{Variables Latentes:}

\begin{itemize}
    \item \textbf{Factor 1:} Las variables \textbf{IMC}, \textbf{TRI} e \textbf{Insulina} tienen las mayores cargas, lo que sugiere que este factor está relacionado con la obesidad y la resistencia a la insulina.
    \item \textbf{Factor 2:} Las variables \textbf{Creatinina Serica} y \textbf{DHL} tienen las mayores cargas, lo que indica que este factor está asociado con el funcionamiento renal y la inflamación.
    \item \textbf{Factor 3:} La variable \textbf{Gluc} tiene la mayor carga, lo que sugiere que este factor está relacionado con el metabolismo de la glucosa.
\end{itemize}

\end{frame}

%----------------------------------------------------------------------------------------
%	CLUSTERING
%----------------------------------------------------------------------------------------

\begin{frame}{Metodología general del análisis de conglomerados}
    \begin{itemize}
        \item Primero, se utilizó el \textbf{índice de Silhouette} con el método jerárquico para estimar el número óptimo de grupos. 
        
        \item Luego, se aplicaron cuatro algoritmos de clustering:
        \begin{enumerate}
            \item K-means
            \item Expectation Maximization (EM)
            \item Jerárquico
            \item DBSCAN (basado en densidad)
        \end{enumerate}
        
        \item Para cada uno, se calculó una medida de \textbf{validación interna} (\textbf{C-index}), ya que no se dispone de una clasificación real.
        
        \item Finalmente, se seleccionó el método con el menor C-index como la mejor solución de agrupamiento.
    \end{itemize}
\end{frame}


\begin{frame}{Selección del número de clusters con Silhouette}

\begin{itemize}
    \item Se aplicó clustering jerárquico y se evaluó el índice de Silhouette promedio para \(k = 2\) hasta \(k = 10\).
    \item Se graficó el promedio del índice para cada \(k\).
    \item La \textbf{altura del punto} en la gráfica representa qué tan buena es la separación entre grupos.
\end{itemize}

\vspace{0.3cm}

\vspace{0.3cm}
\textbf{Conclusión:} El valor máximo del índice se alcanza en \(\mathbf{k = 3}\), por lo que se concluye que \textbf{3 es el número óptimo de clusters} para este conjunto de datos.

\end{frame}

\begin{frame}{Gráfica de Silhouette para clustering jerárquico}
\begin{center}
    \includegraphics[width=0.8\textwidth]{silhouette_jerarquico.png}
\end{center}

\end{frame}


\begin{frame}{¿Qué es el C-index?}
    \begin{itemize}
        \item El \textbf{C-index} es una medida de validación interna que evalúa la compacidad de los grupos formados por un algoritmo de clustering.
        \item Se define como:
        \[
            C = \frac{S - S_{min}}{S_{max} - S_{min}}
        \]
        donde:
        \begin{itemize}
            \item $S$ es la suma de las distancias entre todos los pares de puntos que pertenecen al mismo cluster
            \item $S_{min}$ es la suma mínima posible (pares más cercanos).
            \item $S_{max}$ es la suma máxima posible (pares más lejanos).
        \end{itemize}
        \item \textbf{Interpretación:} Cuanto más cercano a 0, mejor la agrupación.
    \end{itemize}
\end{frame}

\begin{frame}{Comparación de métodos por C-index}
\begin{itemize}
    \item Se construirán cuatro modelos de clustering:
    \begin{itemize}
        \item K-means
        \item Expectation Maximization (EM)
        \item Jerárquico
        \item DBSCAN
    \end{itemize}
    \item Cada modelo se evaluará con el \textbf{índice C-index}, que mide la compacidad de los grupos formados.
    \item Se seleccionará el método con el \textbf{menor C-index}, es decir, el que mejor agrupa las observaciones.
    \item Al método seleccionado se le aplicará su algoritmo \textbf{10 veces} para intentar minimizar aún más el C-index.
\end{itemize}
\end{frame}
\begin{frame}{Validación interna - K-means}
\begin{itemize}
    \item Se aplicó el algoritmo K-means con $k = 3$.
    \item Se obtuvo un \textbf{C-index de 0.1162249}.
    \item Este fue el valor más bajo entre los métodos evaluados.
    \item Se considera la mejor solución inicial.
\end{itemize}
\end{frame}

\begin{frame}{Visualización 3D - K-means}
\begin{center}
    \includegraphics[width=0.6\textwidth]{KMEANS.png}
\end{center}
\end{frame}

%--------------------------------------
\begin{frame}{Validación interna - Expectation Maximization (EM)}
\begin{itemize}
    \item Se aplicó el algoritmo de Expectation Maximization.
    \item El valor del C-index obtenido fue \textbf{0.2454119}.
    \item Agrupación menos compacta en comparación con K-means.
\end{itemize}
\end{frame}

\begin{frame}{Visualización 3D - EM}
\begin{center}
    \includegraphics[width=0.6\textwidth]{EM.png}
\end{center}
\end{frame}

%--------------------------------------
\begin{frame}{Validación interna - Jerárquico}
\begin{itemize}
    \item Se aplicó el clustering jerárquico con método completo.
    \item Se obtuvo un C-index de \textbf{0.1403033}.
    \item Mejor que EM, pero no tan bueno como K-means.
\end{itemize}
\end{frame}

\begin{frame}{Visualización 3D - Jerárquico}
\begin{center}
    \includegraphics[width=0.6\textwidth]{JERARQUICO.png}
\end{center}
\end{frame}

%--------------------------------------
\begin{frame}{Validación interna - DBSCAN}
\begin{itemize}
    \item Se aplicó el algoritmo DBSCAN con parámetros $\varepsilon = 1.5$, minPts = 2.
    \item El valor del C-index fue \textbf{0.2409209}.
    \item Resultado similar a EM, con menor cohesión de los grupos.
\end{itemize}
\end{frame}

\begin{frame}{Visualización 3D - DBSCAN}
\begin{center}
    \includegraphics[width=0.6\textwidth]{DBSCAN.png}
\end{center}
\end{frame}

\begin{frame}{Comparación visual de los métodos de clustering (3D)}

\begin{center}
\begin{tabular}{cc}
    \includegraphics[width=0.25\textwidth]{KMEANS.png} & 
    \includegraphics[width=0.25\textwidth]{EM.png} \\

    \includegraphics[width=0.25\textwidth]{JERARQUICO.png} & 
    \includegraphics[width=0.25\textwidth]{DBSCAN.png} \\

\end{tabular}
\end{center}

\end{frame}

\begin{frame}{Resumen de C-index por método}
\begin{center}
\begin{tabular}{ll}
\toprule
\textbf{Método} & \textbf{C-index} \\
\midrule
K-means & 0.1162249 \\
Expectation Maximization (EM) & 0.2454119 \\
Jerárquico & 0.1403033 \\
DBSCAN & 0.2409209 \\
\bottomrule
\end{tabular}
\end{center}
\vspace{0.5cm}
\begin{itemize}
    \item El método con el \textbf{menor C-index} fue \textbf{K-means}.
    \item Por lo tanto, se aplicará el algoritmo K-means \textbf{10 veces} para intentar obtener una solución aún más óptima.
\end{itemize}
\end{frame}

\begin{frame}{Repetición de K-means para optimización}
\begin{itemize}
    \item Se aplicó el algoritmo K-means 10 veces con diferentes semillas.
    \item Los valores de C-index obtenidos fueron:
    \begin{itemize}
        \item 0.1354802, 0.1162249, 0.1162249, 0.1354802, 0.1354802
        \item 0.1162249, 0.2881972, 0.1162249, 0.1162249, 0.1354802
    \end{itemize}
    \item El valor mínimo fue nuevamente \textbf{0.1162249}, obtenido en varias ejecuciones.
    \item Se confirma que la solución inicial ya era una de las más compactas posibles.
\end{itemize}
\end{frame}

\begin{frame}{Análisis de separación de grupos}
\begin{itemize}
    \item Una vez determinado el mejor modelo (K-means con $k=3$), se analizó si existen variables que ayudan a separar los grupos.
    \item Se calcularon las medias de los factores por grupo:
\end{itemize}
\begin{center}
\begin{tabular}{lccc}
\toprule
\textbf{Grupo} & \textbf{Factor 1} & \textbf{Factor 2} & \textbf{Factor 3} \\
\midrule
1 & 1.18 & -0.20 & -0.92 \\
2 & 0.10 &  1.60 &  0.63 \\
3 & -0.55 & -0.49 &  0.17 \\
\bottomrule
\end{tabular}
\end{center}
\vspace{0.3cm}
\begin{itemize}
    \item El \textbf{Factor 1} y el \textbf{Factor 2} muestran diferencias marcadas entre los grupos.
    \item Esto sugiere que son variables relevantes para diferenciar los conglomerados.
\end{itemize}
\end{frame}

\begin{frame}{Boxplots de Factores por Cluster}
\begin{center}
    \includegraphics[width=0.85\textwidth]{boxplots_colores_por_factor.png}
\end{center}

\end{frame}


%--------------------------------------
\begin{frame}{Prueba de Igualdad de Medias Multivariadas (MANOVA)}

\textbf{Objetivo:} Determinar si los vectores de medias de los $k = 3$ grupos (conglomerados) son iguales.

\vspace{0.3cm}
\textbf{Hipótesis a contrastar:}
\begin{itemize}
    \item $H_0: \bm{\mu}_1 = \bm{\mu}_2 = \bm{\mu}_3$ (los vectores de medias poblacionales son iguales)
    \item $H_1:$ Al menos existe un par $\bm{\mu}_p \neq \bm{\mu}_q$ (algún vector de medias es diferente)
\end{itemize}

\vspace{0.3cm}
\textbf{Estadístico de prueba:} Se usa el estadístico de Wilks Lambda $(\Lambda^*)$, y se transforma en una chi-cuadrada con la siguiente fórmula:

\[
- \left[N - 1 - \frac{p + k}{2} \right] \ln(\Lambda^*) > \chi^2_{\alpha, \; p(k - 1)}
\]

\end{frame}

\begin{frame}{Resultados y Conclusión - MANOVA}

\textbf{Resultados:}
\begin{itemize}
    \item Estadístico calculado (chi-cuadrada): $\chi^2 = 125.42$
    \item Valor crítico: $\chi^2_{0.05, 6} = 12.59$
\end{itemize}

\vspace{0.4cm}
\textbf{Decisión:}
\begin{itemize}
    \item Como $\chi^2_{\text{calc}} = 125.42 > \chi^2_{\text{crítico}} = 12.59$, se \textbf{rechaza la hipótesis nula}.
\end{itemize}

\vspace{0.4cm}
\textbf{Conclusión:}
\begin{itemize}
    \item Existen diferencias estadísticamente significativas entre los vectores de medias de los grupos.
    \item Por lo tanto, al menos una variable factorial distingue claramente entre los conglomerados formados.
\end{itemize}

\end{frame}
\begin{frame}{Conclusiones}

\begin{itemize}
    \item Se aplicaron pruebas de normalidad univariada y multivariada. Aunque algunas variables cumplen con la normalidad, los datos en conjunto no siguen una distribución normal multivariada.
    
    \item El análisis factorial identificó tres factores latentes relevantes:
    \begin{itemize}
        \item \textbf{Factor 1:} Obesidad y resistencia a la insulina
        \item \textbf{Factor 2:} Función renal e inflamación
        \item \textbf{Factor 3:} Metabolismo de la glucosa
    \end{itemize}

    \item El análisis de conglomerados indicó que \textbf{K-means con 3 clusters} fue el mejor método según el C-index.
\end{itemize}

\end{frame}


\begin{frame}{Conclusiones}

\begin{itemize}
    \item El \textbf{Factor 1} y el \textbf{Factor 2} mostraron ser útiles para separar claramente los grupos formados por K-means.

    \item La prueba MANOVA confirmó que existen diferencias estadísticamente significativas entre los grupos.

    \item Esta segmentación puede ser de utilidad para detectar patrones clínicos relevantes en pacientes menores de 15 años.

    \item El enfoque multivariado aplicado puede apoyar futuras decisiones médicas, diagnósticas o de seguimiento clínico.
\end{itemize}

\end{frame}



\begin{frame}{Referencias}

\textbf{Fuente de los datos:}
\begin{itemize}
    \item Información clínica proporcionada por un estudiante de Medicina, de apellido \textbf{Petriz Guzmán}, quien recopiló los datos como parte de una práctica profesional en el centro de salud Santa María Guadalupe Tecola, Puebla.
\end{itemize}

\vspace{0.5cm}

\textbf{Paquetes utilizados en R:}

\smallskip

\texttt{readxl}, \texttt{ggplot2}, \texttt{gridExtra}, \texttt{corrplot}, \texttt{psych}, \texttt{MVN}, \texttt{mclust}, \texttt{clusterCrit}, \texttt{dbscan}, \texttt{plotly}, \texttt{htmlwidgets}, \texttt{cluster}, \texttt{dendextend}, \texttt{ape}

\end{frame}



\end{document}
