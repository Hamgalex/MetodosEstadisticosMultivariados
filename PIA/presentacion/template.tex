%%%%%%%%%%%%%%%%%%%%%%%%%%%%%%%%%%%%%%%%%
% Beamer Presentation
% LaTeX Template
% Version 2.0 (March 8, 2022)
%
% This template originates from:
% https://www.LaTeXTemplates.com
%
% Author:
% Vel (vel@latextemplates.com)
%
% License:
% CC BY-NC-SA 4.0 (https://creativecommons.org/licenses/by-nc-sa/4.0/)
%
%%%%%%%%%%%%%%%%%%%%%%%%%%%%%%%%%%%%%%%%%

%----------------------------------------------------------------------------------------
%	PACKAGES AND OTHER DOCUMENT CONFIGURATIONS
%----------------------------------------------------------------------------------------

\documentclass[
	11pt, % Set the default font size, options include: 8pt, 9pt, 10pt, 11pt, 12pt, 14pt, 17pt, 20pt
	%t, % Uncomment to vertically align all slide content to the top of the slide, rather than the default centered
	%aspectratio=169, % Uncomment to set the aspect ratio to a 16:9 ratio which matches the aspect ratio of 1080p and 4K screens and projectors
]{beamer}

\graphicspath{{Images/}{./}} % Specifies where to look for included images (trailing slash required)
\usepackage{colortbl}
\usepackage{booktabs} % Allows the use of \toprule, \midrule and \bottomrule for better rules in tables

%----------------------------------------------------------------------------------------
%	SELECT LAYOUT THEME
%----------------------------------------------------------------------------------------

\usetheme{Madrid}

%----------------------------------------------------------------------------------------
%	SELECT FONT THEME & FONTS
%----------------------------------------------------------------------------------------

\usefonttheme{default} % Typeset using the default sans serif font
\usepackage[default]{opensans} % Use the Open Sans font for sans serif text

%----------------------------------------------------------------------------------------
%	SELECT INNER THEME
%----------------------------------------------------------------------------------------

\useinnertheme{circles}

%----------------------------------------------------------------------------------------
%	PRESENTATION INFORMATION
%----------------------------------------------------------------------------------------

\title[Análisis de pacientes]{Análisis de pacientes menores a 15 años} % The short title in the optional parameter appears at the bottom of every slide, the full title in the main parameter is only on the title page

\subtitle{en el centro de salud en Puebla} % Presentation subtitle, remove this command if a subtitle isn't required

\author[Héctor Márquez]{Héctor Márquez} % Presenter name(s), the optional parameter can contain a shortened version to appear on the bottom of every slide, while the main parameter will appear on the title slide

\institute[UANL]{Universidad Autónoma de Nuevo León\\ \smallskip \textit{hector.marquez@uanl.edu.mx}} % Your institution, the optional parameter can be used for the institution shorthand and will appear on the bottom of every slide after author names, while the required parameter is used on the title slide and can include your email address or additional information on separate lines

\date[\today]{Métodos Estadísticos Multivariados \\ \today} % Presentation date or conference/meeting name, the optional parameter can contain a shortened version to appear on the bottom of every slide, while the required parameter value is output to the title slide

%----------------------------------------------------------------------------------------

\begin{document}

%----------------------------------------------------------------------------------------
%	TITLE SLIDE
%----------------------------------------------------------------------------------------

\begin{frame}
	\titlepage % Output the title slide, automatically created using the text entered in the PRESENTATION INFORMATION block above
\end{frame}

%----------------------------------------------------------------------------------------
%	TABLE OF CONTENTS SLIDE
%----------------------------------------------------------------------------------------

\begin{frame}
    \frametitle{Tabla de contenido} % Título de la diapositiva
    \begin{itemize}
        \item Objetivos
        \item Introducción
        \item Análisis descriptivo de los datos
        \begin{itemize}
        	\item Matriz de correlación
        	\item Histogramas
        	\item Prueba de normalidad
        	\item Promedios
        \end{itemize}
        \item Análisis Factorial
        \item Análisis de conglomerados 
    \end{itemize}
\end{frame}

%----------------------------------------------------------------------------------------
%	PRESENTATION BODY SLIDES
%----------------------------------------------------------------------------------------

\begin{frame}
    \frametitle{Objetivos} % Slide title
    \textbf{Objetivo del Proyecto:}
    \begin{itemize}
        \item Presentación de resultados donde se aplique una o varias técnicas multivariadas en el análisis de datos de varias variables aplicado a la solución de un problema real.
    \end{itemize}

    \vspace{0.5cm}

    \textbf{Objetivos Específicos:}
    \begin{itemize}
        \item Aplicar un análisis de factores para descubrir nuevas variables latentes y reducir la dimensionalidad del conjunto de datos sin perder información relevante.
        \item Implementar un análisis de conglomerados para agrupar a los pacientes en conjuntos homogéneos según sus características clínicas y evaluar la relación entre estos grupos y los factores identificados previamente.
    \end{itemize}
\end{frame}

%----------------------------------------------------------------------------------------
%	INTRODUCTION SLIDE
%----------------------------------------------------------------------------------------

\begin{frame}
    \frametitle{Introducción} % Slide title
    \begin{itemize}
        \item El conjunto de datos contiene información clínica de 50 pacientes menores de 15 años atendidos en el centro de salud Santa María Guadalupe Tecola, Puebla.
        \item Se incluyen múltiples variables biomédicas que permiten evaluar la salud general y detectar posibles afecciones.
        \item Estas variables serán analizadas mediante técnicas multivariadas para identificar patrones y relaciones ocultas en los datos.
    \end{itemize}
\end{frame}

%----------------------------------------------------------------------------------------
%	VARIABLE DESCRIPTION SLIDE
%----------------------------------------------------------------------------------------

\begin{frame}
    \frametitle{Descripción de las Variables} % Slide title
    \scriptsize % Reduce el tamaño de la fuente
    \begin{table}[]
        \centering
        \begin{tabular}{ll}
            \hline
            \textbf{Variable} & \textbf{Descripción} \\
            \hline
            IMC & Índice de Masa Corporal (kg/m²) \\
            ALT & Alanina aminotransferasa, indica daño hepático (UI/L) \\
            LDL & Colesterol LDL (mg/dL), asociado a enfermedades cardiovasculares \\
            Gluc & Glucosa en sangre (mg/dL), mide el nivel de azúcar \\
            Creatinina & Indicador de función renal (mg/dL) \\
            TRI & Triglicéridos en sangre (mg/dL) \\
            Insulina & Hormona que regula la glucosa (µU/mL) \\
            DHL & Deshidrogenasa láctica, relacionada con daño celular (UI/L) \\
            \hline
        \end{tabular}
    \end{table}
\end{frame}

%----------------------------------------------------------------------------------------
%	DESCRIPTIVE ANALYSIS SLIDE
%----------------------------------------------------------------------------------------

\begin{frame}
    \frametitle{Análisis descriptivo del conjunto de datos} % Slide title
    \begin{itemize}
        \item La matriz de correlación muestra la relación entre las diferentes variables del estudio.
        \item Valores cercanos a 1 o -1 indican correlaciones fuertes, mientras que valores cercanos a 0 indican poca relación.
        \item Se utilizarán estas correlaciones para seleccionar variables clave en el análisis.
    \end{itemize}
\end{frame}

%----------------------------------------------------------------------------------------
%	CORRELATION PLOT SLIDE
%----------------------------------------------------------------------------------------

\begin{frame}
    \frametitle{Matriz de Correlación} % Slide title
    \begin{center}
        \includegraphics[width=0.8\linewidth]{correlation_plot.png} % Ajusta el tamaño según sea necesario
    \end{center}
\end{frame}

%----------------------------------------------------------------------------------------
%	HISTOGRAMS
%----------------------------------------------------------------------------------------

\begin{frame}
    \frametitle{Histogramas} % Slide title
    \begin{center}
        \includegraphics[width=0.8\linewidth]{histogramas_pacientes.png} % Ajusta el tamaño según sea necesario
    \end{center}
\end{frame}

%----------------------------------------------------------------------------------------
%	prueba normalidad
%----------------------------------------------------------------------------------------


\begin{frame}
    \frametitle{Prueba de Normalidad (Shapiro-Wilk)}
    
    \scriptsize
    \begin{table}[]
        \centering
        \rowcolors{2}{gray!25}{white} % Colores alternados en filas
        \begin{tabular}{l c}
            \hline
            \textbf{Variable} & \textbf{p-valor de Shapiro-Wilk} \\
            \hline
            IMC & \textcolor{green}{0.0814} \\
            ALT & \textcolor{red}{4.91e-08} \\
            LDL & \textcolor{red}{0.0108} \\
            Glucosa & \textcolor{green}{0.6602} \\
            Creatinina & \textcolor{red}{3.68e-05} \\
            Triglicéridos & \textcolor{green}{0.0987} \\
            Insulina & \textcolor{red}{2.87e-04} \\
            DHL & \textcolor{red}{2.38e-04} \\
            \hline
        \end{tabular}
    \end{table}
    
    \vspace{0.4cm}
    \textbf{Conclusión:} IMC, Glucosa y Triglicéridos son compatibles con la normalidad.  
    ALT, LDL, Creatinina, Insulina y DHL \textbf{no} siguen una distribución normal.
\end{frame}


%----------------------------------------------------------------------------------------
%	prueba normalidad multivariada
%----------------------------------------------------------------------------------------


\begin{frame}{Prueba de Normalidad Multivariada}

    \textbf{Método:} Prueba de Mardia  

    \textbf{Resultados:}  
    \begin{table}[]
        \centering
        \begin{tabular}{lcc}
            \toprule
            \textbf{Prueba} & \textbf{Estadístico} & \textbf{p-valor} \\
            \midrule
            Mardia Skewness & 219.77 & $< 0.001$ \\
            Mardia Kurtosis & 3.30 & $0.00095$ \\
            \midrule
            \textbf{Conclusión} & \multicolumn{2}{c}{\textcolor{red}{No sigue una distribución normal multivariada}} \\
            \bottomrule
        \end{tabular}
    \end{table}

    \bigskip
    \textbf{Conclusión:}  
    Como los valores p son menores a 0.05, \textbf{rechazamos la hipótesis de normalidad multivariada}.  
    
\end{frame}

%----------------------------------------------------------------------------------------
%	vector promedios y desviacion muestral
%----------------------------------------------------------------------------------------


\begin{frame}
  \frametitle{Vector de Promedios y Desviación Muestral}

  \begin{table}[ht]
    \centering
    \begin{tabular}{|l|c|c|}
      \hline
      \textbf{Variable} & \textbf{Promedio} & \textbf{Desviación Muestral} \\
      \hline
      IMC & 20.3320 & 3.9617 \\
      ALT & 27.5880 & 19.1643 \\
      LDL & 74.6680 & 28.3022 \\
      Gluc & 89.0800 & 6.5085 \\
      CreatininaSerica & 0.5500 & 0.1111 \\
      TRI & 111.1000 & 39.4328 \\
      Insulina & 13.4648 & 6.8496 \\
      DHL & 354.2000 & 168.4154 \\
      \hline
    \end{tabular}
    \caption{Vector de Promedios y Desviación Muestral para cada variable}
  \end{table}

\end{frame}

%---------------------------------------------------
%----------------------------------------------------------------------------------------
%	ANALISIS DE FACTORES
%----------------------------------------------------------------------------------------
%------------------------------------------------------

\begin{frame}
  \frametitle{Análisis de factores}

    \begin{itemize}
    \item \textbf{Prueba de Esfericidad de Bartlett:} Determinar si las variables están correlacionadas.
    \item \textbf{Análisis de Componentes Principales (PCA):} Identificar los componentes principales y la varianza explicada.
    \item \textbf{Gráfico de Codo:} Identificar el número óptimo de factores a extraer.
    \item \textbf{Matriz de Factores:} Observar las cargas de las variables para interpretar los factores.
  \end{itemize}
 
\end{frame}

%----------------------------------------------------------------------------------------
%	BARTLETT
%----------------------------------------------------------------------------------------

\begin{frame}
\frametitle{Prueba de Esfericidad de Bartlett}

\begin{itemize}
    \item La prueba de esfericidad de Bartlett se realiza para verificar si la matriz de correlaciones es adecuada para realizar el análisis de factores.
    \item Resultado de la prueba de Bartlett:
    \begin{itemize}
        \item Valor chi-cuadrado: 72.13
        \item Valor p: \( 9.33 \times 10^{-6} \)
        \item Grados de libertad: 28
    \end{itemize}
    \item Con un valor p mucho menor que 0.05, rechazamos la hipótesis nula de que la matriz de correlaciones es una matriz identidad, lo que sugiere que el análisis de factores es adecuado para estos datos.
\end{itemize}

\end{frame}



% Diapositiva 1: Introducción al PCA
\begin{frame}
\frametitle{Análisis de Componentes Principales (PCA)}

\begin{itemize}
    \item Se realizó un Análisis de Componentes Principales (PCA) para determinar el número de factores extraídos.
    \item El PCA permite reducir la dimensionalidad de los datos y entender qué proporción de la variabilidad total explican los primeros componentes.
    \item A continuación, mostramos la variación explicada por cada componente.
\end{itemize}

\end{frame}

% Diapositiva 2: Tabla de Variación Total Analizada
\begin{frame}
\frametitle{Variación Explicada por los Componentes Principales}

\begin{table}[ht]
\centering
\scriptsize
\begin{tabular}{|c|c|c|}
\hline
\textbf{Componente} & \textbf{Varianza} & \textbf{Acumulada} \\ \hline
PC1 & 29.76\% & 29.76\% \\ \hline
PC2 & 18.27\% & 48.03\% \\ \hline
PC3 & 15.62\% & 63.65\% \\ \hline
PC4 & 11.04\% & 74.69\% \\ \hline
PC5 & 9.59\% & 84.28\% \\ \hline
PC6 & 6.61\% & 90.89\% \\ \hline
PC7 & 5.02\% & 95.90\% \\ \hline
PC8 & 4.10\% & 100.00\% \\ \hline
\end{tabular}
\end{table}

\begin{itemize}
    \item Los primeros tres componentes (PC1, PC2 y PC3) explican el 63.65\% de la varianza total.
    \item Se podría considerar que tres componentes son suficientes para representar los datos de manera efectiva.
\end{itemize}

\end{frame}


\begin{frame}{Gráfico de Codo - Determinación del Número de Factores}
  El gráfico de codo ayuda a identificar cuántos factores extraer. Se utilizó el criterio de Kaiser, que sugiere que se deben extraer solo aquellos factores con un valor propio (eigenvalue) mayor a 1.
    
    
    \begin{center}
        \includegraphics[width=0.6\textwidth]{fa_parallel_plot.png} % Inserta el archivo generado aquí
    \end{center}
\end{frame}



\begin{frame}
\frametitle{Matriz de Factores con Rotación Varimax}

\textbf{Análisis Realizado:}

Se realizó un análisis de factores con \textbf{3 factores} seleccionados, utilizando el método Varimax para la rotación de los factores. La rotación Varimax es una técnica ortogonal que busca maximizar la varianza de las cargas al cuadrado de cada factor, lo que facilita la interpretación de los factores.

\vspace{0.4cm}

\textbf{Resultados:} La matriz de factores obtenida muestra las cargas de cada variable en los factores seleccionados, lo que permite identificar qué variables están relacionadas con cada factor.

\end{frame}

\begin{frame}
\frametitle{Matriz de Factores con Rotación Varimax}

\textbf{Matriz de Factores con Rotación Varimax:}

\begin{table}[ht]
\centering
\begin{tabular}{|l|c|c|c|}
\hline
\textbf{Variable} & \textbf{Factor 1} & \textbf{Factor 2} & \textbf{Factor 3} \\
\hline
\textbf{IMC} & 0.70 & 0.49 & 0.26 \\
\hline
\textbf{ALT} & 0.70 & -0.18 & 0.34 \\
\hline
\textbf{LDL} & 0.08 & 0.26 & -0.58 \\
\hline
\textbf{Gluc} & 0.05 & 0.24 & 0.80 \\
\hline
\textbf{Creatinina Serica} & 0.13 & 0.72 & 0.10 \\
\hline
\textbf{TRI} & 0.80 & -0.20 & -0.14 \\
\hline
\textbf{Insulina} & 0.78 & 0.20 & -0.21 \\
\hline
\textbf{DHL} & 0.26 & -0.69 & 0.25 \\
\hline
\end{tabular}
\end{table}

\end{frame}


\begin{frame}
\frametitle{Identificación de Variables Latentes}

Las variables con las mayores cargas indican una fuerte relación con el factor correspondiente.

\vspace{0.4cm}

\textbf{Variables Latentes:}

\begin{itemize}
    \item \textbf{Factor 1:} Las variables \textbf{IMC}, \textbf{TRI} e \textbf{Insulina} tienen las mayores cargas, lo que sugiere que este factor está relacionado con la obesidad y la resistencia a la insulina.
    \item \textbf{Factor 2:} Las variables \textbf{Creatinina Serica} y \textbf{DHL} tienen las mayores cargas, lo que indica que este factor está asociado con el funcionamiento renal y la inflamación.
    \item \textbf{Factor 3:} La variable \textbf{Gluc} tiene la mayor carga, lo que sugiere que este factor está relacionado con el metabolismo de la glucosa.
\end{itemize}

\end{frame}

%----------------------------------------------------------------------------------------
%	CLUSTERING
%----------------------------------------------------------------------------------------

\begin{frame}{Metodología general del análisis de conglomerados}
    \begin{itemize}
        \item Se aplicarán cuatro algoritmos de clustering:
        \begin{enumerate}
            \item K-means
            \item Expectation Maximization (EM)
            \item Jerárquico
            \item DBSCAN (basado en densidad)
        \end{enumerate}
        \item Para cada uno, se reportará la validación interna mediante \textbf{C-index}, ya que no se cuenta con una clasificación real.
        \item El algoritmo jerárquico incluirá además el dendograma y el diagrama de codo con la medida de Silhouette, para validar el número óptimo de conglomerados.
    \end{itemize}
\end{frame}

\begin{frame}{¿Qué es el C-index?}
    \begin{itemize}
        \item El \textbf{C-index} es una medida de validación interna que evalúa la compacidad de los grupos formados por un algoritmo de clustering.
        \item Se define como:
        \[
            C = \frac{S - S_{min}}{S_{max} - S_{min}}
        \]
        donde:
        \begin{itemize}
            \item $S$ es la suma de las distancias entre todos los pares de puntos que pertenecen al mismo cluster
            \item $S_{min}$ es la suma mínima posible (pares más cercanos).
            \item $S_{max}$ es la suma máxima posible (pares más lejanos).
        \end{itemize}
        \item \textbf{Interpretación:} Cuanto más cercano a 0, mejor la agrupación.
    \end{itemize}
\end{frame}


\begin{frame}{Análisis jerárquico: Selección del número de grupos}
\begin{itemize}
    \item Para determinar el número adecuado de conglomerados, se utilizó el \textbf{índice de Silhouette}.
    \item Este índice evalúa qué tan bien se ajusta cada observación a su grupo comparado con otros grupos.
    \item Se calcularon los valores promedio del índice de Silhouette para $k$ entre 2 y 10.
    \item En el siguiente gráfico se observa que el valor máximo se alcanza en $\mathbf{k = 3}$.
    \item Por lo tanto, se elige dividir los datos en \textbf{3 conglomerados}.
\end{itemize}
\end{frame}

\begin{frame}{Análisis jerárquico: Selección del número de grupos}
\begin{center}
    \includegraphics[width=0.8\textwidth]{silhouette_jerarquico.png}
\end{center}
\end{frame}


\begin{frame}{Comparación de métodos por C-index}
\begin{itemize}
    \item Se construirán cuatro modelos de clustering:
    \begin{itemize}
        \item K-means
        \item Expectation Maximization (EM)
        \item Jerárquico
        \item DBSCAN
    \end{itemize}
    \item Cada modelo se evaluará con el \textbf{índice C-index}, que mide la compacidad de los grupos formados.
    \item Se seleccionará el método con el \textbf{menor C-index}, es decir, el que mejor agrupa las observaciones.
    \item Al método seleccionado se le aplicará su algoritmo \textbf{10 veces} para intentar minimizar aún más el C-index.
\end{itemize}
\end{frame}
\begin{frame}{Validación interna - K-means}
\begin{itemize}
    \item Se aplicó el algoritmo K-means con $k = 3$.
    \item Se obtuvo un \textbf{C-index de 0.1162249}.
    \item Este fue el valor más bajo entre los métodos evaluados.
    \item Se considera la mejor solución inicial.
\end{itemize}
\end{frame}

\begin{frame}{Visualización 3D - K-means}
\begin{center}
    \includegraphics[width=0.6\textwidth]{KMEANS.png}
\end{center}
\end{frame}

%--------------------------------------
\begin{frame}{Validación interna - Expectation Maximization (EM)}
\begin{itemize}
    \item Se aplicó el algoritmo de Expectation Maximization.
    \item El valor del C-index obtenido fue \textbf{0.2454119}.
    \item Agrupación menos compacta en comparación con K-means.
\end{itemize}
\end{frame}

\begin{frame}{Visualización 3D - EM}
\begin{center}
    \includegraphics[width=0.6\textwidth]{EM.png}
\end{center}
\end{frame}

%--------------------------------------
\begin{frame}{Validación interna - Jerárquico}
\begin{itemize}
    \item Se aplicó el clustering jerárquico con método completo.
    \item Se obtuvo un C-index de \textbf{0.1403033}.
    \item Mejor que EM, pero no tan bueno como K-means.
\end{itemize}
\end{frame}

\begin{frame}{Visualización 3D - Jerárquico}
\begin{center}
    \includegraphics[width=0.6\textwidth]{JERARQUICO.png}
\end{center}
\end{frame}

%--------------------------------------
\begin{frame}{Validación interna - DBSCAN}
\begin{itemize}
    \item Se aplicó el algoritmo DBSCAN con parámetros $\varepsilon = 1.5$, minPts = 2.
    \item El valor del C-index fue \textbf{0.2409209}.
    \item Resultado similar a EM, con menor cohesión de los grupos.
\end{itemize}
\end{frame}

\begin{frame}{Visualización 3D - DBSCAN}
\begin{center}
    \includegraphics[width=0.6\textwidth]{DBSCAN.png}
\end{center}
\end{frame}

\begin{frame}{Resumen de C-index por método}
\begin{center}
\begin{tabular}{ll}
\toprule
\textbf{Método} & \textbf{C-index} \\
\midrule
K-means & 0.1162249 \\
Expectation Maximization (EM) & 0.2454119 \\
Jerárquico & 0.1403033 \\
DBSCAN & 0.2409209 \\
\bottomrule
\end{tabular}
\end{center}
\vspace{0.5cm}
\begin{itemize}
    \item El método con el \textbf{menor C-index} fue \textbf{K-means}.
    \item Por lo tanto, se aplicará el algoritmo K-means \textbf{10 veces} para intentar obtener una solución aún más óptima.
\end{itemize}
\end{frame}

\begin{frame}{Repetición de K-means para optimización}
\begin{itemize}
    \item Se aplicó el algoritmo K-means 10 veces con diferentes semillas.
    \item Los valores de C-index obtenidos fueron:
    \begin{itemize}
        \item 0.1354802, 0.1162249, 0.1162249, 0.1354802, 0.1354802
        \item 0.1162249, 0.2881972, 0.1162249, 0.1162249, 0.1354802
    \end{itemize}
    \item El valor mínimo fue nuevamente \textbf{0.1162249}, obtenido en varias ejecuciones.
    \item Se confirma que la solución inicial ya era una de las más compactas posibles.
\end{itemize}
\end{frame}

\begin{frame}{Análisis de separación de grupos}
\begin{itemize}
    \item Una vez determinado el mejor modelo (K-means con $k=3$), se analizó si existen variables que ayudan a separar los grupos.
    \item Se calcularon las medias de los factores por grupo:
\end{itemize}
\begin{center}
\begin{tabular}{lccc}
\toprule
\textbf{Grupo} & \textbf{Factor 1} & \textbf{Factor 2} & \textbf{Factor 3} \\
\midrule
1 & 1.18 & -0.20 & -0.92 \\
2 & 0.10 &  1.60 &  0.63 \\
3 & -0.55 & -0.49 &  0.17 \\
\bottomrule
\end{tabular}
\end{center}
\vspace{0.3cm}
\begin{itemize}
    \item El \textbf{Factor 1} y el \textbf{Factor 2} muestran diferencias marcadas entre los grupos.
    \item Esto sugiere que son variables relevantes para diferenciar los conglomerados.
\end{itemize}
\end{frame}


\end{document}
