%%%%%%%%%%%%%%%%%%%%%%%%%%%%%%%%%%%%%%%%%
% Beamer Presentation
% LaTeX Template
% Version 2.0 (March 8, 2022)
%
% This template originates from:
% https://www.LaTeXTemplates.com
%
% Author:
% Vel (vel@latextemplates.com)
%
% License:
% CC BY-NC-SA 4.0 (https://creativecommons.org/licenses/by-nc-sa/4.0/)
%
%%%%%%%%%%%%%%%%%%%%%%%%%%%%%%%%%%%%%%%%%

%----------------------------------------------------------------------------------------
%	PACKAGES AND OTHER DOCUMENT CONFIGURATIONS
%----------------------------------------------------------------------------------------

\documentclass[
	11pt, % Set the default font size, options include: 8pt, 9pt, 10pt, 11pt, 12pt, 14pt, 17pt, 20pt
	%t, % Uncomment to vertically align all slide content to the top of the slide, rather than the default centered
	%aspectratio=169, % Uncomment to set the aspect ratio to a 16:9 ratio which matches the aspect ratio of 1080p and 4K screens and projectors
]{beamer}

\graphicspath{{Images/}{./}} % Specifies where to look for included images (trailing slash required)
\usepackage{colortbl}
\usepackage{booktabs} % Allows the use of \toprule, \midrule and \bottomrule for better rules in tables

%----------------------------------------------------------------------------------------
%	SELECT LAYOUT THEME
%----------------------------------------------------------------------------------------

\usetheme{Madrid}

%----------------------------------------------------------------------------------------
%	SELECT FONT THEME & FONTS
%----------------------------------------------------------------------------------------

\usefonttheme{default} % Typeset using the default sans serif font
\usepackage[default]{opensans} % Use the Open Sans font for sans serif text

%----------------------------------------------------------------------------------------
%	SELECT INNER THEME
%----------------------------------------------------------------------------------------

\useinnertheme{circles}

%----------------------------------------------------------------------------------------
%	PRESENTATION INFORMATION
%----------------------------------------------------------------------------------------

\title[Análisis de pacientes]{Análisis de pacientes menores a 15 años} % The short title in the optional parameter appears at the bottom of every slide, the full title in the main parameter is only on the title page

\subtitle{en el centro de salud en Puebla} % Presentation subtitle, remove this command if a subtitle isn't required

\author[Héctor Márquez]{Héctor Márquez} % Presenter name(s), the optional parameter can contain a shortened version to appear on the bottom of every slide, while the main parameter will appear on the title slide

\institute[UANL]{Universidad Autónoma de Nuevo León\\ \smallskip \textit{hector.marquez@uanl.edu.mx}} % Your institution, the optional parameter can be used for the institution shorthand and will appear on the bottom of every slide after author names, while the required parameter is used on the title slide and can include your email address or additional information on separate lines

\date[\today]{Métodos Estadísticos Multivariados \\ \today} % Presentation date or conference/meeting name, the optional parameter can contain a shortened version to appear on the bottom of every slide, while the required parameter value is output to the title slide

%----------------------------------------------------------------------------------------

\begin{document}

%----------------------------------------------------------------------------------------
%	TITLE SLIDE
%----------------------------------------------------------------------------------------

\begin{frame}
	\titlepage % Output the title slide, automatically created using the text entered in the PRESENTATION INFORMATION block above
\end{frame}

%----------------------------------------------------------------------------------------
%	TABLE OF CONTENTS SLIDE
%----------------------------------------------------------------------------------------

\begin{frame}
    \frametitle{Tabla de contenido} % Título de la diapositiva
    \begin{itemize}
        \item Objetivos
        \item Introducción
        \item Análisis descriptivo de los datos
        \begin{itemize}
        	\item Matriz de correlación
        	\item Histogramas
        	\item Prueba de normalidad
        	\item Promedios
        \end{itemize}
        \item Análisis Factorial
        \item Análisis de conglomerados 
    \end{itemize}
\end{frame}



%---------------------------------------------------
%----------------------------------------------------------------------------------------
%	ANALISIS DE FACTORES
%----------------------------------------------------------------------------------------
%------------------------------------------------------

\begin{frame}
  \frametitle{Análisis de factores}

    \begin{itemize}
    \item \textbf{Prueba de Esfericidad de Bartlett:} Determinar si las variables están correlacionadas.
    \item \textbf{Análisis de Componentes Principales (PCA):} Identificar los componentes principales y la varianza explicada.
    \item \textbf{Gráfico de Codo:} Identificar el número óptimo de factores a extraer.
    \item \textbf{Matriz de Factores:} Observar las cargas de las variables para interpretar los factores.
  \end{itemize}
 
\end{frame}

%----------------------------------------------------------------------------------------
%	BARTLETT
%----------------------------------------------------------------------------------------

\begin{frame}
\frametitle{Prueba de Esfericidad de Bartlett}

\begin{itemize}
    \item La prueba de esfericidad de Bartlett se realiza para verificar si la matriz de correlaciones es adecuada para realizar el análisis de factores.
    \item Resultado de la prueba de Bartlett:
    \begin{itemize}
        \item Valor chi-cuadrado: 72.13
        \item Valor p: \( 9.33 \times 10^{-6} \)
        \item Grados de libertad: 28
    \end{itemize}
    \item Con un valor p mucho menor que 0.05, rechazamos la hipótesis nula de que la matriz de correlaciones es una matriz identidad, lo que sugiere que el análisis de factores es adecuado para estos datos.
\end{itemize}

\end{frame}



% Diapositiva 1: Introducción al PCA
\begin{frame}
\frametitle{Análisis de Componentes Principales (PCA)}

\begin{itemize}
    \item Se realizó un Análisis de Componentes Principales (PCA) para determinar el número de factores extraídos.
    \item El PCA permite reducir la dimensionalidad de los datos y entender qué proporción de la variabilidad total explican los primeros componentes.
    \item A continuación, mostramos la variación explicada por cada componente.
\end{itemize}

\end{frame}

% Diapositiva 2: Tabla de Variación Total Analizada
\begin{frame}
\frametitle{Variación Explicada por los Componentes Principales}

\begin{table}[ht]
\centering
\scriptsize
\begin{tabular}{|c|c|c|}
\hline
\textbf{Componente} & \textbf{Varianza} & \textbf{Acumulada} \\ \hline
PC1 & 29.76\% & 29.76\% \\ \hline
PC2 & 18.27\% & 48.03\% \\ \hline
PC3 & 15.62\% & 63.65\% \\ \hline
PC4 & 11.04\% & 74.69\% \\ \hline
PC5 & 9.59\% & 84.28\% \\ \hline
PC6 & 6.61\% & 90.89\% \\ \hline
PC7 & 5.02\% & 95.90\% \\ \hline
PC8 & 4.10\% & 100.00\% \\ \hline
\end{tabular}
\end{table}

\begin{itemize}
    \item Los primeros tres componentes (PC1, PC2 y PC3) explican el 63.65\% de la varianza total.
    \item Se podría considerar que tres componentes son suficientes para representar los datos de manera efectiva.
\end{itemize}

\end{frame}


\begin{frame}{Gráfico de Codo - Determinación del Número de Factores}
  El gráfico de codo ayuda a identificar cuántos factores extraer. Se utilizó el criterio de Kaiser, que sugiere que se deben extraer solo aquellos factores con un valor propio (eigenvalue) mayor a 1.
    
    
    \begin{center}
        \includegraphics[width=0.6\textwidth]{fa_parallel_plot.png} % Inserta el archivo generado aquí
    \end{center}
\end{frame}



\begin{frame}
\frametitle{Matriz de Factores con Rotación Varimax}

\textbf{Análisis Realizado:}

Se realizó un análisis de factores con \textbf{3 factores} seleccionados, utilizando el método Varimax para la rotación de los factores. La rotación Varimax es una técnica ortogonal que busca maximizar la varianza de las cargas al cuadrado de cada factor, lo que facilita la interpretación de los factores.

\vspace{0.4cm}

\textbf{Resultados:} La matriz de factores obtenida muestra las cargas de cada variable en los factores seleccionados, lo que permite identificar qué variables están relacionadas con cada factor.

\end{frame}

\begin{frame}
\frametitle{Matriz de Factores con Rotación Varimax}

\textbf{Matriz de Factores con Rotación Varimax:}

\begin{table}[ht]
\centering
\begin{tabular}{|l|c|c|c|}
\hline
\textbf{Variable} & \textbf{Factor 1} & \textbf{Factor 2} & \textbf{Factor 3} \\
\hline
\textbf{IMC} & 0.70 & 0.49 & 0.26 \\
\hline
\textbf{ALT} & 0.70 & -0.18 & 0.34 \\
\hline
\textbf{LDL} & 0.08 & 0.26 & -0.58 \\
\hline
\textbf{Gluc} & 0.05 & 0.24 & 0.80 \\
\hline
\textbf{Creatinina Serica} & 0.13 & 0.72 & 0.10 \\
\hline
\textbf{TRI} & 0.80 & -0.20 & -0.14 \\
\hline
\textbf{Insulina} & 0.78 & 0.20 & -0.21 \\
\hline
\textbf{DHL} & 0.26 & -0.69 & 0.25 \\
\hline
\end{tabular}
\end{table}

\end{frame}


\begin{frame}
\frametitle{Identificación de Variables Latentes}

Las variables con las mayores cargas indican una fuerte relación con el factor correspondiente.

\vspace{0.4cm}

\textbf{Variables Latentes:}

\begin{itemize}
    \item \textbf{Factor 1:} Las variables \textbf{IMC}, \textbf{TRI} e \textbf{Insulina} tienen las mayores cargas, lo que sugiere que este factor está relacionado con la obesidad y la resistencia a la insulina.
    \item \textbf{Factor 2:} Las variables \textbf{Creatinina Serica} y \textbf{DHL} tienen las mayores cargas, lo que indica que este factor está asociado con el funcionamiento renal y la inflamación.
    \item \textbf{Factor 3:} La variable \textbf{Gluc} tiene la mayor carga, lo que sugiere que este factor está relacionado con el metabolismo de la glucosa.
\end{itemize}

\end{frame}


\end{document}
