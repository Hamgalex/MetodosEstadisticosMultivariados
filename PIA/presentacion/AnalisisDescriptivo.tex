%----------------------------------------------------------------------------------------
%	PRESENTATION BODY SLIDES
%----------------------------------------------------------------------------------------

\begin{frame}
    \frametitle{Objetivos} % Slide title
    \textbf{Objetivo del Proyecto:}
    \begin{itemize}
        \item Presentación de resultados donde se aplique una o varias técnicas multivariadas en el análisis de datos de varias variables aplicado a la solución de un problema real.
    \end{itemize}

    \vspace{0.5cm}

    \textbf{Objetivos Específicos:}
    \begin{itemize}
        \item Aplicar un análisis de factores para descubrir nuevas variables latentes y reducir la dimensionalidad del conjunto de datos sin perder información relevante.
        \item Implementar un análisis de conglomerados para agrupar a los pacientes en conjuntos homogéneos según sus características clínicas y evaluar la relación entre estos grupos y los factores identificados previamente.
    \end{itemize}
\end{frame}

%----------------------------------------------------------------------------------------
%	INTRODUCTION SLIDE
%----------------------------------------------------------------------------------------

\begin{frame}
    \frametitle{Introducción} % Slide title
    \begin{itemize}
        \item El conjunto de datos contiene información clínica de 50 pacientes menores de 15 años atendidos en el centro de salud Santa María Guadalupe Tecola, Puebla.
        \item Se incluyen múltiples variables biomédicas que permiten evaluar la salud general y detectar posibles afecciones.
        \item Estas variables serán analizadas mediante técnicas multivariadas para identificar patrones y relaciones ocultas en los datos.
    \end{itemize}
\end{frame}

%----------------------------------------------------------------------------------------
%	VARIABLE DESCRIPTION SLIDE
%----------------------------------------------------------------------------------------

\begin{frame}
    \frametitle{Descripción de las Variables} % Slide title
    \scriptsize % Reduce el tamaño de la fuente
    \begin{table}[]
        \centering
        \begin{tabular}{ll}
            \hline
            \textbf{Variable} & \textbf{Descripción} \\
            \hline
            IMC & Índice de Masa Corporal (kg/m²) \\
            ALT & Alanina aminotransferasa, indica daño hepático (UI/L) \\
            LDL & Colesterol LDL (mg/dL), asociado a enfermedades cardiovasculares \\
            Gluc & Glucosa en sangre (mg/dL), mide el nivel de azúcar \\
            Creatinina & Indicador de función renal (mg/dL) \\
            TRI & Triglicéridos en sangre (mg/dL) \\
            Insulina & Hormona que regula la glucosa (µU/mL) \\
            DHL & Deshidrogenasa láctica, relacionada con daño celular (UI/L) \\
            \hline
        \end{tabular}
    \end{table}
\end{frame}

%----------------------------------------------------------------------------------------
%	DESCRIPTIVE ANALYSIS SLIDE
%----------------------------------------------------------------------------------------

\begin{frame}
    \frametitle{Análisis descriptivo del conjunto de datos} % Slide title
    \begin{itemize}
        \item La matriz de correlación muestra la relación entre las diferentes variables del estudio.
        \item Valores cercanos a 1 o -1 indican correlaciones fuertes, mientras que valores cercanos a 0 indican poca relación.
        \item Se utilizarán estas correlaciones para seleccionar variables clave en el análisis.
    \end{itemize}
\end{frame}

%----------------------------------------------------------------------------------------
%	CORRELATION PLOT SLIDE
%----------------------------------------------------------------------------------------

\begin{frame}
    \frametitle{Matriz de Correlación} % Slide title
    \begin{center}
        \includegraphics[width=0.8\linewidth]{correlation_plot.png} % Ajusta el tamaño según sea necesario
    \end{center}
\end{frame}

%----------------------------------------------------------------------------------------
%	HISTOGRAMS
%----------------------------------------------------------------------------------------

\begin{frame}
    \frametitle{Histogramas} % Slide title
    \begin{center}
        \includegraphics[width=0.8\linewidth]{histogramas_pacientes.png} % Ajusta el tamaño según sea necesario
    \end{center}
\end{frame}

%----------------------------------------------------------------------------------------
%	prueba normalidad
%----------------------------------------------------------------------------------------


\begin{frame}
    \frametitle{Prueba de Normalidad (Shapiro-Wilk)}
    
    \scriptsize
    \begin{table}[]
        \centering
        \rowcolors{2}{gray!25}{white} % Colores alternados en filas
        \begin{tabular}{l c}
            \hline
            \textbf{Variable} & \textbf{p-valor de Shapiro-Wilk} \\
            \hline
            IMC & \textcolor{green}{0.0814} \\
            ALT & \textcolor{red}{4.91e-08} \\
            LDL & \textcolor{red}{0.0108} \\
            Glucosa & \textcolor{green}{0.6602} \\
            Creatinina & \textcolor{red}{3.68e-05} \\
            Triglicéridos & \textcolor{green}{0.0987} \\
            Insulina & \textcolor{red}{2.87e-04} \\
            DHL & \textcolor{red}{2.38e-04} \\
            \hline
        \end{tabular}
    \end{table}
    
    \vspace{0.4cm}
    \textbf{Conclusión:} IMC, Glucosa y Triglicéridos son compatibles con la normalidad.  
    ALT, LDL, Creatinina, Insulina y DHL \textbf{no} siguen una distribución normal.
\end{frame}


%----------------------------------------------------------------------------------------
%	prueba normalidad multivariada
%----------------------------------------------------------------------------------------


\begin{frame}{Prueba de Normalidad Multivariada}

    \textbf{Método:} Prueba de Mardia  

    \textbf{Resultados:}  
    \begin{table}[]
        \centering
        \begin{tabular}{lcc}
            \toprule
            \textbf{Prueba} & \textbf{Estadístico} & \textbf{p-valor} \\
            \midrule
            Mardia Skewness & 219.77 & $< 0.001$ \\
            Mardia Kurtosis & 3.30 & $0.00095$ \\
            \midrule
            \textbf{Conclusión} & \multicolumn{2}{c}{\textcolor{red}{No sigue una distribución normal multivariada}} \\
            \bottomrule
        \end{tabular}
    \end{table}

    \bigskip
    \textbf{Conclusión:}  
    Como los valores p son menores a 0.05, \textbf{rechazamos la hipótesis de normalidad multivariada}.  
    
\end{frame}

%----------------------------------------------------------------------------------------
%	vector promedios y desviacion muestral
%----------------------------------------------------------------------------------------


\begin{frame}
  \frametitle{Vector de Promedios y Desviación Muestral}

  \begin{table}[ht]
    \centering
    \begin{tabular}{|l|c|c|}
      \hline
      \textbf{Variable} & \textbf{Promedio} & \textbf{Desviación Muestral} \\
      \hline
      IMC & 20.3320 & 3.9617 \\
      ALT & 27.5880 & 19.1643 \\
      LDL & 74.6680 & 28.3022 \\
      Gluc & 89.0800 & 6.5085 \\
      CreatininaSerica & 0.5500 & 0.1111 \\
      TRI & 111.1000 & 39.4328 \\
      Insulina & 13.4648 & 6.8496 \\
      DHL & 354.2000 & 168.4154 \\
      \hline
    \end{tabular}
    \caption{Vector de Promedios y Desviación Muestral para cada variable}
  \end{table}

\end{frame}