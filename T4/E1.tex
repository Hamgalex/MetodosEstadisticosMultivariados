\begin{problem}{1}
Se colectaron datos sobre la contaminacion del aire en cierta ciudad. Suponer normalidad
en los datos.\\
A) Determinar si el vector de media poblacional es $\mu_0=(8,74,5,2,10,9,3)$ con un nivel
de significancia del 5\% incluir el p-valor. \\  
B) Obtener IC simultaneos para las medias de cada componente, con un nivel de confianza  
global del 95\%. \\  
C) Obtener IC simultaneos para las diferencias de medias poblacionales (ignorando  
unidades) con un nivel de confianza global del 95\%. \\  
D) Obtener IC simultaneos para la media de cada componente, con un nivel de confianza  
global del 95\% usando el metodo de Bonferroni.  
\end{problem}
\begin{sol}
\begin{itemize}
\item A) Determinar si el vector de media poblacional es $\mu_0=(8,74,5,2,10,9,3)$ con un nivel
de significancia del 5\% incluir el p-valor.\\\\
Debido a que el vector de medias poblacionales y la matriz de covarianzas poblacionales son desconocidas, entonces usaremos la prueba de hipótesis para $\mu$ con parámetros desconocidos. Sea:
$H_0:\mu = \mu_0 $ v.s. $H_1:\mu \neq \mu_0$.
El estadístico de prueba es:
\begin{align*}
T^2=n(\bar{\mathbf{X}} - \mu_0)'\mathbf{S}^{-1}(\mathbf{\bar{X}}-\mu_0)
\end{align*}
Para esto, necesitamos calcular dos cosas, $\mathbf{\bar{X}}$, $\mathbf{S}$ y $\mathbf{S}^{-1}$, lo haremos con R:\\\\
\includegraphics[width=1\textwidth]{img/1.png}\\\\
Esto nos da:
\begin{align*}
\mathbf{\bar{X}} &= \begin{pmatrix}
  7.5000 \\
  73.8571 \\
  4.5476 \\
  2.1905 \\
  10.0476 \\
  9.4048 \\
  3.0952
\end{pmatrix}\\
\mathbf{S} &= \begin{pmatrix}
  2.5000 & -2.7805 & -0.3780 & -0.4634 & -0.5854 & -2.2317 &  0.1707 \\
 -2.7805 & 300.5157 &  3.9094 & -1.3868 &  6.7631 & 30.7909 &  0.6237 \\
 -0.3780 &   3.9094 &  1.5221 &  0.6736 &  2.3148 &  2.8217 &  0.1417 \\
 -0.4634 &  -1.3868 &  0.6736 &  1.1823 &  1.0883 & -0.8107 &  0.1765 \\
 -0.5854 &   6.7631 &  2.3148 &  1.0883 & 11.3635 &  3.1266 &  1.0441 \\
 -2.2317 &  30.7909 &  2.8217 & -0.8107 &  3.1266 & 30.9785 &  0.5947 \\
  0.1707 &   0.6237 &  0.1417 &  0.1765 &  1.0441 &  0.5947 &  0.4785
\end{pmatrix} \\
\mathbf{S^{-1}} &= \begin{pmatrix}
  0.5651 &  0.0023 & -0.2490 &  0.4382 &  0.0639 &  0.0762 & -0.5267 \\
  0.0023 &  0.0038 & -0.0067 &  0.0082 & -0.0006 & -0.0026 & -0.0021 \\
 -0.2490 & -0.0067 &  1.8812 & -1.1633 & -0.3075 & -0.1991 &  0.8880 \\
  0.4382 &  0.0082 & -1.1633 &  1.8575 &  0.1171 &  0.1852 & -0.9935 \\
  0.0639 & -0.0006 & -0.3075 &  0.1171 &  0.1707 &  0.0263 & -0.3793 \\
  0.0762 & -0.0026 & -0.1991 &  0.1852 &  0.0263 &  0.0640 & -0.1701 \\
 -0.5267 & -0.0021 &  0.8880 & -0.9935 & -0.3793 & -0.1701 &  3.4233
\end{pmatrix}
\end{align*}
Ahora calcularemos esta operación en R, $T^2=n(\bar{\mathbf{X}} - \mu_0)'\mathbf{S}^{-1}(\mathbf{\bar{X}}-\mu_0)$: \\\\
\includegraphics[width=1\textwidth]{img/2.png}\\\\
Por tanto $T^2 = 27.0684$. Nuestra regla de desición será que se rechaza $H_0$ con nivel de significancia $\alpha$ si:
\begin{align*}
T^2 > \frac{(n-1)p}{n-p}F_{\alpha,p,n-p}
\end{align*}
\includegraphics[width=1\textwidth]{img/3.png}\\\\
Debido a que $T^2>18.7389$ entonces rechazamos $H_0$ y aceptamos la hipótesis alternativa. Esto quiere decir que el vector de medias poblacionales no es igual a $\mu_0$\\
Para calcular el p-valor tenemos que:
\begin{align*}
P(F>\frac{(n-p)T^2}{(n-1)p})
\end{align*}
A continuación calcularemos esta probabilidad en R:\\\\
\includegraphics[width=1\textwidth]{img/4.png}\\\\
Por tanto p-valor$=0.0084$. \pagebreak
\item B) Obtener IC simultaneos para las medias de cada componente, con un nivel de confianza  
global del 95\%. \\  
Para esto usaremos la fórmula para los intervalos de confianza simultáneos, esto es:
\begin{align*}
\bar{x}_i-\sqrt{\frac{p(n-1)}{n(n-p)}F_{\alpha,p,n-p}s_{ii}}<\mu_i<\bar{x_i}+\sqrt{\frac{p(n-1)}{n(n-p)}F_{\alpha,p,n-p}s_{ii}}
\end{align*}
Calcularemos esto en R para todas las medias:\\\\
\includegraphics[width=1\textwidth]{img/5.png}\\\\
Esto es:
\begin{align*}
 4.4757 < \mu_1 &< 10.5243 \\
 40.6992 < \mu_2 &< 107.0151 \\
 2.1878 < \mu_3 &< 6.9074 \\
0.1107 < \mu_4 &< 4.2703 \\
3.5998 < \mu_5 &< 16.4954 \\
 -1.2412 < \mu_6 &< 20.0507 \\
 1.7721 < \mu_7 &< 4.4184 \\
\end{align*}
\pagebreak
\item C) Obtener IC simultaneos para las diferencias de medias poblacionales (ignorando  
unidades) con un nivel de confianza global del 95\%. \\
Para esto usaremos la fórmula de IC simultáneos para la diferencia de medias:
\begin{align*}
(\bar{x_i}-\bar{x_j}) \pm \sqrt{\frac{p(n-1)}{n(n-p)}F_{\alpha,p,n-p}(s_{ii}-2s_{ij}+s_{jj})}
\end{align*}
Observamos que al tener 7 variables entonces habrá $\binom{7}{2} = 21$ intervalos. Lo calcularemos en R:\\\\
\includegraphics[width=1\textwidth]{img/6.png}\\\\
Nos dan los siguientes resultados:
\begin{align*}
 -99.9569 < \mu_{1} - \mu_{2} &< -32.7574 \\
 -1.2287 < \mu_{1} - \mu_{3} &< 7.1334 \\
 1.2031 < \mu_{1} - \mu_{4} &< 9.4160 \\
 -9.9641 < \mu_{1} - \mu_{5} &< 4.8688 \\
 -13.6866 < \mu_{1} - \mu_{6} &< 9.8771 \\
 1.2987 < \mu_{1} - \mu_{7} &< 7.5109 \\
 36.5008 < \mu_{2} - \mu_{3} &< 102.1183 \\
 38.2912 < \mu_{2} - \mu_{4} &< 105.0422 \\
 30.7711 < \mu_{2} - \mu_{5} &< 96.8480 \\
 33.0281 < \mu_{2} - \mu_{6} &< 95.8767 \\
 37.6464 < \mu_{2} - \mu_{7} &< 103.8775 \\
 0.1289 < \mu_{3} - \mu_{4} &< 4.5854 \\
 -10.9959 < \mu_{3} - \mu_{5} &< -0.0041 \\
 -14.7697 < \mu_{3} - \mu_{6} &< 5.0554 \\
 -1.0541 < \mu_{3} - \mu_{7} &< 3.9589 \\
 -14.0164 < \mu_{4} - \mu_{5} &< -1.6979 \\
 -18.3316 < \mu_{4} - \mu_{6} &< 3.9030 \\
 -3.0921 < \mu_{4} - \mu_{7} &< 1.2826 \\
 -10.8477 < \mu_{5} - \mu_{6} &< 12.1334 \\
 0.9787 < \mu_{5} - \mu_{7} &< 12.9261 \\
 -4.2136 < \mu_{6} - \mu_{7} &< 16.8326 \\
\end{align*}
\pagebreak
\item D) Obtener IC simultaneos para la media de cada componente, con un nivel de confianza  
global del 95\% usando el metodo de Bonferroni. \\
El método de Bonferroni nos dice que ternemos el siguiente IC:
\begin{align*}
\bar{x_i}-t_{\alpha / 2p,n-1}\sqrt{\frac{s_{ii}}{n}}<\mu_i<\bar{x_i}+t_{\alpha / 2p,n-1}\sqrt{\frac{s_{ii}}{n}}
\end{align*}
\includegraphics[width=1\textwidth]{img/7.png}\\\\
Esto es:
\begin{align*}
6.8091 < \mu_1 &< 8.1909 \\
66.2823 < \mu_2 &< 81.4320 \\
4.0085 < \mu_3 &< 5.0867 \\
1.7153 < \mu_4 &< 2.6656 \\
8.5746 < \mu_5 &< 11.5206 \\
6.9727 < \mu_6 &< 11.8368 \\
2.7930 < \mu_7 &< 3.3975 \\
\end{align*}
\end{itemize}
\end{sol}
