%%%%%%%%%%%%%%%%%%%%%%%%%%%%%%%%%%%%%%%%%%%%%%%%%%%%%%%%%%%%%%%%%%%%%%%%%%%%%%%%%%%%
%Do not alter this block of commands.  If you're proficient at LaTeX, you may include additional packages, create macros, etc. immediately below this block of commands, but make sure to NOT alter the header, margin, and comment settings here. 
\documentclass[12pt]{article}
 \usepackage[margin=1in]{geometry} 
\usepackage{amsmath,amsthm,amssymb,amsfonts, enumitem, fancyhdr, color, comment, graphicx, environ}
\pagestyle{fancy}
\setlength{\headheight}{65pt}
\newenvironment{problem}[2][Problema]{\begin{trivlist}
\item[\hskip \labelsep {\bfseries #1}\hskip \labelsep {\bfseries #2.}]}{\end{trivlist}}
\newenvironment{sol}
    {\emph{Solución:}
    }
    {
    }
\specialcomment{com}{ \color{blue} \textbf{Comment:} }{\color{black}} %for instructor comments while grading
\NewEnviron{probscore}{\marginpar{ \color{blue} \tiny Problem Score: \BODY \color{black} }}
%%%%%%%%%%%%%%%%%%%%%%%%%%%%%%%%%%%%%%%%%%%%%%%%%%%%%%%%%%%%%%%%%%%%%%%%%%%%%%%%%





%%%%%%%%%%%%%%%%%%%%%%%%%%%%%%%%%%%%%%%%%%%%%
%Fill in the appropriate information below
\lhead{Héctor Alejandro Márquez González}  %replace with your name
\rhead{Métodos Estadísticos Multivariados \\ 1989936} %replace XYZ with the homework course number, semester (e.g. ``Spring 2019"), and assignment number.
%%%%%%%%%%%%%%%%%%%%%%%%%%%%%%%%%%%%%%%%%%%%%


%%%%%%%%%%%%%%%%%%%%%%%%%%%%%%%%%%%%%%
%Do not alter this block.
\begin{document}
%%%%%%%%%%%%%%%%%%%%%%%%%%%%%%%%%%%%%%
%\begin{problem}{1}
Se colectaron datos sobre la contaminacion del aire en cierta ciudad. Suponer normalidad
en los datos.\\
A) Determinar si el vector de media poblacional es $\mu_0=(8,74,5,2,10,9,3)$ con un nivel
de significancia del 5\% incluir el p-valor. \\  
B) Obtener IC simultaneos para las medias de cada componente, con un nivel de confianza  
global del 95\%. \\  
C) Obtener IC simultaneos para las diferencias de medias poblacionales (ignorando  
unidades) con un nivel de confianza global del 95\%. \\  
D) Obtener IC simultaneos para la media de cada componente, con un nivel de confianza  
global del 95\% usando el metodo de Bonferroni.  
\end{problem}
\begin{sol}
\begin{itemize}
\item A) Determinar si el vector de media poblacional es $\mu_0=(8,74,5,2,10,9,3)$ con un nivel
de significancia del 5\% incluir el p-valor.\\\\
Debido a que el vector de medias poblacionales y la matriz de covarianzas poblacionales son desconocidas, entonces usaremos la prueba de hipótesis para $\mu$ con parámetros desconocidos. Sea:
$H_0:\mu = \mu_0 $ v.s. $H_1:\mu \neq \mu_0$.
El estadístico de prueba es:
\begin{align*}
T^2=n(\bar{\mathbf{X}} - \mu_0)'\mathbf{S}^{-1}(\mathbf{\bar{X}}-\mu_0)
\end{align*}
Para esto, necesitamos calcular dos cosas, $\mathbf{\bar{X}}$, $\mathbf{S}$ y $\mathbf{S}^{-1}$, lo haremos con R:\\\\
\includegraphics[width=1\textwidth]{img/1.png}\\\\
Esto nos da:
\begin{align*}
\mathbf{\bar{X}} &= \begin{pmatrix}
  7.5000 \\
  73.8571 \\
  4.5476 \\
  2.1905 \\
  10.0476 \\
  9.4048 \\
  3.0952
\end{pmatrix}\\
\mathbf{S} &= \begin{pmatrix}
  2.5000 & -2.7805 & -0.3780 & -0.4634 & -0.5854 & -2.2317 &  0.1707 \\
 -2.7805 & 300.5157 &  3.9094 & -1.3868 &  6.7631 & 30.7909 &  0.6237 \\
 -0.3780 &   3.9094 &  1.5221 &  0.6736 &  2.3148 &  2.8217 &  0.1417 \\
 -0.4634 &  -1.3868 &  0.6736 &  1.1823 &  1.0883 & -0.8107 &  0.1765 \\
 -0.5854 &   6.7631 &  2.3148 &  1.0883 & 11.3635 &  3.1266 &  1.0441 \\
 -2.2317 &  30.7909 &  2.8217 & -0.8107 &  3.1266 & 30.9785 &  0.5947 \\
  0.1707 &   0.6237 &  0.1417 &  0.1765 &  1.0441 &  0.5947 &  0.4785
\end{pmatrix} \\
\mathbf{S^{-1}} &= \begin{pmatrix}
  0.5651 &  0.0023 & -0.2490 &  0.4382 &  0.0639 &  0.0762 & -0.5267 \\
  0.0023 &  0.0038 & -0.0067 &  0.0082 & -0.0006 & -0.0026 & -0.0021 \\
 -0.2490 & -0.0067 &  1.8812 & -1.1633 & -0.3075 & -0.1991 &  0.8880 \\
  0.4382 &  0.0082 & -1.1633 &  1.8575 &  0.1171 &  0.1852 & -0.9935 \\
  0.0639 & -0.0006 & -0.3075 &  0.1171 &  0.1707 &  0.0263 & -0.3793 \\
  0.0762 & -0.0026 & -0.1991 &  0.1852 &  0.0263 &  0.0640 & -0.1701 \\
 -0.5267 & -0.0021 &  0.8880 & -0.9935 & -0.3793 & -0.1701 &  3.4233
\end{pmatrix}
\end{align*}
Ahora calcularemos esta operación en R, $T^2=n(\bar{\mathbf{X}} - \mu_0)'\mathbf{S}^{-1}(\mathbf{\bar{X}}-\mu_0)$: \\\\
\includegraphics[width=1\textwidth]{img/2.png}\\\\
Por tanto $T^2 = 27.0684$. Nuestra regla de desición será que se rechaza $H_0$ con nivel de significancia $\alpha$ si:
\begin{align*}
T^2 > \frac{(n-1)p}{n-p}F_{\alpha,p,n-p}
\end{align*}
\includegraphics[width=1\textwidth]{img/3.png}\\\\
Debido a que $T^2>18.7389$ entonces rechazamos $H_0$ y aceptamos la hipótesis alternativa. Esto quiere decir que el vector de medias poblacionales no es igual a $\mu_0$\\
Para calcular el p-valor tenemos que:
\begin{align*}
P(F>\frac{(n-p)T^2}{(n-1)p})
\end{align*}
A continuación calcularemos esta probabilidad en R:\\\\
\includegraphics[width=1\textwidth]{img/4.png}\\\\
Por tanto p-valor$=0.0084$. \pagebreak
\item B) Obtener IC simultaneos para las medias de cada componente, con un nivel de confianza  
global del 95\%. \\  
Para esto usaremos la fórmula para los intervalos de confianza simultáneos, esto es:
\begin{align*}
\bar{x}_i-\sqrt{\frac{p(n-1)}{n(n-p)}F_{\alpha,p,n-p}s_{ii}}<\mu_i<\bar{x_i}+\sqrt{\frac{p(n-1)}{n(n-p)}F_{\alpha,p,n-p}s_{ii}}
\end{align*}
Calcularemos esto en R para todas las medias:\\\\
\includegraphics[width=1\textwidth]{img/5.png}\\\\
Esto es:
\begin{align*}
6.4439 < \mu_1 &< 8.5561 \\
62.2779 < \mu_2 &< 85.4364 \\
3.7235 < \mu_3 &< 5.3717 \\
1.4642 < \mu_4 &< 2.9168 \\
7.7959 < \mu_5 &< 12.2993 \\
5.6870 < \mu_6 &< 13.1225 \\
2.6332 < \mu_7 &< 3.5573 \\
\end{align*}
\pagebreak
\item C) Obtener IC simultaneos para las diferencias de medias poblacionales (ignorando  
unidades) con un nivel de confianza global del 95\%. \\
Para esto usaremos la fórmula de IC simultáneos para la diferencia de medias:
\begin{align*}
(\bar{x_i}-\bar{x_j}) \pm \sqrt{\frac{p(n-1)}{n(n-p)}F_{\alpha,p,n-p}(s_{ii}-2s_{ij}+s_{jj})}
\end{align*}
Observamos que al tener 7 variables entonces habrá $\binom{7}{2} = 21$ intervalos. Lo calcularemos en R:\\\\
\includegraphics[width=1\textwidth]{img/6.png}\\\\
Nos dan los siguientes resultados:
\begin{align*}
-78.0907 < \mu_{1} - \mu_{2} &< -54.6236 \\
1.4923 < \mu_{1} - \mu_{3} &< 4.4125 \\
3.8755 < \mu_{1} - \mu_{4} &< 6.7436 \\
-5.1376 < \mu_{1} - \mu_{5} &< 0.0423 \\
-6.0192 < \mu_{1} - \mu_{6} &< 2.2096 \\
3.3201 < \mu_{1} - \mu_{7} &< 5.4895 \\
57.8522 < \mu_{2} - \mu_{3} &< 80.7668 \\
60.0114 < \mu_{2} - \mu_{4} &< 83.3219 \\
52.2720 < \mu_{2} - \mu_{5} &< 75.3471 \\
53.4785 < \mu_{2} - \mu_{6} &< 75.4262 \\
59.1975 < \mu_{2} - \mu_{7} &< 82.3264 \\
1.5790 < \mu_{3} - \mu_{4} &< 3.1353 \\
-7.4193 < \mu_{3} - \mu_{5} &< -3.5807 \\
-8.3187 < \mu_{3} - \mu_{6} &< -1.3955 \\
0.5771 < \mu_{3} - \mu_{7} &< 2.3277 \\
-10.0081 < \mu_{4} - \mu_{5} &< -5.7062 \\
-11.0966 < \mu_{4} - \mu_{6} &< -3.3320 \\
-1.6686 < \mu_{4} - \mu_{7} &< -0.1409 \\
-3.3698 < \mu_{5} - \mu_{6} &< 4.6555 \\
4.8663 < \mu_{5} - \mu_{7} &< 9.0385 \\
2.6347 < \mu_{6} - \mu_{7} &< 9.9844 \\
\end{align*}
\pagebreak
\item D) Obtener IC simultaneos para la media de cada componente, con un nivel de confianza  
global del 95\% usando el metodo de Bonferroni. \\
El método de Bonferroni nos dice que ternemos el siguiente IC:
\begin{align*}
\bar{x_i}-t_{\alpha / 2p,n-1}\sqrt{\frac{s_{ii}}{n}}<\mu_i<\bar{x_i}+t_{\alpha / 2p,n-1}\sqrt{\frac{s_{ii}}{n}}
\end{align*}
\includegraphics[width=1\textwidth]{img/7.png}\\\\
Esto es:
\begin{align*}
6.8091 < \mu_1 &< 8.1909 \\
66.2823 < \mu_2 &< 81.4320 \\
4.0085 < \mu_3 &< 5.0867 \\
1.7153 < \mu_4 &< 2.6656 \\
8.5746 < \mu_5 &< 11.5206 \\
6.9727 < \mu_6 &< 11.8368 \\
2.7930 < \mu_7 &< 3.3975 \\
\end{align*}
\end{itemize}
\end{sol}
 \pagebreak
\begin{problem}{2}
Se colectaron datos de tratamiento de radioterapia en muchos pacientes. El investigador considera que la muestra es muy grande, usar la teía de muestras muy grandes.\\
 A) Determinar cuáles vectores $\mathbf{\mu}$ están en la región de confianza del 95\%. 
\begin{align*}
\mu_1 = \begin{pmatrix} 3.60 \\ 2.00 \\ 2.10 \\ 2.15 \\ 2.60 \\ 1.30 \end{pmatrix}, 
\mu_2 = \begin{pmatrix} 3.60 \\ 1.90 \\ 2.10 \\ 2.15 \\ 2.60 \\ 1.30 \end{pmatrix}, 
\mu_3 = \begin{pmatrix} 3.60 \\ 2.00 \\ 2.10 \\ 2.15 \\ 3.00 \\ 1.30 \end{pmatrix}
\end{align*}
B) Con la teoría de muestras muy grandes obtener IC simultáneos con un nivel de confianza global del 95\% para la media de cada variable.\\
C) Resolver B con el método de Bonferroni.
\end{problem}
\begin{sol}
\begin{itemize}
\item A) Determinar cuáles vectores $\mathbf{\mu}$ están en la región de confianza del 95\%. \\
Debido a que es una muestra grande entonces haremos una prueba de hipótesis para cada uno de las entradas de $\mu$ en dónde $H_0 : \mu = \mu_i$ v.s. $H_1:\mu \neq \mu_i$ con: 
\begin{align*}
T^2 = n (\mathbf{\bar{X}} - \mu_i)' \mathbf{S}^{-1} (\mathbf{\bar{X}} - \mu_i)
\end{align*}
teniendo una distribución aproximada $\chi_p^2$ entonces se rechaza $H_0$ con nivel de significancia $\alpha$ si $T^2 > \chi_{(\alpha,p)}^2$. Haremos los cálculos en R:\\\\
\includegraphics[width=1\textwidth]{img/8.png}\\\\
Nos queda: 
\begin{align*}
\mathbf{\bar{X}} &=
\begin{pmatrix} 
  3.5423 \\ 
  1.8094 \\ 
  2.1376 \\ 
  2.2090 \\ 
  2.5748 \\ 
  1.2755 
\end{pmatrix} \\
\mathbf{S} &=
\begin{pmatrix}
  4.6548 &  0.9313 &  0.5897 &  0.2769 &  1.0749 &  0.1582 \\
  0.9313 &  0.6128 &  0.1109 &  0.1185 &  0.3889 & -0.0249 \\
  0.5897 &  0.1109 &  0.5714 &  0.0870 &  0.3480 &  0.1101 \\
  0.2769 &  0.1185 &  0.0870 &  0.1104 &  0.2174 &  0.0218 \\
  1.0749 &  0.3889 &  0.3480 &  0.2174 &  0.8622 & -0.0088 \\
  0.1582 & -0.0249 &  0.1101 &  0.0218 & -0.0088 &  0.8615
\end{pmatrix} \\
\mathbf{S^{-1}} &=
\begin{pmatrix}
  0.3660 & -0.3995 & -0.1578 &  0.1338 & -0.2468 & -0.0645 \\
 -0.3995 &  2.7908 &  0.4089 & -1.0291 & -0.6650 &  0.1208 \\
 -0.1578 &  0.4089 &  2.4975 &  0.0252 & -1.0050 & -0.2894 \\
  0.1338 & -1.0291 &  0.0252 & 18.6016 & -4.4092 & -0.5736 \\
 -0.2468 & -0.6650 & -1.0050 & -4.4092 &  3.2880 &  0.2999 \\
 -0.0645 &  0.1208 & -0.2894 & -0.5736 &  0.2999 &  1.2307
\end{pmatrix}
\end{align*}
Ahora calcularemos $T^2$:\\\\
\includegraphics[width=1\textwidth]{img/9.png}\\\\
Por tanto:
\begin{align*}
T_{\mu_1}^2=18.9681,\quad T_{\mu_2}^2=11.1078,\quad T_{\mu_3}^2=89.8923
\end{align*}
Ahora calcularemos $\chi_{(\alpha,p)}^2$:\\\\
\includegraphics[width=1\textwidth]{img/10.png}\\\\
Por tanto $\chi_{(\alpha,p)}^2=12.5915$.
\begin{itemize}
\item Para $\mu_1$ tenemos $T^2=18.9681 > 12.5915$ por tanto se rechaza $H_0$. No están en la región de confianza del 95\%.
\item Para $\mu_2$ tenemos $T^2=11.1078 > 12.5915$ por tanto no se rechaza $H_0$. Esto nos dice que $\mu_2$ sí esta en la región de confianza del 95\%.
\item Para $\mu_3$ tenemos $T^2=89.8923 > 12.5915$ por tanto se rechaza $H_0$.No están en la región de confianza del 95\%.
\end{itemize}
$\therefore$ El vector que está en la región de confianza del 95\% es el $\mu_2$ \pagebreak

\item B) Con la teoría de muestras muy grandes obtener IC simultáneos con un nivel de confianza global del 95\% para la media de cada variable.\\
Los intervalos los encontraremos con:
\begin{align*}
\bar{x_i}-\sqrt{\chi_{\alpha,p}^2(\frac{s_{ii}}{n})} < \mu_i < \bar{x_i}+\sqrt{\chi_{\alpha,p}^2(\frac{s_{ii}}{n})}
\end{align*}
Lo calcularemos con R:\\\\
\includegraphics[width=1\textwidth]{img/11.png}\\\\
Esto es: 
\begin{align*}
 1.5972 < \mu_1 &< 5.4875 \\
 1.1036 < \mu_2 &< 2.5151 \\
 1.4561 < \mu_3 &< 2.8191 \\
 1.9094 < \mu_4 &< 2.5086 \\
 1.7377 < \mu_5 &< 3.4120 \\
 0.4387 < \mu_6 &< 2.1123 \\
\end{align*} \pagebreak
\item C) Resolver B con el método de Bonferroni. \\
El método de Bonferroni nos dice que ternemos el siguiente IC:
\begin{align*}
\bar{x_i}-t_{\alpha / 2p,n-1}\sqrt{\frac{s_{ii}}{n}}<\mu_i<\bar{x_i}+t_{\alpha / 2p,n-1}\sqrt{\frac{s_{ii}}{n}}
\end{align*}
\includegraphics[width=1\textwidth]{img/12.png}\\\\
Esto es:
\begin{align*}
2.9553 < \mu_1 &< 4.1294 \\
1.5964 < \mu_2 &< 2.0223 \\
1.9319 < \mu_3 &< 2.3433 \\
2.1186 < \mu_4 &< 2.2994 \\
2.3222 < \mu_5 &< 2.8275 \\
1.0230 < \mu_6 &< 1.5280 \\
\end{align*}

\end{itemize}
\end{sol}

%%%%%%%%%%%%%%%%%%%%%%%%%%%%%%%%%%%%%
%Do not alter anything below this line.
\end{document}