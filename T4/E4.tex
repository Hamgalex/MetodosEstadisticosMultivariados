\begin{problem}{4}
Se colectaron datos sobre el puntaje de exámenes en estudiantes universitarios. \\
A) Determinar que vectores de medias están en la región de confianza del 95\%.
\begin{align*}
\mu_1 = \begin{pmatrix}527 \\ 53 \\ 25 \end{pmatrix}, 
\mu_2 = \begin{pmatrix}523 \\ 55 \\ 26 \end{pmatrix}, 
\mu_3 = \begin{pmatrix} 525 \\ 53 \\ 26 \end{pmatrix}
\end{align*}
B) Obtener IC simultáneos para diferencias de medias con un nivel de confianza global del 95\%.\\
C) Mencionar los supuestos para que los resultados anteriores sean válidos.
\end{problem}

\begin{sol}
\begin{itemize}
\item A) Determinar que vectores de medias están en la región de confianza del 95\%. \\
Debido a que $n=87$ y $p=3$ entonces es muestra grande, por lo que haremos una prueba de hipótesis para cada uno de las entradas de $\mu$ en dónde $H_0 : \mu = \mu_i$ v.s. $H_1:\mu \neq \mu_i$. El estadístico de prueba es:
\begin{align*}
T^2=n(\bar{\mathbf{X}} - \mu_0)'\mathbf{S}^{-1}(\mathbf{\bar{X}}-\mu_0)
\end{align*}
Para esto, necesitamos calcular tres cosas, $\mathbf{\bar{X}}$, $\mathbf{S}$ y $\mathbf{S}^{-1}$, lo haremos con R:\\\\
\includegraphics[width=1\textwidth]{img/19.png}\\\\
\begin{align*}
    \mathbf{\bar{X}}&= \begin{pmatrix} 
        526.5862 \\ 
        54.6897 \\ 
        25.1264 
    \end{pmatrix} \\
    \mathbf{S} &= \begin{pmatrix}
        5808.0593 & 597.8352 & 222.0297 \\ 
        597.8352 & 126.0537 & 23.3885 \\ 
        222.0297 & 23.3885 & 23.1117 
    \end{pmatrix} \\
    \mathbf{S^{-1}} &= \begin{pmatrix}
        0.0004 & -0.0016 & -0.0026 \\ 
        -0.0016 & 0.0155 & -0.0006 \\ 
        -0.0026 & -0.0006 & 0.0684
    \end{pmatrix}
\end{align*}
Ahora calcularemos $T^2$:\\\\
\includegraphics[width=1\textwidth]{img/20.png}\\\\
Por tanto:
\begin{align*}
T_{\mu_1}^2=4.1462,\quad T_{\mu_2}^2=6.8261 \quad T_{\mu_3}^2=8.5145
\end{align*}
Nuestra regla de desición será que se rechaza $H_0$ con nivel de significancia $\alpha$ si:
\begin{align*}
T^2 >\chi_{\alpha, p}^2
\end{align*}
Lo calcularemos en R:\\\\
\includegraphics[width=1\textwidth]{img/21.png}\\
\begin{itemize}
\item Para $\mu_1$ tenemos $T^2=4.1462 < 7.8147$ por tanto no hay suficiente evidencia para rechazar $H_0$. Esto nos dice que $\mu_1$ sí esta en la región de confianza del 95\%.
\item Para $\mu_2$ tenemos $T^2=6.8261< 7.8147$ por tanto no hay suficiente evidencia para rechazar $H_0$. Esto nos dice que $\mu_2$ sí esta en la región de confianza del 95\%.
\item Para $\mu_3$ tenemos $T^2=8.5145 >7.8147$ por tanto se rechaza $H_0$. No están en la región de confianza del 95\%.
\end{itemize} 
$\therefore$ Los vectores que está en la región de confianza del 95\% son $\mu_1,\mu_2$ \pagebreak

\item B) Obtener IC simultáneos para diferencias de medias con un nivel de confianza global del 95\%.\\ 
Para esto usaremos la fórmula de ic simultáneos para muestras grandes:
\begin{align*}
(\bar{x}_i-\bar{x}_j)\pm \sqrt{\chi_{\alpha,p}^2(s_{ii}-2s_{ij}+s_{jj})/n}
\end{align*}
\includegraphics[width=1\textwidth]{img/22.png}\\
Esto es:
\begin{align*}
451.2658 < \mu_1 - \mu_2 &< 492.5273 \\
479.4622 < \mu_1 - \mu_3 &< 523.4574 \\
26.5306 < \mu_2 - \mu_3 &< 32.5959 \\
\end{align*} \pagebreak

\item C) Mencionar los supuestos para que los resultados anteriores sean válidos. \\
Para que lo que acabamos de hacer sea válido, debe de ser una muestra grande, esto se checa con $n-p$ y recordemos que nosotros tenemos $n=87$ y $p=3$ por lo que $n-p=84>40$ pues 40 es neustro mínimo para considerar grande a una muestra. Debido a esto, no es necesario hacer un test de normalidad.
\end{itemize}

\end{sol}