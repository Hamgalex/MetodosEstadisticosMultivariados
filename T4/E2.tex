\begin{problem}{2}
Se colectaron datos de tratamiento de radioterapia en muchos pacientes. El investigador considera que la muestra es muy grande, usar la teía de muestras muy grandes.\\
 A) Determinar cuáles vectores $\mathbf{\mu}$ están en la región de confianza del 95\%. 
\begin{align*}
\mu_1 = \begin{pmatrix} 3.60 \\ 2.00 \\ 2.10 \\ 2.15 \\ 2.60 \\ 1.30 \end{pmatrix}, 
\mu_2 = \begin{pmatrix} 3.60 \\ 1.90 \\ 2.10 \\ 2.15 \\ 2.60 \\ 1.30 \end{pmatrix}, 
\mu_3 = \begin{pmatrix} 3.60 \\ 2.00 \\ 2.10 \\ 2.15 \\ 3.00 \\ 1.30 \end{pmatrix}
\end{align*}
B) Con la teoría de muestras muy grandes obtener IC simultáneos con un nivel de confianza global del 95\% para la media de cada variable.\\
C) Resolver B con el método de Bonferroni.
\end{problem}
\begin{sol}
\begin{itemize}
\item A) Determinar cuáles vectores $\mathbf{\mu}$ están en la región de confianza del 95\%. \\
Debido a que es una muestra grande entonces haremos una prueba de hipótesis para cada uno de las entradas de $\mu$ en dónde $H_0 : \mu = \mu_i$ v.s. $H_1:\mu \neq \mu_i$ con: 
\begin{align*}
T^2 = n (\mathbf{\bar{X}} - \mu_i)' \mathbf{S}^{-1} (\mathbf{\bar{X}} - \mu_i)
\end{align*}
teniendo una distribución aproximada $\chi_p^2$ entonces se rechaza $H_0$ con nivel de significancia $\alpha$ si $T^2 > \chi_{(\alpha,p)}^2$. Haremos los cálculos en R:\\\\
\includegraphics[width=1\textwidth]{img/8.png}\\\\
Nos queda: 
\begin{align*}
\mathbf{\bar{X}} &=
\begin{pmatrix} 
  3.5423 \\ 
  1.8094 \\ 
  2.1376 \\ 
  2.2090 \\ 
  2.5748 \\ 
  1.2755 
\end{pmatrix} \\
\mathbf{S} &=
\begin{pmatrix}
  4.6548 &  0.9313 &  0.5897 &  0.2769 &  1.0749 &  0.1582 \\
  0.9313 &  0.6128 &  0.1109 &  0.1185 &  0.3889 & -0.0249 \\
  0.5897 &  0.1109 &  0.5714 &  0.0870 &  0.3480 &  0.1101 \\
  0.2769 &  0.1185 &  0.0870 &  0.1104 &  0.2174 &  0.0218 \\
  1.0749 &  0.3889 &  0.3480 &  0.2174 &  0.8622 & -0.0088 \\
  0.1582 & -0.0249 &  0.1101 &  0.0218 & -0.0088 &  0.8615
\end{pmatrix} \\
\mathbf{S^{-1}} &=
\begin{pmatrix}
  0.3660 & -0.3995 & -0.1578 &  0.1338 & -0.2468 & -0.0645 \\
 -0.3995 &  2.7908 &  0.4089 & -1.0291 & -0.6650 &  0.1208 \\
 -0.1578 &  0.4089 &  2.4975 &  0.0252 & -1.0050 & -0.2894 \\
  0.1338 & -1.0291 &  0.0252 & 18.6016 & -4.4092 & -0.5736 \\
 -0.2468 & -0.6650 & -1.0050 & -4.4092 &  3.2880 &  0.2999 \\
 -0.0645 &  0.1208 & -0.2894 & -0.5736 &  0.2999 &  1.2307
\end{pmatrix}
\end{align*}
Ahora calcularemos $T^2$:\\\\
\includegraphics[width=1\textwidth]{img/9.png}\\\\
Por tanto:
\begin{align*}
T_{\mu_1}^2=18.9681,\quad T_{\mu_2}^2=11.1078,\quad T_{\mu_3}^2=89.8923
\end{align*}
Ahora calcularemos $\chi_{(\alpha,p)}^2$:\\\\
\includegraphics[width=1\textwidth]{img/10.png}\\\\
Por tanto $\chi_{(\alpha,p)}^2=12.5915$.
\begin{itemize}
\item Para $\mu_1$ tenemos $T^2=18.9681 > 12.5915$ por tanto se rechaza $H_0$. No están en la región de confianza del 95\%.
\item Para $\mu_2$ tenemos $T^2=11.1078 > 12.5915$ por tanto no se rechaza $H_0$. Esto nos dice que $\mu_2$ sí esta en la región de confianza del 95\%.
\item Para $\mu_3$ tenemos $T^2=89.8923 > 12.5915$ por tanto se rechaza $H_0$.No están en la región de confianza del 95\%.
\end{itemize}
$\therefore$ El vector que está en la región de confianza del 95\% es el $\mu_2$ \pagebreak

\item B) Con la teoría de muestras muy grandes obtener IC simultáneos con un nivel de confianza global del 95\% para la media de cada variable.\\
Los intervalos los encontraremos con:
\begin{align*}
\bar{x_i}-\sqrt{\chi_{\alpha,p}^2(\frac{s_{ii}}{n})} < \mu_i < \bar{x_i}+\sqrt{\chi_{\alpha,p}^2(\frac{s_{ii}}{n})}
\end{align*}
Lo calcularemos con R:\\\\
\includegraphics[width=1\textwidth]{img/11.png}\\\\
Esto es: 
\begin{align*}
 2.7690 < \mu_1 &< 4.3157 \\
 1.5288 < \mu_2 &< 2.0900 \\
 1.8666 < \mu_3 &< 2.4086 \\
 2.0899 < \mu_4 &< 2.3281 \\
 2.2420 < \mu_5 &< 2.9077 \\
 0.9428 < \mu_6 &< 1.6082 \\
\end{align*} \pagebreak
\item C) Resolver B con el método de Bonferroni. \\
El método de Bonferroni nos dice que ternemos el siguiente IC:
\begin{align*}
\bar{x_i}-z_{\alpha/ 2p}\sqrt{\frac{s_{ii}}{n}}<\mu_i<\bar{x_i}+z_{\alpha/ 2p}\sqrt{\frac{s_{ii}}{n}}
\end{align*}
\includegraphics[width=1\textwidth]{img/12.png}\\\\
Esto es:
\begin{align*}
2.9674 < \mu_1 &< 4.1173 \\
1.6007 < \mu_2 &< 2.0180 \\
1.9361 < \mu_3 &< 2.3391 \\
2.1204 < \mu_4 &< 2.2976 \\
2.3274 < \mu_5 &< 2.8223 \\
1.0282 < \mu_6 &< 1.5229 \\
\end{align*}
\end{itemize}
\end{sol}
