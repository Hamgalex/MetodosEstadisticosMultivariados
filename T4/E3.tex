\begin{problem}{3}
Se tiene una muestra aleatoria de datos sobre minerales contenidos en los huesos. \\
A) Determinar si $\mu_0'=(0.85,0.79,1.80,1.70,0.7,0.7)$ con un nivel de significancia del 5\%.\\
B) Obtener IC simultáneos para las diferencias de medias con un nivel de significancia global del 95\%.\\
C) Validar los supuestos para que los resultados anteriores sean válidos. 
\end{problem}

\begin{sol}
\begin{itemize}
\item A) Determinar si $\mu_0=(0.85,0.79,1.80,1.70,0.7,0.7)$ con un nivel de significancia del 5\%.\\\\
Debido a que el vector de medias poblacionales y la matriz de covarianzas poblacionales son desconocidas, entonces usaremos la prueba de hipótesis para $\mu$ con parámetros desconocidos. Sea:
$H_0:\mu = \mu_0 $ v.s. $H_1:\mu \neq \mu_0$.
El estadístico de prueba es:
\begin{align*}
T^2=n(\bar{\mathbf{X}} - \mu_0)'\mathbf{S}^{-1}(\mathbf{\bar{X}}-\mu_0)
\end{align*}
Para esto, necesitamos calcular dos cosas, $\mathbf{\bar{X}}$, $\mathbf{S}$ y $\mathbf{S}^{-1}$, lo haremos con R:\\\\
\includegraphics[width=1\textwidth]{img/13.png}\\\\
\begin{align*}
    \mathbf{\bar{X}}&= \begin{pmatrix} 0.8438 \\ 0.8183 \\ 1.7927 \\ 1.7348 \\ 0.7044 \\ 0.6938 \end{pmatrix} \\
    \mathbf{S} &= \begin{pmatrix}
        0.0130 & 0.0104 & 0.0223 & 0.0201 & 0.0091 & 0.0080 \\
        0.0104 & 0.0114 & 0.0185 & 0.0211 & 0.0085 & 0.0089 \\
        0.0223 & 0.0185 & 0.0804 & 0.0668 & 0.0168 & 0.0128 \\
        0.0201 & 0.0211 & 0.0668 & 0.0695 & 0.0177 & 0.0168 \\
        0.0091 & 0.0085 & 0.0168 & 0.0177 & 0.0116 & 0.0081 \\
        0.0080 & 0.0089 & 0.0128 & 0.0168 & 0.0081 & 0.0106
    \end{pmatrix} \\
    \mathbf{S^{-1}} &= \begin{pmatrix}
        500.7803 & -400.8874 & -142.4433 & 143.6613 & -99.3684 & -18.3431 \\
        -400.8874 & 704.2568 & 118.2579 & -173.2867 & 3.7850 & -162.5968 \\
        -142.4433 & 118.2579 & 112.3374 & -116.3159 & 2.3127 & 53.9227 \\
        143.6613 & -173.2867 & -116.3159 & 154.1152 & -14.5927 & -54.3027 \\
        -99.3684 & 3.7850 & 2.3127 & -14.5927 & 249.6035 & -98.3283 \\
        -18.3431 & -162.5968 & 53.9227 & -54.3027 & -98.3283 & 340.3370
    \end{pmatrix}
\end{align*}
Ahora calcularemos esta operación en R, $T^2=n(\bar{\mathbf{X}} - \mu_0)'\mathbf{S}^{-1}(\mathbf{\bar{X}}-\mu_0)$: \\\\
\includegraphics[width=1\textwidth]{img/14.png}\\\\
Por tanto $T^2 = 15.4904$. Nuestra regla de desición será que se rechaza $H_0$ con nivel de significancia $\alpha$ si:
\begin{align*}
T^2 > \frac{(n-1)p}{n-p}F_{\alpha,p,n-p}
\end{align*}
\includegraphics[width=1\textwidth]{img/15.png}\\\\
Debido a que $T^2<19.9198$ no hay suficiente evidencia para rechazar $H_0$, por lo que podremos decir que el vector de medias es  $\mu_0'=(0.85,0.79,1.80,1.70,0.7,0.7)$ \pagebreak

\item B) Obtener IC simultáneos para las diferencias de medias con un nivel de significancia global del 95\%.\\
Para esto usaremos la fórmula de IC simultáneos para la diferencia de medias:
\begin{align*}
(\bar{x_i}-\bar{x_j}) \pm \sqrt{\frac{p(n-1)}{n(n-p)}F_{\alpha,p,n-p}(s_{ii}-2s_{ij}+s_{jj})}
\end{align*}
Observamos que al tener 6 variables entonces habrá $\binom{6}{2} = 15$ intervalos. Lo calcularemos en R:\\\\
\includegraphics[width=1\textwidth]{img/16.png}\\\\
Nos dan los siguientes resultados:
\begin{align*}
    -0.1232 < \mu_{1} - \mu_{2} &< 0.1742 \\
    -1.4910 < \mu_{1} - \mu_{3} &< -0.4068 \\
    -1.3965 < \mu_{1} - \mu_{4} &< -0.3855 \\
    -0.0561 < \mu_{1} - \mu_{5} &< 0.3349 \\
    -0.0655 < \mu_{1} - \mu_{6} &< 0.3654 \\
    -1.5491 < \mu_{2} - \mu_{3} &< -0.3996 \\
    -1.4000 < \mu_{2} - \mu_{4} &< -0.4331 \\
    -0.0753 < \mu_{2} - \mu_{5} &< 0.3031 \\
    -0.0348 < \mu_{2} - \mu_{6} &< 0.2837 \\
    -0.2558 < \mu_{3} - \mu_{4} &< 0.3715 \\
    0.4952 < \mu_{3} - \mu_{5} &< 1.6814 \\
    0.4711 < \mu_{3} - \mu_{6} &< 1.7266 \\
    0.5058 < \mu_{4} - \mu_{5} &< 1.5551 \\
    0.5111 < \mu_{4} - \mu_{6} &< 1.5709 \\
    -0.1802 < \mu_{5} - \mu_{6} &< 0.2013
\end{align*} \pagebreak
\item C) Validar los supuestos para que los resultados anteriores sean válidos. \\
Para que nuestros resultados sean válidos, nuestras observaciones deben de ser independientes, lo cual se asume debido a que es una muestra aleatoria y también deben de seguir una distribución normal multivariada, para probar esto, nos basaremos en las distancias generalizadas al cuadrado:
\begin{align*}
d_i^2=(\mathbf{X_i-\bar{X}})'\mathbf{S}^{-1}(\mathbf{X_i-\bar{X}})
\end{align*}
Debido a que $n=25$ y $p=6$ entonces cada una de las distancias al cuadrado debe seguir una distribución chi cuadrada. A continuación evaluaremos el ajuste a través de QQ-plots de $d_i^2$ vs $\chi_p^2$ mediante R, usando la función  mahalanobis para acelerar el proceso de sacar todas las distancias y la función qchisq para sacar los cuantiles de la distribución chi cuadrado con 4 grados de libertad. Luego graficaremos estos datos con la función qqplot. \\\\
\includegraphics[width=1\textwidth]{img/17.png}
\includegraphics[width=1\textwidth]{img/18.png}
Debido a que los cuantiles teóricos y los cuantiles muestrales se alinean a la línea de referencia, entonces la distribución de distancias se aproxima a la chi cuadrada con 6 grados de libertad. Esto quiere decir que el conjunto de datos tiene una distribución normal multivariante.\\

\end{itemize}
\end{sol}