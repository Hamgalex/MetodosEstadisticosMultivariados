%%%%%%%%%%%%%%%%%%%%%%%%%%%%%%%%%%%%%%%%%%%%%%%%%%%%%%%%%%%%%%%%%%%%%%%%%%%%%%%%%%%%
%Do not alter this block of commands.  If you're proficient at LaTeX, you may include additional packages, create macros, etc. immediately below this block of commands, but make sure to NOT alter the header, margin, and comment settings here. 
\documentclass[12pt]{article}
 \usepackage[margin=1in]{geometry} 
\usepackage{amsmath,amsthm,amssymb,amsfonts, enumitem, fancyhdr, color, comment, graphicx, environ}
\pagestyle{fancy}
\setlength{\headheight}{65pt}
\newenvironment{problem}[2][Problema]{\begin{trivlist}
\item[\hskip \labelsep {\bfseries #1}\hskip \labelsep {\bfseries #2.}]}{\end{trivlist}}
\newenvironment{sol}
    {\emph{Solución:}
    }
    {
    }
\specialcomment{com}{ \color{blue} \textbf{Comment:} }{\color{black}} %for instructor comments while grading
\NewEnviron{probscore}{\marginpar{ \color{blue} \tiny Problem Score: \BODY \color{black} }}
%%%%%%%%%%%%%%%%%%%%%%%%%%%%%%%%%%%%%%%%%%%%%%%%%%%%%%%%%%%%%%%%%%%%%%%%%%%%%%%%%





%%%%%%%%%%%%%%%%%%%%%%%%%%%%%%%%%%%%%%%%%%%%%
%Fill in the appropriate information below
\lhead{Héctor Alejandro Márquez González}  %replace with your name
\rhead{Métodos Estadísticos Multivariados \\ 1989936} %replace XYZ with the homework course number, semester (e.g. ``Spring 2019"), and assignment number.
%%%%%%%%%%%%%%%%%%%%%%%%%%%%%%%%%%%%%%%%%%%%%


%%%%%%%%%%%%%%%%%%%%%%%%%%%%%%%%%%%%%%
%Do not alter this block.
\begin{document}
%%%%%%%%%%%%%%%%%%%%%%%%%%%%%%%%%%%%%%


%Solutions to problems go below.  Please follow the guidelines from https://www.overleaf.com/read/sfbcjxcgsnsk/


%Copy the following block of text for each problem in the assignment.
\begin{problem}{I} 
Considera las matrices: $\textbf{A}=
\begin{bmatrix}
2 & 0 & 7 & 8\\
6 & 4 & 6 & -2 \\
0 & 7 & 9 & 2 \\
6 & 5 & 4 & -8\\
\end{bmatrix},
\textbf{B}= 
\begin{bmatrix}
2 & 0 \\
-5 & 3 \\
6 & 9 \\
0 & -2 \\
\end{bmatrix},
\textbf{C}=
\begin{bmatrix}
3 & -3 & 6 & 9 \\
4 & 0 & -2 & 0 \\
\end{bmatrix}$
\\ \\
Calcular: a)\textbf{B'A}\:b)\textbf{2A+5B} \: c)\textbf{3BC+4A} \: d)\textbf{5C'-6B} \: e)\textbf{BA}
\end{problem}
\begin{sol}\\ \\
a)\textbf{B'A}

\begin{align*}
\textbf{B'} &=
	\begin{bmatrix}
		2 & -5 & 6 & 0 \\
		0 & 3 & 9 & -2 \\
	\end{bmatrix}	 \\
\textbf{B'A} &= 
 	\begin{bmatrix}
		2 & -5 & 6 & 0 \\
		0 & 3 & 9 & -2 \\
	\end{bmatrix}
	\begin{bmatrix}
		2 & 0 & 7 & 8\\
		6 & 4 & 6 & -2 \\
		0 & 7 & 9 & 2 \\
		6 & 5 & 4 & -8\\
	\end{bmatrix}
\end{align*}
Debido a que \textbf{B'} es de 2x4 y \textbf{A} es de 4x4 entonces si es posible hacer el producto de matrices, donde el resultado será de 2x4.
\begin{align*}
\textbf{B'A} &= 
 	\begin{bmatrix}
		2 & -5 & 6 & 0 \\
		0 & 3 & 9 & -2 \\
	\end{bmatrix}
	\begin{bmatrix}
		2 & 0 & 7 & 8\\
		6 & 4 & 6 & -2 \\
		0 & 7 & 9 & 2 \\
		6 & 5 & 4 & -8\\
	\end{bmatrix}\\
	&=
	\begin{bmatrix}
		4-30 & -20+42 & 14-30+54 & 16+10+12 \\
		18-12 & 12+63-10 & 18+81-8 & -6+18+16
	\end{bmatrix}\\
	&=
	\begin{bmatrix}
		-26 & 22 & 38 & 38 \\
		6 & 65 & 91 & 28
	\end{bmatrix}
\end{align*}
b) \textbf{2A+5B}\\
Observamos que \textbf{A} es de 4x4 y \textbf{B} es de 4x2, entonces no podremos calcular 2\textbf{A} + 5\textbf{B} porque son de distintas dimensiones.\\ \\
c) \textbf{3BC}+ 4\textbf{A}\\
Primero calcularemos 3\textbf{BC} en donde \textbf{B} es de 4x2 y \textbf{C} es de 2x4 por lo tanto es posible hacer el producto y el resultado será de 4x4.\\
\begin{align*}
	\textbf{BC} &= 
	\begin{bmatrix}
		2 & 0 \\
		-5 & 3 \\
		6 & 9 \\
		0 & -2
	\end{bmatrix}
	\begin{bmatrix}
		3 & -3 & 6 & 9 \\
		4 & 0 & -2 & 0
	\end{bmatrix} \\
	&= \begin{bmatrix}
		6 & 6 & 12 & 18 \\
		-15+12 & 15 & -30-6 & -45 \\
		18+36 & -18 & 36-18 & 54 \\
		-8 & 0 & 4 & 0
	\end{bmatrix}\\
	&= \begin{bmatrix}
		6 & -6 & 12 & 18 \\
		-3 & 15 & -36 & -45 \\
		54 & -18 & 18 & 54 \\
		-8 & 0 & 4 & 0
	\end{bmatrix}\\
	3\textbf{BC}&=
	\begin{bmatrix}
		18 & -18 & 36 & 54 \\
		-9 & 45 & -108 & -135 \\
		162 & -54 & 54 & 162 \\
		-24 & 0 & 12 & 0
	\end{bmatrix}\\
	\end{align*}
Ahora 4\textbf{A}:
\begin{align*}
	4\textbf{A} &= 4
	\begin{bmatrix}
		2 & 0 & 7 & 8 \\
		6 & 4 & 6 & -2 \\
		0 & 7 & 9 & 2 \\
		6 & 5 & 4 & -8
	\end{bmatrix} \\
	&= \begin{bmatrix}
		8 & 0 & 28 & 32 \\
		24 & 16 & 24 & -8 \\
		0 & 28 & 36 & 8 \\
		24 & 20 & 16 & -32
	\end{bmatrix}
\end{align*}
Por tanto:
\begin{align*}
3\textbf{BC}+4\textbf{A} &=
\begin{bmatrix}
18 & -18 & 36 & 54 \\
-9 & 45 & -108 & -135 \\
162 & -54 & 54 & 162 \\
-24 & 0 & 12 & 0
\end{bmatrix}
+
\begin{bmatrix}
8 & 0 & 28 & 32 \\
24 & 16 & 24 & -8 \\
0 & 28 & 36 & 8 \\
24 & 20 & 16 & -32
\end{bmatrix} \\
&= \begin{bmatrix}
26 & -18 & 64 & 86 \\
15 & 61 & -84 & -143 \\
162 & -26 & 90 & 170 \\
0 & 20 & 28 & -32
\end{bmatrix}
\end{align*}

d)\textbf{5C'-6B} \\
Primero calcularemos transpocisión de \textbf{C}, esto es:\\
\begin{align*}
\textbf{C'}=
\begin{bmatrix}
	3 & 4 \\
	-3 & 0 \\
	6 & -2 \\
	9 & 0 
\end{bmatrix}
\end{align*}
Ahora 5\textbf{C'}:
\begin{align*}
5\textbf{C'} &= 5
\begin{bmatrix}
	3 & 4 \\
	-3 & 0 \\
	6 & -2 \\
	9 & 0 
\end{bmatrix}
=
\begin{bmatrix}
	15 & 20 \\
	-15 & 0 \\
	30 & -10 \\
	45 & 0 
\end{bmatrix}
\end{align*}
Ahora 6\textbf{B}
\begin{align*}
6\textbf{B} &=6
\begin{bmatrix}
2 & 0 \\
-5 & 3 \\
6 & 9 \\
0 & -2 \\
\end{bmatrix}
=
\begin{bmatrix}
12 & 0 \\
-30 & 18 \\
36 & 54 \\
0 & -12 \\
\end{bmatrix}
\end{align*}
Por tanto:
\begin{align*}
5\textbf{C'}-6\textbf{B} &=
\begin{bmatrix}
	15 & 20 \\
	-15 & 0 \\
	30 & -10 \\
	45 & 0 
\end{bmatrix}
-
\begin{bmatrix}
12 & 0 \\
-30 & 18 \\
36 & 54 \\
0 & -12 \\
\end{bmatrix} \\
&=
\begin{bmatrix}
3 & 20 \\
15 & -18 \\
-6 & -64 \\
45 & 12 \\
\end{bmatrix}
\end{align*}
e) \textbf{BA} \\
Observamos que \textbf{B} es de 4x2 y \textbf{A} es de 4x4, entonces no podremos calcular \textbf{B}\textbf{A} porque para que suceda el producto deben ser de dimensiones MxN y NxP
\end{sol}

\pagebreak

\begin{problem}{2} 
En los siguientes sistemas de ecuaciones lineales determinar las matrices \textbf{A}, \textbf{X}, \textbf{C} para poder
expresarlas como \textbf{AX} $=$ \textbf{C}. \\
\begin{align*}
a) 
\begin{matrix}
6x+2y=5 \\
3x+y=8
\end{matrix}
\quad b)
\begin{matrix}
4x-5y+10z =0 \\
6x+2y-5z=12 \\
5x+3y = 9
\end{matrix}
\quad
c)\begin{matrix}
2x+5y-8z+4w = 6\\
6x + 9z -10w =8 \\
4x +3y -9w = 15 \\
7x - 8z =30
\end{matrix}
\end{align*}
\end{problem}
\begin{sol} \\ \\
$
a) \quad 
\begin{matrix}
6x+2y=5 \\
3x+y=8
\end{matrix}
$ \\
Tenemos:
\begin{align*}
\textbf{A} &=
\begin{bmatrix}
6 & 2 \\
3 & 1 
\end{bmatrix} 
\\
\textbf{X} &=
\begin{bmatrix}
x \\
y
\end{bmatrix}
\\
\textbf{C} &= 
\begin{bmatrix}
5\\
8
\end{bmatrix}
\end{align*}
Por tanto, podemos expresar ese sistema de ecuaciones como: 
\begin{align*}
\textbf{AX} &= \textbf{C} \\
\begin{bmatrix}
6 & 2 \\
3 & 1 
\end{bmatrix} 
\begin{bmatrix}
x \\
y
\end{bmatrix}
&=
\begin{bmatrix}
5\\
8
\end{bmatrix}
\end{align*}
$b)\quad
\begin{matrix}
4x-5y+10z =0 \\
6x+2y-5z=12 \\
5x+3y = 9
\end{matrix}$
Tenemos:
\begin{align*}
\textbf{A} &=
\begin{bmatrix}
4 & -5 & 10 \\
6 & 2 & -5 \\
5 & 3 & 0 
\end{bmatrix} 
\\
\textbf{X} &=
\begin{bmatrix}
x \\
y\\
z
\end{bmatrix}
\\
\textbf{C} &= 
\begin{bmatrix}
0\\
12\\
9
\end{bmatrix}
\end{align*}
Por tanto, podemos expresar ese sistema de ecuaciones como: 
\begin{align*}
\textbf{AX} &= \textbf{C} \\
\begin{bmatrix}
4 & -5 & 10 \\
6 & 2 & -5 \\
5 & 3 & 0 
\end{bmatrix} 
\begin{bmatrix}
x \\
y\\
z
\end{bmatrix}
&=
\begin{bmatrix}
0\\
12\\
9
\end{bmatrix}
\end{align*}

$
c) \quad
\begin{matrix}
2x+5y-8z+4w = 6\\
6x + 9z -10w =8 \\
4x +3y -9w = 15 \\
7x - 8z =30
\end{matrix}
$
Tenemos:
\begin{align*}
\textbf{A} &=
\begin{bmatrix}
2 & 5 & -8 & 4 \\ 
6 & 0 & 9 & -10 \\
4 & 3 & 0 & -9 \\
7 & 0 & -8 & 0
\end{bmatrix} 
\\
\textbf{X} &=
\begin{bmatrix}
x \\
y\\
z\\
w
\end{bmatrix}
\\
\textbf{C} &= 
\begin{bmatrix}
6\\
8\\
15\\
30
\end{bmatrix}
\end{align*}
Por tanto, podemos expresar ese sistema de ecuaciones como: 
\begin{align*}
\textbf{AX} &= \textbf{C} \\
\begin{bmatrix}
2 & 5 & -8 & 4 \\ 
6 & 0 & 9 & -10 \\
4 & 3 & 0 & -9 \\
7 & 0 & -8 & 0
\end{bmatrix}  
\begin{bmatrix}
x \\
y\\
z\\
w
\end{bmatrix}
&=
\begin{bmatrix}
6\\
8\\
15\\
30
\end{bmatrix}
\end{align*}

\end{sol}

\pagebreak

\begin{problem}{3}
Verificar las siguientes afirmaciones:\\
a) Si \textbf{A} = 
$
\begin{bmatrix} 
2 & 5 \\
4 & 8 
\end{bmatrix} 
\rightarrow 
\mathbf{A}^{-1} = 
\begin{bmatrix}
-2 & 1.25 \\ 
1 & -0.5
\end{bmatrix}
$
\\ \\
b) Si \textbf{B}=
$
\begin{bmatrix}
2&4&0\\
-3&5&2\\
8&0&-2
\end{bmatrix}
\rightarrow
\mathbf{B}^{-1}=
\begin{bmatrix}
-0.5 & 0.4 & 0.4\\
0.5 & -0.2 & -0.2\\
-2 & 1.6 & 1.1
\end{bmatrix}
$
\\ \\
c) Si \textbf{C}=
$
\begin{bmatrix}
0&0&0&-5\\
1&1&1&0\\
0&2&0&1\\
6&0&1&1
\end{bmatrix}
\rightarrow
\mathbf{C}^{-1}=
\begin{bmatrix}
0.06 & -0.2 & 0.1 & 0.2\\
0.1 & 0 & 0.5&0\\
-0.16 & 1.2 & -0.6 & -0.2 \\
-0.2 & 0 & 0 & 0
\end{bmatrix}
$
\end{problem}
\begin{sol}
Para verificar si esas matrices son las inversas, usaremos la propiedad que nos dice:
$\mathbf{AA^{-1}}$= $\mathbf{A^{-1}A}$ = \textbf{I}, donde \textbf{I} es la matriz identidad.\\ \\
a) 
\begin{align*}
\mathbf{AA^{-1}} &= 
\begin{bmatrix} 
2 & 5 \\
4 & 8 
\end{bmatrix} 
\begin{bmatrix}
-2 & 1.25 \\ 
1 & -0.5
\end{bmatrix}\\
&= 
\begin{bmatrix}
-4 + 5 & 2.5 - 2.5 \\
-8 + 8 & 5 - 4
\end{bmatrix}\\
&= 
\begin{bmatrix}
1 & 0 \\
0 & 1
\end{bmatrix}\\
&= \mathbf{I}
\end{align*}
Debido a que el resultado del producto nos da la matriz identidad, entonces la afirmación a) es correcta. \\ \\ 
b)
\begin{align*}
\mathbf{BB^{-1}} &= 
\begin{bmatrix}
2&4&0\\
-3&5&2\\
8&0&-2
\end{bmatrix}
\begin{bmatrix}
-0.5 & 0.4 & 0.4\\
0.5 & -0.2 & -0.2\\
-2 & 1.6 & 1.1
\end{bmatrix} \\
&=
\begin{bmatrix}
-1 + 2 + 0 & 0.8 - 0.8 + 0 & 0.8 - 0.8 + 0 \\
1.5 + 2.5 - 4 & -1.2 - 1 + 3.2 & -1.2 - 1 + 2.2 \\
-4 + 0 + 4 & 3.2 + 0 - 3.2 & 3.2 + 0 - 2.2
\end{bmatrix}
\\ &=
\begin{bmatrix}
1 & 0 & 0 \\
0 & 1 & 0 \\
0 & 0 & 1
\end{bmatrix}\\
&= \mathbf{I}
\end{align*}
Debido a que el resultado del producto nos da la matriz identidad, entonces la afirmación b) es correcta. \\ \\ 
c)
\begin{align*}
\mathbf{CC^{-1}} &= 
\begin{bmatrix}
0&0&0&-5\\
1&1&1&0\\
0&2&0&1\\
6&0&1&1
\end{bmatrix}
\begin{bmatrix}
0.06 & -0.2 & 0.1 & 0.2\\
0.1 & 0 & 0.5&0\\
-0.16 & 1.2 & -0.6 & -0.2 \\
-0.2 & 0 & 0 & 0
\end{bmatrix} \\
&= 
\begin{bmatrix}
(-5)(-0.2) & 0 & 0 & 0 \\
0.06 + 0.1 -0.16 & -0.2 +  1.2  & 0.1 + 0.5 -0.6 & 0.2 -0.2  \\
(2)(0.1) -0.2 & 0 &  (2)(0.5) & 0 \\
(6)(0.06) -0.16 -0.2 & (6)(-0.2) +  1.2  & (6)(0.1) -0.6 & (6)(0.2) -0.2 
\end{bmatrix} \\
&=
\begin{bmatrix}
1 & 0 & 0 & 0 \\
0 & 1 & 0 & 0 \\
0 & 0 & 1 & 0 \\
0 & 0 & 0 & 1
\end{bmatrix}\\
&= \mathbf{I}
\end{align*}
Debido a que el resultado del producto nos da la matriz identidad, entonces la afirmación c) es correcta. \\ \\ 
\end{sol}

\pagebreak

\begin{problem}{4}
Sea \textbf{A} una matriz tal que:
\begin{align*}
\mathbf{A}=
\begin{bmatrix}
a_{11} & 0 & 0 & 0 \\
0 & a_{22} & 0 & 0 \\
0 & 0 & a_{33} & 0 \\
0 & 0 & 0 & a_{44} \\
\end{bmatrix}
\end{align*}
Con $a_{ii}\neq 0$ para todo $i$, verificar que:
\begin{align*}
\mathbf{A}^{-1}=
\begin{bmatrix}
\frac{1}{a_{11}} & 0 & 0 & 0 \\
0 & \frac{1}{a_{22}} & 0 & 0 \\
0 & 0 & \frac{1}{a_{33}} & 0 \\
0 & 0 & 0 & \frac{1}{a_{44}} \\
\end{bmatrix}
\end{align*}
\end{problem}
\begin{sol}
Para verificar si esta matriz es la inversa, usaremos la propiedad que nos dice:
$\mathbf{AA^{-1}}$= $\mathbf{A^{-1}A}$ = \textbf{I}, donde \textbf{I} es la matriz identidad.\\ \\
\begin{align*}
\mathbf{AA^{-1}} &=
\begin{bmatrix}
a_{11} & 0 & 0 & 0 \\
0 & a_{22} & 0 & 0 \\
0 & 0 & a_{33} & 0 \\
0 & 0 & 0 & a_{44} \\
\end{bmatrix}
\begin{bmatrix}
\frac{1}{a_{11}} & 0 & 0 & 0 \\
0 & \frac{1}{a_{22}} & 0 & 0 \\
0 & 0 & \frac{1}{a_{33}} & 0 \\
0 & 0 & 0 & \frac{1}{a_{44}} \\
\end{bmatrix} \\
&= 
\begin{bmatrix}
a_{11} \cdot \frac{1}{a_{11}} & 0 & 0 & 0 \\
0 & a_{22} \cdot\frac{1}{a_{22}} & 0 & 0 \\
0 & 0 & a_{33} \cdot\frac{1}{a_{33}} & 0 \\
0 & 0 & 0 & a_{44} \cdot\frac{1}{a_{44}} \\
\end{bmatrix}\\
&=
\begin{bmatrix}
1 & 0 & 0 & 0 \\
0 & 1 & 0 & 0 \\
0 & 0 & 1 & 0 \\
0 & 0 & 0 & 1 \\
\end{bmatrix} \\
&= \mathbf{I}
\end{align*}
Debido a que el resultado del producto nos da la matriz identidad, entonces $\mathbf{A}^{-1}$ es la inversa de \textbf{A} 
\end{sol}

\pagebreak

\begin{problem}{5}
Verificar que la matriz
\begin{align*}
\mathbf{C} = 
\begin{bmatrix}
2 & -2 & -4 \\
-1 & 3 & 4 \\
1 & -2 & -3 
\end{bmatrix}
\end{align*}
Es idempotente $(\mathbf{C^2}=\mathbf{CC}=\mathbf{C})$
\end{problem}
\begin{sol}
Haremos la multiplicación $\mathbf{CC}$ y si nos da el resultado igual a $\mathbf{C}$ entonces es idempotente:
\begin{align*}
\mathbf{CC} &= 
\begin{bmatrix}
2 & -2 & -4 \\
-1 & 3 & 4 \\
1 & -2 & -3 
\end{bmatrix}
\begin{bmatrix}
2 & -2 & -4 \\
-1 & 3 & 4 \\
1 & -2 & -3 
\end{bmatrix} \\
&=
\begin{bmatrix}
4 + 2 - 4 & -4 - 6 + 8 & -8 - 8 + 12 \\
-2 - 3 + 4 & 2 + 9 - 8 & 4 + 12 - 12 \\
2 + 2 - 3 & -2 - 6 + 6 & -4 - 8 + 9
\end{bmatrix}\\
&=
\begin{bmatrix}
2 & -2 & -4 \\
-1 & 3 & 4 \\
1 & -2 & -3 
\end{bmatrix}\\
&=
\mathbf{C}
\end{align*}
Debido a que $\mathbf{CC}=\mathbf{C}$ entonces $\mathbf{C}$ es idempotente.
\end{sol}

\pagebreak

\begin{problem}{6}
Dar ejemplo de una matriz \textbf{A} de orden 2x2 que no sea matriz cero ( no todos los elementos son ceros) y $\mathbf{A^2}=\mathbf{0}$ ( donde \textbf{0} es la matriz cero).
\end{problem}

\begin{sol}
Expresaremos $\mathbf{A^2}=\mathbf{0}$ con matrices:
\begin{align*}
\mathbf{A^2}=\mathbf{AA} &= \mathbf{0} \\
\begin{bmatrix}
a & b \\
c & d \\
\end{bmatrix}
\begin{bmatrix}
a & b \\
c & d \\
\end{bmatrix}
&=
\begin{bmatrix}
0 & 0 \\
0 & 0 \\
\end{bmatrix}\\
\begin{bmatrix}
a^2+bc & ab+bd \\
ca+cd & bc + d^2
\end{bmatrix}
&= 
\begin{bmatrix}
0 & 0 \\
0 & 0
\end{bmatrix}
\end{align*}
De esta igualdad obtenemos el siguiente sistema de ecuaciones:
\begin{align*}
(1) \quad a^2+bc &=0 \\
(2) \quad ab+bd &= 0\\
(3) \quad ca+cd &= 0\\
(4) \quad bc + d^2 &= 0
\end{align*}
Entonces $\mathbf{A^2}=\mathbf{0}$ se va a cumplir si sus entradas satisfacen ese sistema de ecuaciones.\\
Digamos que $c=0$ entonces:
\begin{align*}
a^2 &=0 \\
ab+bd &= 0\\
d^2 &= 0
\end{align*}
$a=0 , d=0 $ y $b$ puede ser cualquier número real, por tanto, una matriz de forma:
\begin{align*}
\begin{bmatrix}
0 & b \\
0 & 0
\end{bmatrix}
\end{align*}
Cumplirá el enunciado, por ejemplo, si $b=1$:
\begin{align*}
\mathbf{A^2}=\mathbf{AA}=
\begin{bmatrix}
0 & 1 \\
0 & 0
\end{bmatrix}
\begin{bmatrix}
0 & 1 \\
0 & 0
\end{bmatrix}
&=\begin{bmatrix}
0 & 0 \\
0 & 0
\end{bmatrix}
\\
&= \mathbf{0}
\end{align*}

\end{sol}

\pagebreak

\begin{problem}{7}
Determina el valor de $x$ para que la matriz 
$\begin{bmatrix}
	x & x+1 \\
	2 & x+3 
\end{bmatrix}$ sea singular, sea no singular.
\end{problem}

\begin{sol}
Para que la matriz sea singular su determinante tiene que ser igual a cero, le llamaremos \textbf{A} a la matriz del enunciado.
\begin{align*}
\mathbf{A}&=
\begin{bmatrix}
	x & x+1 \\
	2 & x+3 
\end{bmatrix}\\
|\mathbf{A}| &= 
x(x+3) -[ 2(x+1)] = 0 \\
&= x^2+3x -2x-2 = 0 \\
&= x^2 +x -2 =0\\
&= (x+2)(x-1)=0\\
x&=-2\\
x&=1
\end{align*}
Por tanto, la matriz será singular si $x=-2$ ó $x=1$ y NO será singular si es cualquier número real excepto $-2$ y $1$.
\end{sol}

\pagebreak

\begin{problem}{8}
Obtener la norma de los siguientes vectores: $a)\quad\mathbf{v}=[2,-5]' \in \Re^2$,\\
 $b)\quad\mathbf{w}=[1,-2,8]' \in \Re^3$, $\quad c) \quad\mathbf{c}=[2,5,-3,1,-1]' \in \Re^5$
\end{problem}
\begin{sol}
\begin{align*}
||\mathbf{v}|| &= \sqrt{2^2+(-5)^2}=\sqrt{4+25}=\sqrt{29}\approx  5.3851\\
||\mathbf{w}|| &= \sqrt{1^2+(-2)^2+8^2} = \sqrt{1+4+64} = \sqrt{69} \approx 8.3066 \\
||\mathbf{c}|| &=  \sqrt{2^2+5^2+(-3)^2+1^2+(-1)^2} = \sqrt{4+25+9+1+1} = \sqrt{40} = 2\sqrt{10} \approx 6.3245
\end{align*}


\end{sol}

\pagebreak

\begin{problem}{9}
Leer el documento “derivadas y eigen analisis.pdf” y analizar los ejemplos de Eigenvalor y Eigenvector,
después obtener los valores y vectores característicos de la matriz indicada.
\begin{align*}
a)\quad \textbf{A} &=
\begin{bmatrix}
1&2 \\
3&0
\end{bmatrix} \\
b) \quad \textbf{A} &= 
\begin{bmatrix}
2 & 0 & 0\\
0 & 3 & 0 \\
0 & 0 & 5 
\end{bmatrix}
\end{align*}
\end{problem}
\begin{sol}\\
a)\\
1) Tenemos que:
\begin{align*}
\mathbf{A}-\lambda\mathbf{I} &= 
\begin{bmatrix}
1 & 2\\
3 & 0
\end{bmatrix}
-
\begin{bmatrix}
\lambda & 0 \\
0 & \lambda
\end{bmatrix}
=
\begin{bmatrix}
1-\lambda & 2 \\
3 & -\lambda
\end{bmatrix}
\end{align*}
Luego:
\begin{align*}
p(\lambda)=det(\mathbf{A}-\lambda\mathbf{I}) &= 
(1-\lambda)(-\lambda) -6  \\
&= -\lambda + \lambda^2 -6 \\
&= \lambda^2 - \lambda -6
\end{align*}
2) Entonces $p(\lambda)= \lambda^2 - \lambda -6$, y sus raíces se obtienen resolviendo $\lambda^2 - \lambda -6 = 0$. Luego $(\lambda-3)(\lambda+2)= 0$ y los eigenvalores son: $\lambda_1 = 3,\lambda_2=-2$.\\ \\
3) Se resuelve el sistema homogéneo $(\mathbf{A}-\lambda\mathbf{I})\mathbf{v}=\mathbf{0}$ para $\lambda_1=3$.
\begin{align*}
\begin{bmatrix}
1-3 & 2 \\
3 & -3
\end{bmatrix}
\begin{bmatrix}
x\\y
\end{bmatrix}
&= 
\begin{bmatrix}
-2 & 2 \\
3 & -3
\end{bmatrix}
\begin{bmatrix}
x\\y
\end{bmatrix} =
\begin{bmatrix}
-2x+2y\\3x-3y
\end{bmatrix}=
\begin{bmatrix}
0\\0
\end{bmatrix}
\end{align*}
Esto es equivalente a la ecuación $-2x+2y=x-y=0$. Por tanto $x=y$. Haciendo $x=1$ tenemos que $y=1$ y el eigenvector correspondiente a $\lambda_1=3$ es $(1,1)$.\\\\
Ahora se resuelve el sistema homogéneo $(\mathbf{A}-\lambda\mathbf{I})\mathbf{v}=\mathbf{0}$ para $\lambda_2=-2$.
\begin{align*}
\begin{bmatrix}
1+2 & 2 \\
3 & 2
\end{bmatrix}
\begin{bmatrix}
x\\y
\end{bmatrix}
&= 
\begin{bmatrix}
3 & 2 \\
3 & 2
\end{bmatrix}
\begin{bmatrix}
x\\y
\end{bmatrix} =
\begin{bmatrix}
3x+2y\\3x+2y
\end{bmatrix}=
\begin{bmatrix}
0\\0
\end{bmatrix}
\end{align*}
Esto es equivalente a la ecuación $3x+2y=0$. Por tanto $y=-\frac{3x}{2}$. Haciendo $x=1$ tenemos que $y=-\frac{3}{2}$ y el eigenvector correspondiente a $\lambda_2=2$ es $(1,-\frac{3}{2})$.\\
Por tanto, los valores característicos son $\lambda_1=3,\lambda_2=-2$ y sus vectores característicos son:
\begin{align*}\mathbf{v_1}=
\begin{bmatrix}
1\\1
\end{bmatrix},\mathbf{v_2}=
\begin{bmatrix}
1 \\ -\frac{3}{2}
\end{bmatrix}
\end{align*} \\ \\
b)\\
\begin{align*}
\mathbf{A}-\lambda\mathbf{I} &= 
\begin{bmatrix}
2 & 0 & 0\\
0 & 3 & 0 \\
0 & 0 & 5 
\end{bmatrix}
-
\begin{bmatrix}
\lambda & 0 & 0 \\
0 & \lambda & 0 \\
0 & 0 & \lambda
\end{bmatrix} 
=
\begin{bmatrix}
2-\lambda & 0 & 0\\
0 & 3-\lambda & 0 \\
0 & 0 & 5-\lambda
\end{bmatrix}
\end{align*}
Luego:
\begin{align*}
p(\lambda)=det(\mathbf{A}-\lambda\mathbf{I}) &= 
(2-\lambda) ( 3-\lambda) (5-\lambda).
\end{align*}
2) Entonces $p(\lambda)= (2-\lambda) ( 3-\lambda) (5-\lambda)$, y sus raíces se obtienen resolviendo $(2-\lambda) ( 3-\lambda) (5-\lambda) = 0$ y los eigenvalores son: $\lambda_1 = 2,\lambda_2=3,\lambda_3=5$.\\ \\
3) Se resuelve el sistema homogéneo $(\mathbf{A}-\lambda\mathbf{I})\mathbf{v}=\mathbf{0}$ para $\lambda_1=2$.
\begin{align*}
\begin{bmatrix}
2-2 & 0 & 0\\
0 & 3-2 & 0 \\
0 & 0 & 5-2
\end{bmatrix}
\begin{bmatrix}
x\\y\\z
\end{bmatrix}
&= 
\begin{bmatrix}
0 & 0 & 0\\
0 & 1 & 0 \\
0 & 0 & 3
\end{bmatrix}
\begin{bmatrix}
x\\y\\z
\end{bmatrix} =
\begin{bmatrix}
0\\y\\3z
\end{bmatrix}=
\begin{bmatrix}
0\\0\\0
\end{bmatrix}
\end{align*}
Esto es equivalente a la igualdad $y=z=0$. Haciendo $x=1$ tenemos que $y=0,z=0$ y el eigenvector correspondiente a $\lambda_1=2$ es $(1,0,0)$.\\\\
Ahora se resuelve el sistema homogéneo $(\mathbf{A}-\lambda\mathbf{I})\mathbf{v}=\mathbf{0}$ para $\lambda_2=3$.
\begin{align*}
\begin{bmatrix}
2-3 & 0 & 0\\
0 & 3-3 & 0 \\
0 & 0 & 5-3
\end{bmatrix}
\begin{bmatrix}
x\\y\\z
\end{bmatrix}
&= 
\begin{bmatrix}
-1 & 0 & 0\\
0 & 0 & 0 \\
0 & 0 & 2
\end{bmatrix}
\begin{bmatrix}
x\\y\\z
\end{bmatrix} =
\begin{bmatrix}
-x\\0\\2z
\end{bmatrix}=
\begin{bmatrix}
0\\0\\0
\end{bmatrix}
\end{align*}
Esto es equivalente a la igualdad $x=z=0$. Haciendo $y=1$ tenemos que $x=0,z=0$ y el eigenvector correspondiente a $\lambda_2=3$ es $(0,1,0)$.\\\\
Ahora se resuelve el sistema homogéneo $(\mathbf{A}-\lambda\mathbf{I})\mathbf{v}=\mathbf{0}$ para $\lambda_3=5$.
\begin{align*}
\begin{bmatrix}
2-5 & 0 & 0\\
0 & 3-5 & 0 \\
0 & 0 & 5-5
\end{bmatrix}
\begin{bmatrix}
x\\y\\z
\end{bmatrix}
&= 
\begin{bmatrix}
-3 & 0 & 0\\
0 & -2 & 0 \\
0 & 0 & 0
\end{bmatrix}
\begin{bmatrix}
x\\y\\z
\end{bmatrix} =
\begin{bmatrix}
-3x\\-2y\\0
\end{bmatrix}=
\begin{bmatrix}
0\\0\\0
\end{bmatrix}
\end{align*}
Esto es equivalente a la igualdad $x=y=0$. Haciendo $z=1$ tenemos que $x=0,y=0$ y el eigenvector correspondiente a $\lambda_3=5$ es $(0,0,1)$.\\\\
Por tanto, los valores característicos son $\lambda_1=2,\lambda_2=3,\lambda_3=5$ y sus vectores característicos son:
\begin{align*}\mathbf{v_1}=
\begin{bmatrix}
1\\0\\0
\end{bmatrix},\mathbf{v_2}=
\begin{bmatrix}
0\\1\\0
\end{bmatrix},\mathbf{v_3}=
\begin{bmatrix}
0\\0\\1
\end{bmatrix}
\end{align*} \\ \\
\end{sol}

%%%%%%%%%%%%%%%%%%%%%%%%%%%%%%%%%%%%%
%Do not alter anything below this line.
\end{document}