%%%%%%%%%%%%%%%%%%%%%%%%%%%%%%%%%%%%%%%%%%%%%%%%%%%%%%%%%%%%%%%%%%%%%%%%%%%%%%%%%%%%
%Do not alter this block of commands.  If you're proficient at LaTeX, you may include additional packages, create macros, etc. immediately below this block of commands, but make sure to NOT alter the header, margin, and comment settings here. 
\documentclass[12pt]{article}
 \usepackage[margin=1in]{geometry} 
\usepackage{amsmath,amsthm,amssymb,amsfonts, enumitem, fancyhdr, color, comment, graphicx, environ,bm}
\pagestyle{fancy}
\setlength{\headheight}{65pt}
\newenvironment{problem}[2][Problema]{\begin{trivlist}
\item[\hskip \labelsep {\bfseries #1}\hskip \labelsep {\bfseries #2.}]}{\end{trivlist}}
\newenvironment{sol}
    {\emph{Solución:}
    }
    {
    }
\specialcomment{com}{ \color{blue} \textbf{Comment:} }{\color{black}} %for instructor comments while grading
\NewEnviron{probscore}{\marginpar{ \color{blue} \tiny Problem Score: \BODY \color{black} }}
%%%%%%%%%%%%%%%%%%%%%%%%%%%%%%%%%%%%%%%%%%%%%%%%%%%%%%%%%%%%%%%%%%%%%%%%%%%%%%%%%





%%%%%%%%%%%%%%%%%%%%%%%%%%%%%%%%%%%%%%%%%%%%%
%Fill in the appropriate information below
\lhead{Héctor Alejandro Márquez González}  %replace with your name
\rhead{Métodos Estadísticos Multivariados \\ 1989936} %replace XYZ with the homework course number, semester (e.g. ``Spring 2019"), and assignment number.
%%%%%%%%%%%%%%%%%%%%%%%%%%%%%%%%%%%%%%%%%%%%%


%%%%%%%%%%%%%%%%%%%%%%%%%%%%%%%%%%%%%%
%Do not alter this block.
\begin{document}
%%%%%%%%%%%%%%%%%%%%%%%%%%%%%%%%%%%%%%

%\begin{problem}{1}
Se colectaron datos sobre la contaminacion del aire en cierta ciudad. Suponer normalidad
en los datos.\\
A) Determinar si el vector de media poblacional es $\mu_0=(8,74,5,2,10,9,3)$ con un nivel
de significancia del 5\% incluir el p-valor. \\  
B) Obtener IC simultaneos para las medias de cada componente, con un nivel de confianza  
global del 95\%. \\  
C) Obtener IC simultaneos para las diferencias de medias poblacionales (ignorando  
unidades) con un nivel de confianza global del 95\%. \\  
D) Obtener IC simultaneos para la media de cada componente, con un nivel de confianza  
global del 95\% usando el metodo de Bonferroni.  
\end{problem}
\begin{sol}
\begin{itemize}
\item A) Determinar si el vector de media poblacional es $\mu_0=(8,74,5,2,10,9,3)$ con un nivel
de significancia del 5\% incluir el p-valor.\\\\
Debido a que el vector de medias poblacionales y la matriz de covarianzas poblacionales son desconocidas, entonces usaremos la prueba de hipótesis para $\mu$ con parámetros desconocidos. Sea:
$H_0:\mu = \mu_0 $ v.s. $H_1:\mu \neq \mu_0$.
El estadístico de prueba es:
\begin{align*}
T^2=n(\bar{\mathbf{X}} - \mu_0)'\mathbf{S}^{-1}(\mathbf{\bar{X}}-\mu_0)
\end{align*}
Para esto, necesitamos calcular dos cosas, $\mathbf{\bar{X}}$, $\mathbf{S}$ y $\mathbf{S}^{-1}$, lo haremos con R:\\\\
\includegraphics[width=1\textwidth]{img/1.png}\\\\
Esto nos da:
\begin{align*}
\mathbf{\bar{X}} &= \begin{pmatrix}
  7.5000 \\
  73.8571 \\
  4.5476 \\
  2.1905 \\
  10.0476 \\
  9.4048 \\
  3.0952
\end{pmatrix}\\
\mathbf{S} &= \begin{pmatrix}
  2.5000 & -2.7805 & -0.3780 & -0.4634 & -0.5854 & -2.2317 &  0.1707 \\
 -2.7805 & 300.5157 &  3.9094 & -1.3868 &  6.7631 & 30.7909 &  0.6237 \\
 -0.3780 &   3.9094 &  1.5221 &  0.6736 &  2.3148 &  2.8217 &  0.1417 \\
 -0.4634 &  -1.3868 &  0.6736 &  1.1823 &  1.0883 & -0.8107 &  0.1765 \\
 -0.5854 &   6.7631 &  2.3148 &  1.0883 & 11.3635 &  3.1266 &  1.0441 \\
 -2.2317 &  30.7909 &  2.8217 & -0.8107 &  3.1266 & 30.9785 &  0.5947 \\
  0.1707 &   0.6237 &  0.1417 &  0.1765 &  1.0441 &  0.5947 &  0.4785
\end{pmatrix} \\
\mathbf{S^{-1}} &= \begin{pmatrix}
  0.5651 &  0.0023 & -0.2490 &  0.4382 &  0.0639 &  0.0762 & -0.5267 \\
  0.0023 &  0.0038 & -0.0067 &  0.0082 & -0.0006 & -0.0026 & -0.0021 \\
 -0.2490 & -0.0067 &  1.8812 & -1.1633 & -0.3075 & -0.1991 &  0.8880 \\
  0.4382 &  0.0082 & -1.1633 &  1.8575 &  0.1171 &  0.1852 & -0.9935 \\
  0.0639 & -0.0006 & -0.3075 &  0.1171 &  0.1707 &  0.0263 & -0.3793 \\
  0.0762 & -0.0026 & -0.1991 &  0.1852 &  0.0263 &  0.0640 & -0.1701 \\
 -0.5267 & -0.0021 &  0.8880 & -0.9935 & -0.3793 & -0.1701 &  3.4233
\end{pmatrix}
\end{align*}
Ahora calcularemos esta operación en R, $T^2=n(\bar{\mathbf{X}} - \mu_0)'\mathbf{S}^{-1}(\mathbf{\bar{X}}-\mu_0)$: \\\\
\includegraphics[width=1\textwidth]{img/2.png}\\\\
Por tanto $T^2 = 27.0684$. Nuestra regla de desición será que se rechaza $H_0$ con nivel de significancia $\alpha$ si:
\begin{align*}
T^2 > \frac{(n-1)p}{n-p}F_{\alpha,p,n-p}
\end{align*}
\includegraphics[width=1\textwidth]{img/3.png}\\\\
Debido a que $T^2>18.7389$ entonces rechazamos $H_0$ y aceptamos la hipótesis alternativa. Esto quiere decir que el vector de medias poblacionales no es igual a $\mu_0$\\
Para calcular el p-valor tenemos que:
\begin{align*}
P(F>\frac{(n-p)T^2}{(n-1)p})
\end{align*}
A continuación calcularemos esta probabilidad en R:\\\\
\includegraphics[width=1\textwidth]{img/4.png}\\\\
Por tanto p-valor$=0.0084$. \pagebreak
\item B) Obtener IC simultaneos para las medias de cada componente, con un nivel de confianza  
global del 95\%. \\  
Para esto usaremos la fórmula para los intervalos de confianza simultáneos, esto es:
\begin{align*}
\bar{x}_i-\sqrt{\frac{p(n-1)}{n(n-p)}F_{\alpha,p,n-p}s_{ii}}<\mu_i<\bar{x_i}+\sqrt{\frac{p(n-1)}{n(n-p)}F_{\alpha,p,n-p}s_{ii}}
\end{align*}
Calcularemos esto en R para todas las medias:\\\\
\includegraphics[width=1\textwidth]{img/5.png}\\\\
Esto es:
\begin{align*}
6.4439 < \mu_1 &< 8.5561 \\
62.2779 < \mu_2 &< 85.4364 \\
3.7235 < \mu_3 &< 5.3717 \\
1.4642 < \mu_4 &< 2.9168 \\
7.7959 < \mu_5 &< 12.2993 \\
5.6870 < \mu_6 &< 13.1225 \\
2.6332 < \mu_7 &< 3.5573 \\
\end{align*}
\pagebreak
\item C) Obtener IC simultaneos para las diferencias de medias poblacionales (ignorando  
unidades) con un nivel de confianza global del 95\%. \\
Para esto usaremos la fórmula de IC simultáneos para la diferencia de medias:
\begin{align*}
(\bar{x_i}-\bar{x_j}) \pm \sqrt{\frac{p(n-1)}{n(n-p)}F_{\alpha,p,n-p}(s_{ii}-2s_{ij}+s_{jj})}
\end{align*}
Observamos que al tener 7 variables entonces habrá $\binom{7}{2} = 21$ intervalos. Lo calcularemos en R:\\\\
\includegraphics[width=1\textwidth]{img/6.png}\\\\
Nos dan los siguientes resultados:
\begin{align*}
-78.0907 < \mu_{1} - \mu_{2} &< -54.6236 \\
1.4923 < \mu_{1} - \mu_{3} &< 4.4125 \\
3.8755 < \mu_{1} - \mu_{4} &< 6.7436 \\
-5.1376 < \mu_{1} - \mu_{5} &< 0.0423 \\
-6.0192 < \mu_{1} - \mu_{6} &< 2.2096 \\
3.3201 < \mu_{1} - \mu_{7} &< 5.4895 \\
57.8522 < \mu_{2} - \mu_{3} &< 80.7668 \\
60.0114 < \mu_{2} - \mu_{4} &< 83.3219 \\
52.2720 < \mu_{2} - \mu_{5} &< 75.3471 \\
53.4785 < \mu_{2} - \mu_{6} &< 75.4262 \\
59.1975 < \mu_{2} - \mu_{7} &< 82.3264 \\
1.5790 < \mu_{3} - \mu_{4} &< 3.1353 \\
-7.4193 < \mu_{3} - \mu_{5} &< -3.5807 \\
-8.3187 < \mu_{3} - \mu_{6} &< -1.3955 \\
0.5771 < \mu_{3} - \mu_{7} &< 2.3277 \\
-10.0081 < \mu_{4} - \mu_{5} &< -5.7062 \\
-11.0966 < \mu_{4} - \mu_{6} &< -3.3320 \\
-1.6686 < \mu_{4} - \mu_{7} &< -0.1409 \\
-3.3698 < \mu_{5} - \mu_{6} &< 4.6555 \\
4.8663 < \mu_{5} - \mu_{7} &< 9.0385 \\
2.6347 < \mu_{6} - \mu_{7} &< 9.9844 \\
\end{align*}
\pagebreak
\item D) Obtener IC simultaneos para la media de cada componente, con un nivel de confianza  
global del 95\% usando el metodo de Bonferroni. \\
El método de Bonferroni nos dice que ternemos el siguiente IC:
\begin{align*}
\bar{x_i}-t_{\alpha / 2p,n-1}\sqrt{\frac{s_{ii}}{n}}<\mu_i<\bar{x_i}+t_{\alpha / 2p,n-1}\sqrt{\frac{s_{ii}}{n}}
\end{align*}
\includegraphics[width=1\textwidth]{img/7.png}\\\\
Esto es:
\begin{align*}
6.8091 < \mu_1 &< 8.1909 \\
66.2823 < \mu_2 &< 81.4320 \\
4.0085 < \mu_3 &< 5.0867 \\
1.7153 < \mu_4 &< 2.6656 \\
8.5746 < \mu_5 &< 11.5206 \\
6.9727 < \mu_6 &< 11.8368 \\
2.7930 < \mu_7 &< 3.3975 \\
\end{align*}
\end{itemize}
\end{sol}
\pagebreak
%\begin{problem}{2}
Se comparan dos maquinarias pesadas de perforación profunda. Se miden las siguientes variables:X1 = peso de arrastre, X2 = peso de rotación, X3 = ángulo, X4 = velocidad de la barrena,
X5 = peso del lodo, X6 = peso de levante y X7 = Torque de fondo.\\
A) Determinar si los vectores de medias poblacionales son iguales en las maquinarias (alfa = 5\%).\\
B) Obtener intervalos de confianza (95\%) de diferencia de medias e interpretar resultados.\\
C) Menciona los supuestos que hiciste.
\end{problem}
\begin{sol}
A) Determinar si los vectores de medias poblacionales son iguales en las maquinarias (alfa = 5\%).\\
Debido a que el enunciado no nos dice que las muestras sigan una distribución normal y también se cumple que $n_1-p,n_2-p>40$ entonces usaremos la prueba de hipótesis para dos medias poblacionales grandes.
Sea $H_0:\mu_1-\mu_2 = \delta_0$ vs $H_1: \mu_1 - \mu_2
\neq \delta_0$ donde:
\begin{align*}
\delta_0 = \begin{pmatrix}
0\\0\\0\\0
\end{pmatrix}
\end{align*}
Usaremos el estadístico de prueba
\begin{align*}
T^2 = (\bm{\bar{X_1}-\bar{X_2}-\delta_0})' \left[ \frac{1}{n_1}\bm{S_1}+\frac{1}{n_2}\bm{S_2} \right] ^{-1} (\bm{\bar{X_1}-\bar{X_2}-\delta_0})
\end{align*}
Y rechazaremos $H_0$ con nivel de significancia $\alpha$ si $T^2 > \chi_{\alpha, p}^2$. Debido a que ya contamos con todos los datos necesarios, solo debemos de calcular $T^2$ y $ \chi_{\alpha, p}^2$, lo haremos en R:\\\\
\includegraphics[width=1\textwidth]{img/5.png}\\\\
Debido a que $T^2=292.6722 > \chi_{\alpha, p}^2=14.06714$ entonces rechazamos $H_0$, esto es que hay evidencia de que las medias no son iguales.

\pagebreak
B) Obtener intervalos de confianza (95\%) de diferencia de medias e interpretar resultados.\\
Como es muestra grande calcularemos los intervalos con:
\begin{align*}
(\bar{X_{1i}}-\bar{X_{2i}}) \pm \sqrt{\chi_{\alpha, p}^2 \left[\frac{1}{n_1}s_{ii(1)}+\frac{1}{n_2}s_{ii(2)} \right] }
\end{align*}
Lo calcularemos en R:\\
\includegraphics[width=1\textwidth]{img/6.png}\\\\
Esto es:
\begin{align*}
-17.2501765 &< \mu_1 - \mu_2 < 13.8501765 \\
-20.7756225 &< \mu_1 - \mu_2 < 12.7356225 \\
-0.6489871  &< \mu_1 - \mu_2 < 2.1589871 \\
-1.6828236  &< \mu_1 - \mu_2 < 0.9248236 \\
-5.9186822  &< \mu_1 - \mu_2 < 6.6586822 \\
-6.0141268  &< \mu_1 - \mu_2 < 4.0221268 \\
-1.4152759  &< \mu_1 - \mu_2 < 0.8912759 \\
\end{align*}
Observamos que todos los intervalos van de negativo a positivo, esto quiere decir que todos los intervalos de confianza contienen al cero, lo cual puede sonar contradictorio debido a que en la prueba de hipotesis anterior nos dio que hay evidencia de que las medias no son iguales.

\pagebreak

C) Mencione los supuestos que hiciste.\\
En este caso debido a que es una muestra grande, no es necesario que las dos muestras sigan una distribución normal p-variada debido a que la media muestral se aproximará a una distribución normal. Lo que nosotros tuvimos que asumir es que las observaiones son independientes unas de otras.
\end{sol}
\pagebreak
\begin{problem}{3}
En una planta de lácteos se comparar las siguientes variables en la elaboración de quesos y se desean
comparar tres turnos.
X1 = temperatura de cuajado (°C), X2 = temperatura de cocimiento (°C),
X3 = tiempo de cocimiento (min), X4 = tiempo de fundido (min),
X5 = tiempo de transferencia, X6 = humedad (\%).\\
A) Determinar si los vectores de medias poblacionales de los tres turnos son iguales (incluir el
estadístico de Bartlett), alfa = 5\% e interpretar resultado.\\
B) Obtener IC simultáneos para diferencia de medias (95\%).\\
C) Aplicar la prueba M de Box e interpretar resultado.
\end{problem}
\begin{sol}
A) Determinar si los vectores de medias poblacionales de los tres turnos son iguales (incluir el
estadístico de Bartlett), alfa = 5\% e interpretar resultado.\\
Para esto haremos el MANOVA en R, con los tests de Pillai, Wilks, Hotelling-Lawley y Roy, luego calcularemos el estadístico de Bartlett. Esto se hace para hacer nuestra prueba de hipótesis donde:
$H_0:\bm{\mu_1}=\cdots = \bm{\mu_k}$ vs $H_1:$ Al menos hay $\bm{\mu_p} \neq \bm{\mu_q}$, para esto compararemos el estadístico de Bartlett, donde se rechazara $H_0$ con nivel de significancia $\alpha$ si:
\begin{align*}
-[N-1-(p+k)/2]\ln{(\Lambda^*)}>\chi_{\alpha,p(k-1)}^2
\end{align*}
Lo haremos en R:\\\\
\includegraphics[width=1\textwidth]{img/7.png}\\\\
Observamos que $12.5404 < \chi_{\alpha,p(k-1)}^2$ por tanto no rechazamos $H_0$ por lo que no hay suficiente evidencia para decir que las medias son distintas. Por tanto inferimos que los vectores de medias son iguales en los tres turnos. 

\pagebreak

B) Obtener IC simultáneos para diferencia de medias (95\%).\\
Para obtener esto primero necesitamos calcular $\bm{W}=(n_1-1)\bm{S_1}+\cdots+(n_k-1)\bm{S_k}$, $\frac{\bm{W}}{N-k}$ pues con esta matriz estimamos $\bm{\Sigma}$ y luego $\hat{V}(\bar{X_{qi}}-\bar{X_{ri}})=(\frac{1}{n_q}+\frac{1}{n_r})[\frac{w_{ii}}{N-k}]$, con estos datos nuestros intervalos de confianza serán:
\begin{align*}
(\bar{X_{qi}}-\bar{X_{ri}}) \pm t_{\alpha/(pk(k-1)),N-k}\sqrt{(\frac{1}{n_q}+\frac{1}{n_r})[\frac{w_{ii}}{N-k}]}
\end{align*}
Lo calcularemos en Python:\\
\includegraphics[width=1\textwidth]{img/8.png}\\
\includegraphics[width=1\textwidth]{img/9.png}\\
\includegraphics[width=1\textwidth]{img/10.png}\\
Esto es, para las muestras 1 y 2:
\begin{align*}
-0.6810228 &< \mu_{1} - \mu_{2} < 0.9543562 \\
-1.5234393 &< \mu_{2} - \mu_{2} < 0.5767726 \\
-3.7931644 &< \mu_{3} - \mu_{3} < 4.3931644 \\
-22.0522471 &< \mu_{4} - \mu_{4} < 15.3189137 \\
-19.8476293 &< \mu_{5} - \mu_{5} < 34.4476293 \\
-1.2104656 &< \mu_{6} - \mu_{6} < 1.7464656 \\
\end{align*}
Para las muestras 2 y 3:
\begin{align*}
-0.7648785 &< \mu_{2} - \mu_{3} < 0.7648785 \\
-0.6622842 &< \mu_{2} - \mu_{3} < 1.3022842 \\
-1.6288047 &< \mu_{2} - \mu_{3} < 6.0288047 \\
-15.0620933 &< \mu_{2} - \mu_{3} < 19.8954266 \\
-30.9942819 &< \mu_{2} - \mu_{3} < 19.7942819 \\
-2.0980613 &< \mu_{2} - \mu_{3} < 0.6678946 \\
\end{align*}
Para las muestras 1 y 3:
\begin{align*}
-0.6282118 &< \mu_{1} - \mu_{3} < 0.9015452 \\
-1.1356175 &< \mu_{2} - \mu_{3} < 0.8289508 \\
-1.3288047 &< \mu_{3} - \mu_{3} < 6.3288047 \\
-18.4287600 &< \mu_{4} - \mu_{4} < 16.5287600 \\
-23.6942819 &< \mu_{5} - \mu_{5} < 27.0942819 \\
-1.8300613 &< \mu_{6} - \mu_{6} < 0.9358946 \\
\end{align*}

\pagebreak

C) Aplicar la prueba M de Box e interpretar resultado.\\
Para la prueba M de box tendremos que calcular lo siguiente:
\begin{align*}
\Lambda &= \prod_{i}(\frac{|\bm{S_i}|}{|\bm{S_{pond}}|})^{\frac{n_i-1}{2}}  \\
M &= -2\ln{\Lambda}\\
f_1 &= (k-1)p(p+1)/2\\
\rho &= 1- \frac{2p^2+3p-1}{6(p+1)(k-1)}(\sum_i\frac{1}{(n_i-1)}-\frac{1}{N-k})\\
\tau &= \frac{(p-1)(p+2)}{6(k-1)}(\sum_i\frac{1}{(n_i-1)^2}-\frac{1}{(N-k)^2})\\
f_2 &= \frac{f_1+2}{|\tau-(1-p)^2|}\\
\rho &= \frac{\pi-f_1/f_2}{f_1}
\end{align*}
Estos valores nos ayudarán con nuestra prueba de hipótesis donde: $H_0:\bm{\Sigma_1=\Sigma_2=\cdots = \Sigma_k}$ vs $H_1:$ Al menos hay dos diferentes $\bm{\Sigma_u} \neq \bm{\Sigma_v}$. Nuestro estadístico de prueba será $\gamma* M$ en donde se rechazará $H_0$ con nivel de significancia $\alpha$ si $\gamma*M>F_{\alpha,f_1,f_2}$. Haremos nuestros cálculos en R:\\
\includegraphics[width=1\textwidth]{img/11.png}\\
\includegraphics[width=1\textwidth]{img/12.png}\\
Debido a que $\gamma* M = 1.8394 > F_{\alpha,f_1,f_2}$ entonces rechazamos $H_0$ y decimos que no hay suficiente evidencia para decir que las matrices de covarianza son iguales.
\end{sol}

%%%%%%%%%%%%%%%%%%%%%%%%%%%%%%%%%%%%%
%Do not alter anything below this line.
\end{document}