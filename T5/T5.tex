%%%%%%%%%%%%%%%%%%%%%%%%%%%%%%%%%%%%%%%%%%%%%%%%%%%%%%%%%%%%%%%%%%%%%%%%%%%%%%%%%%%%
%Do not alter this block of commands.  If you're proficient at LaTeX, you may include additional packages, create macros, etc. immediately below this block of commands, but make sure to NOT alter the header, margin, and comment settings here. 
\documentclass[12pt]{article}
 \usepackage[margin=1in]{geometry} 
\usepackage{amsmath,amsthm,amssymb,amsfonts, enumitem, fancyhdr, color, comment, graphicx, environ,bm}
\pagestyle{fancy}
\setlength{\headheight}{65pt}
\newenvironment{problem}[2][Problema]{\begin{trivlist}
\item[\hskip \labelsep {\bfseries #1}\hskip \labelsep {\bfseries #2.}]}{\end{trivlist}}
\newenvironment{sol}
    {\emph{Solución:}
    }
    {
    }
\specialcomment{com}{ \color{blue} \textbf{Comment:} }{\color{black}} %for instructor comments while grading
\NewEnviron{probscore}{\marginpar{ \color{blue} \tiny Problem Score: \BODY \color{black} }}
%%%%%%%%%%%%%%%%%%%%%%%%%%%%%%%%%%%%%%%%%%%%%%%%%%%%%%%%%%%%%%%%%%%%%%%%%%%%%%%%%





%%%%%%%%%%%%%%%%%%%%%%%%%%%%%%%%%%%%%%%%%%%%%
%Fill in the appropriate information below
\lhead{Héctor Alejandro Márquez González}  %replace with your name
\rhead{Métodos Estadísticos Multivariados \\ 1989936} %replace XYZ with the homework course number, semester (e.g. ``Spring 2019"), and assignment number.
%%%%%%%%%%%%%%%%%%%%%%%%%%%%%%%%%%%%%%%%%%%%%


%%%%%%%%%%%%%%%%%%%%%%%%%%%%%%%%%%%%%%
%Do not alter this block.
\begin{document}
%%%%%%%%%%%%%%%%%%%%%%%%%%%%%%%%%%%%%%

%\begin{problem}{1}
Se tiene la tasa de retorno semanal de 5 acciones bursátiles del NYSE.\\
A) Tratar de reducir la información a menos de 5 dimensiones usando componenetes principales.\\
B) Obtener los valores de los componentes pricipales obtenidos en el a) para:
\begin{table}[h]
    \centering
    \begin{tabular}{lccccc}
        \toprule
         Allied Chemical & Du Pont & Union Carbide & Exxon & Texaco \\
        \midrule
        -0.030717 & 0.020202 & -0.04086 & -0.03905 & -0.05051 \\
        -0.003521 & 0.118812 & 0.089686 & 0.06007 & 0.021276 \\
        0.060071  & 0.079646 & 0.028807 & 0.036666 & 0.026041 \\
        \bottomrule
    \end{tabular}
    \label{tab:datos}
\end{table}
\end{problem}
\begin{sol}A)
Para reducir la dimensionalidad, aplicamos el Análisis de Componentes Principales (PCA) con datos estandarizados:
\begin{verbatim}
pca_matriz_correlacion <- function(datos){
  pca_corr <- prcomp(datos, center = TRUE, scale = TRUE)

  plot(pca_corr, type = "l", main = "Grafico de codo de x1, x2, ..., x6")

  print(pca_corr)

  print(summary(pca_corr))

  biplot(pca_corr, scale = 0)

  cat("eigenvalues: ", pca_corr$sdev^2 , "\n" )

  return(pca_corr)

}
\end{verbatim}
Nos resulta:\\\\
\includegraphics[width=1\textwidth]{img/1.png}\\
Esto nos quiere decir que tendremos:
\begin{align*}
\lambda_1 = 2.8564, \lambda_2 = 0.80911, \lambda_3 = 0.5400, \lambda_4 = 0.4513 , \lambda_5 = 0.343
\end{align*}
También de los datos podemos concluir que los primeros dos componentes principales explican más del 73\% de la variabilidad de los datos, por tanto, reducimos las cinco variables a 2:
\[
PC_1 = 0.4635 \cdot \text{Allied Chemical} + 0.4571 \cdot \text{Du Pont} + 0.4700 \cdot \text{Union Carbide} + 0.4217 \cdot \text{Exxon} + 0.4213 \cdot \text{Texaco}
\]
\[
PC_2 = 0.2408 \cdot \text{Allied Chemical} + 0.5091 \cdot \text{Du Pont} + 0.2606 \cdot \text{Union Carbide} - 0.5253 \cdot \text{Exxon} - 0.5822 \cdot \text{Texaco}
\]
Nuestras cinco variables originales se combinaron linealmente para formar las dos primeras componentes principales PC1 y PC2.

\pagebreak

B) Usaremos el siguiente código: 
\begin{verbatim}
calcular_datos_dada_pca_corr <- function(pca_corr,datos_valores, medias, desvest){

  for (i in 1:nrow(datos_valores)) {

    # se tienen que escalar los valores debido a que para el pca
    # se uso la correlacion en vez de la covarianza.
    datos_valores_escalados <- scale(datos_valores[i,], center = medias, scale = desvest)

    cat("fila ",i,"\n")
    cat("y1=",sum(datos_valores_escalados*pca_corr$rotation[, "PC1"]),"\n")
    cat("y2=",sum(datos_valores_escalados*pca_corr$rotation[, "PC2"]),"\n\n")
  }
}
\end{verbatim}
Nos resulta:\\\\
\includegraphics[width=1\textwidth]{img/2.png}\\
\begin{table}[ht]
    \centering
    \begin{tabular}{|c|c|c|}
        \hline
        \textbf{Fila} & $y_1$ & $y_2$ \\
        \hline
        1 & -2.273133 & 1.687177 \\
        2 & 3.453575 & 0.7881083 \\
        3 & 2.672294 & 0.5299272 \\
        \hline
    \end{tabular}
\end{table}


\end{sol}
\pagebreak
%\begin{problem}{2}
Se tiene la tasa de retorno semanal de 5 acciones bursátiles del NYSE. Tratar de identificar factores no observables que estén relacionados con estas variables.
\end{problem}
\begin{sol}
Primero leeremos los datos y encontraremos la matriz de correlación:
\begin{verbatim}
file_path <- "C:/Users/hamga/Downloads/datos_tarea 7.xlsx"

datos <- read_excel(file_path, sheet = 1)
datos <- datos[,-1]
head(datos)

cor_matrix <- cor(datos)
n_obs <- nrow(datos)

corrplot(cor_matrix, method = "circle", addCoef.col = "black",
         order = "original", type = "upper")
\end{verbatim}
\includegraphics[width=1\textwidth]{img/9.png}\\
Para poder identificar los factores primero tenemos que hacer la prueba de esfericidad de Bartlett para evaluar si la matriz de correlaciones es significativamente diferente de $\sigma^2 \bm{I}$, lo que indicaría que hay correlaciones significativas entre las variables y que el análisis factorial es apropiado:
\begin{verbatim}
> cortest.bartlett(cor_matrix, n=n_obs)
\end{verbatim}
\includegraphics[width=1\textwidth]{img/10.png}\\
Observamos que el p value es muy pequeño, por lo que rechazamos nuestra hipótesis nula, por tanto, la matriz de correlaciones es significativamente diferente de $\sigma^2 \bm{I}$. Se infiere que si hay correlacion entre las variables, ahora determinaremos el número de factores:
\begin{verbatim}
fa.parallel(cor_matrix, fm = "pa", n.obs = n_obs, ylabel = "Eigenvalues")
\end{verbatim}
\includegraphics[width=1\textwidth]{img/11.png}\\
Observamos que es recomendable usar 2 factores. Ahora haremos el análisis de factores utilizando componentes principales:
\begin{verbatim}
acp <- principal(cor_matrix, nfactors = 2, rotate = "none")
print(acp)
\end{verbatim}
\includegraphics[width=1\textwidth]{img/12.png}\\
Podemos observar las cargas de los dos componentes PC1 y PC2. Ahora lo haremos de nuevo pero con una rotación varimax para obtener resultados mas precisos:
\begin{verbatim}
acp <- principal(cor_matrix, nfactors = 2, rotate = "varimax", n.obs = n_obs)

(acp$r.scores)

print(acp)
\end{verbatim}
\includegraphics[width=1\textwidth]{img/13.png}\\
Esto nos dice que el primer componente tendrá mayor carga para las variables 1,2 y 3, mientras que el segundo tendra para la variable 4 y 5. Veremos la gráfica del análisis de componentes principales con rotación varimax:\\
\includegraphics[width=1\textwidth]{img/14.png}\\
Observamos que las variables 5 y 4 pertenecen al segundo componente principal y las 1,2 y 3 pertenecen al primero.
Ahora haremos el análisis de factores:
\begin{verbatim}
mlf <- fa(cor_matrix, nfactors = 2, fm="ml", rotate = "varimax", n.obs = n_obs,
          scores = "regression")
print(mlf)
\end{verbatim}
\includegraphics[width=1\textwidth]{img/15.png}\\
Los resultados nos dicen que el primer factor explica el 33\% de la varianza total, mientras que el segundo explica el 26\%, lo que hace que ambos factores expliquen el 60 \% de la varianza total.\\ En cuanto a los factores, el primero esta asociado fuertemente a la primera, segunda y tercer variable mientras que el segundo factor esta asociado fuertemente a la cuarta y quinta variable.\\ 
Podemos ver en la siguiente gráfica cuando es de color morado tiene mayor carga para el factor dos y si es color blanco tiene mayor carga para el factor uno.\\
\includegraphics[width=1\textwidth]{img/16.png}\\
La comunalidad (h2) nos dice que tanto se explica de la variabilidad de esa variable, observamos que para las variables Texaco y Du point se explica altamente la variabilidad mientras que para las otras no.\\
Al investigar a que se dedica cada empresa, tenemos que se dedican a la venta de:
\begin{itemize}
\item Allied Chemical: Productos químicos
\item Du pont: Ciencia, productos químicos y biotecnología
\item Union Carbide: Productos químicos
\item Exxon: Energía
\item Texaco: Energía
\end{itemize}
En conclusión, los resultados sugieren que las tasas de retorno de estas acciones están influenciadas por factores sectoriales, con un grupo relacionado con la industria química (factor 1) y otro con el sector energético (factor 2). Esta información puede ser útil para identificar patrones de comportamiento en el mercado y mejorar estrategias de diversificación.
\end{sol}\pagebreak
\begin{problem}{3}
En una planta de lácteos se comparar las siguientes variables en la elaboración de quesos y se desean
comparar tres turnos.
X1 = temperatura de cuajado (°C), X2 = temperatura de cocimiento (°C),
X3 = tiempo de cocimiento (min), X4 = tiempo de fundido (min),
X5 = tiempo de transferencia, X6 = humedad (\%).\\
A) Determinar si los vectores de medias poblacionales de los tres turnos son iguales (incluir el
estadístico de Bartlett), alfa = 5\% e interpretar resultado.\\
B) Obtener IC simultáneos para diferencia de medias (95\%).\\
C) Aplicar la prueba M de Box e interpretar resultado.
\end{problem}
\begin{sol}

\end{sol}

%%%%%%%%%%%%%%%%%%%%%%%%%%%%%%%%%%%%%
%Do not alter anything below this line.
\end{document}