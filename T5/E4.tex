\begin{problem}{4}
Se comparan los puntajes de un examen de admisión a posgrado de estudiantes de tres universidades.\\
A) Determinar si los vectores de medias poblacionales de los tres turnos son iguales (incluir el
estadístico de Bartlett), alfa = 5\% e interpretar resultado.\\
B) Obtener IC simultáneos para diferencia de medias (95\%).\\
C) Aplicar la prueba M de Box e interpretar resultado.
\end{problem}
\begin{sol}
A) Determinar si los vectores de medias poblacionales de los tres turnos son iguales (incluir el
estadístico de Bartlett), alfa = 5\% e interpretar resultado.\\
Para esto haremos el MANOVA en R, con los tests de Pillai, Wilks, Hotelling-Lawley y Roy, luego calcularemos el estadístico de Bartlett. Esto se hace para hacer nuestra prueba de hipótesis donde:
$H_0:\bm{\mu_1}=\cdots = \bm{\mu_k}$ vs $H_1:$ Al menos hay $\bm{\mu_p} \neq \bm{\mu_q}$, para esto compararemos el estadístico de Bartlett, donde se rechazara $H_0$ con nivel de significancia $\alpha$ si:
\begin{align*}
-[N-1-(p+k)/2]\ln{(\Lambda^*)}>\chi_{\alpha,p(k-1)}^2
\end{align*}
Lo haremos en R:\\
\includegraphics[width=1\textwidth]{img/13.png}\\\\
\includegraphics[width=1\textwidth]{img/14.png}\\\\
Observamos que $20.9785 < \chi_{\alpha,p(k-1)}^2$ por tanto no rechazamos $H_0$ por lo que no hay suficiente evidencia para decir que las medias son distintas. Por tanto inferimos que los vectores de medias son iguales en las tres universidades.

\pagebreak
B) Obtener IC simultáneos para diferencia de medias (95\%).\\
Para obtener esto primero necesitamos calcular $\bm{W}=(n_1-1)\bm{S_1}+\cdots+(n_k-1)\bm{S_k}$, $\frac{\bm{W}}{N-k}$ pues con esta matriz estimamos $\bm{\Sigma}$ y luego $\hat{V}(\bar{X_{qi}}-\bar{X_{ri}})=(\frac{1}{n_q}+\frac{1}{n_r})[\frac{w_{ii}}{N-k}]$, con estos datos nuestros intervalos de confianza serán:
\begin{align*}
(\bar{X_{qi}}-\bar{X_{ri}}) \pm t_{\alpha/(pk(k-1)),N-k}\sqrt{(\frac{1}{n_q}+\frac{1}{n_r})[\frac{w_{ii}}{N-k}]}
\end{align*}
Lo calcularemos en R:\\
\includegraphics[width=1\textwidth]{img/15.png}\\
\includegraphics[width=1\textwidth]{img/16.png}\\
\includegraphics[width=1\textwidth]{img/17.png}\\
\includegraphics[width=1\textwidth]{img/18.png}\\
Debido a que nuestra muestra 1 es la universidad A, la muestra 2 es la universidad B y la muestra 3 es la universidad C. Tenemos los siguientes resultados:
Para las universidades A vs B
\begin{align*}
-64.949171 &< \mu_{A1} - \mu_{B1} < 19.9026595 \\
-10.034469 &< \mu_{A2} - \mu_{B2} < -0.5146263 \\
-5.156442 &< \mu_{A3} - \mu_{B3} < 3.9749175 \\
-7.956788 &< \mu_{A4} - \mu_{B4} < 1.9309480 \\
-4.631016 &< \mu_{A5} - \mu_{B5} < 4.1090523 \\
-4.946035 &< \mu_{A6} - \mu_{B6} < 3.7541743 \\
\end{align*}

Para las universidades B vs C
\begin{align*}
-24.157744 &< \mu_{B1} - \mu_{C1} < 56.386074 \\
-1.977294 &< \mu_{B2} - \mu_{C2} < 7.059217 \\
-4.763315 &< \mu_{B3} - \mu_{C3} < 3.904436 \\
-3.674892 &< \mu_{B4} - \mu_{C4} < 5.710833 \\
-4.537434 &< \mu_{B5} - \mu_{C5} < 3.758893 \\
-1.112861 &< \mu_{B6} - \mu_{C6} < 7.145630 \\
\end{align*}

Para las universidades A vs C
\begin{align*}
-48.614737 &< \mu_{A1} - \mu_{C1} < 35.796556 \\
-7.468795 &< \mu_{A2} - \mu_{C2} < 2.001623 \\
-5.562178 &< \mu_{A3} - \mu_{C3} < 3.521773 \\
-6.913150 &< \mu_{A4} - \mu_{C4} < 2.923251 \\
-4.997598 &< \mu_{A5} - \mu_{C5} < 3.697093 \\
-1.907065 &< \mu_{A6} - \mu_{C6} < 6.747974 \\
\end{align*}

\pagebreak
C) Aplicar la prueba M de Box e interpretar resultado.
Para la prueba M de box tendremos que calcular lo siguiente:
\begin{align*}
\Lambda &= \prod_{i}(\frac{|\bm{S_i}|}{|\bm{S_{pond}}|})^{\frac{n_i-1}{2}}  \\
M &= -2\ln{\Lambda}\\
f_1 &= (k-1)p(p+1)/2\\
\rho &= 1- \frac{2p^2+3p-1}{6(p+1)(k-1)}(\sum_i\frac{1}{(n_i-1)}-\frac{1}{N-k})\\
\tau &= \frac{(p-1)(p+2)}{6(k-1)}(\sum_i\frac{1}{(n_i-1)^2}-\frac{1}{(N-k)^2})\\
f_2 &= \frac{f_1+2}{|\tau-(1-p)^2|}\\
\rho &= \frac{\pi-f_1/f_2}{f_1}
\end{align*}
Estos valores nos ayudarán con nuestra prueba de hipótesis donde: $H_0:\bm{\Sigma_1=\Sigma_2=\cdots = \Sigma_k}$ vs $H_1:$ Al menos hay dos diferentes $\bm{\Sigma_u} \neq \bm{\Sigma_v}$. Nuestro estadístico de prueba será $\gamma* M$ en donde se rechazará $H_0$ con nivel de significancia $\alpha$ si $\gamma*M>F_{\alpha,f_1,f_2}$. Haremos nuestros cálculos en R:\\
\includegraphics[width=1\textwidth]{img/19.png}\\
\includegraphics[width=1\textwidth]{img/20.png}\\
Debido a que $\gamma* M = 1.3263 < F_{\alpha,f_1,f_2}$ entonces no rechazamos $H_0$ y decimos que hay suficiente evidencia para decir que las matrices de covarianza son iguales. Esto que hicimos no era necesario debido a que usamos una teoría de muestra grande y la teoría de muestra grande implica que, con suficientes datos, los estimadores serán consistentes y los efectos de las diferencias de varianza/covarianza se atenúan. 
\end{sol}