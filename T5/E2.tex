\begin{problem}{2}
Se comparan dos maquinarias pesadas de perforación profunda. Se miden las siguientes variables:X1 = peso de arrastre, X2 = peso de rotación, X3 = ángulo, X4 = velocidad de la barrena,
X5 = peso del lodo, X6 = peso de levante y X7 = Torque de fondo.\\
A) Determinar si los vectores de medias poblacionales son iguales en las maquinarias (alfa = 5\%).\\
B) Obtener intervalos de confianza (95\%) de diferencia de medias e interpretar resultados.\\
C) Menciona los supuestos que hiciste.
\end{problem}
\begin{sol}
A) Determinar si los vectores de medias poblacionales son iguales en las maquinarias (alfa = 5\%).\\
Debido a que el enunciado no nos dice que las muestras sigan una distribución normal y también se cumple que $n_1-p,n_2-p>40$ entonces usaremos la prueba de hipótesis para dos medias poblacionales grandes.
Sea $H_0:\mu_1-\mu_2 = \delta_0$ vs $H_1: \mu_1 - \mu_2
\neq \delta_0$ donde:
\begin{align*}
\delta_0 = \begin{pmatrix}
0\\0\\0\\0
\end{pmatrix}
\end{align*}
Usaremos el estadístico de prueba
\begin{align*}
T^2 = (\bm{\bar{X_1}-\bar{X_2}-\delta_0})' \left[ \frac{1}{n_1}\bm{S_1}+\frac{1}{n_2}\bm{S_2} \right] ^{-1} (\bm{\bar{X_1}-\bar{X_2}-\delta_0})
\end{align*}
Y rechazaremos $H_0$ con nivel de significancia $\alpha$ si $T^2 > \chi_{\alpha, p}^2$. Debido a que ya contamos con todos los datos necesarios, solo debemos de calcular $T^2$ y $ \chi_{\alpha, p}^2$, lo haremos en R:\\\\
\includegraphics[width=1\textwidth]{img/5.png}\\\\
Debido a que $T^2=292.6722 > \chi_{\alpha, p}^2=14.06714$ entonces rechazamos $H_0$, esto es que hay evidencia de que las medias no son iguales.

\pagebreak
B) Obtener intervalos de confianza (95\%) de diferencia de medias e interpretar resultados.\\
Como es muestra grande calcularemos los intervalos con:
\begin{align*}
(\bar{X_{1i}}-\bar{X_{2i}}) \pm \sqrt{\chi_{\alpha, p}^2 \left[\frac{1}{n_1}s_{ii(1)}+\frac{1}{n_2}s_{ii(2)} \right] }
\end{align*}
Lo calcularemos en R:\\
\includegraphics[width=1\textwidth]{img/6.png}\\\\
Esto es:
\begin{align*}
-17.2501765 &< \mu_1 - \mu_2 < 13.8501765 \\
-20.7756225 &< \mu_1 - \mu_2 < 12.7356225 \\
-0.6489871  &< \mu_1 - \mu_2 < 2.1589871 \\
-1.6828236  &< \mu_1 - \mu_2 < 0.9248236 \\
-5.9186822  &< \mu_1 - \mu_2 < 6.6586822 \\
-6.0141268  &< \mu_1 - \mu_2 < 4.0221268 \\
-1.4152759  &< \mu_1 - \mu_2 < 0.8912759 \\
\end{align*}
Observamos que todos los intervalos van de negativo a positivo, esto quiere decir que todos los intervalos de confianza contienen al cero, lo cual puede sonar contradictorio debido a que en la prueba de hipotesis anterior nos dio que hay evidencia de que las medias no son iguales.

\pagebreak

C) Mencione los supuestos que hiciste.\\
En este caso debido a que es una muestra grande, no es necesario que las dos muestras sigan una distribución normal p-variada debido a que la media muestral se aproximará a una distribución normal. Lo que nosotros tuvimos que asumir es que las observaiones son independientes unas de otras.
\end{sol}
